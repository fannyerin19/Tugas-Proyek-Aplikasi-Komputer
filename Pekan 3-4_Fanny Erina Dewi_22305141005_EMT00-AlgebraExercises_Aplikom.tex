\documentclass{article}

\usepackage{eumat}

\begin{document}
\begin{eulernotebook}
\eulersubheading{Tugas Aplikasi Komputer (Pekan 3-4)}
\begin{eulercomment}
Nama  : Fanny Erina Dewi\\
NIM   : 22305141005\\
Kelas : Matematika B 2022\\
\end{eulercomment}
\eulersubheading{}
\eulerheading{EMT untuk Perhitungan Aljabar}
\begin{eulercomment}
Pada notebook ini Anda belajar menggunakan EMT untuk melakukan
berbagai perhitungan terkait dengan materi atau topik dalam Aljabar.
Kegiatan yang harus Anda lakukan adalah sebagai berikut:

- Membaca secara cermat dan teliti notebook ini;\\
- Menerjemahkan teks bahasa Inggris ke bahasa Indonesia;\\
- Mencoba contoh-contoh perhitungan (perintah EMT) dengan cara
meng-ENTER setiap perintah EMT yang ada (pindahkan kursor ke baris
perintah)\\
- Jika perlu Anda dapat memodifikasi perintah yang ada dan memberikan
keterangan/penjelasan tambahan terkait hasilnya.\\
- Menyisipkan baris-baris perintah baru untuk mengerjakan soal-soal
Aljabar dari file PDF yang saya berikan;\\
- Memberi catatan hasilnya.\\
- Jika perlu tuliskan soalnya pada teks notebook (menggunakan format
LaTeX).\\
- Gunakan tampilan hasil semua perhitungan yang eksak atau simbolik
dengan format LaTeX. (Seperti contoh-contoh pada notebook ini.)

\end{eulercomment}
\eulersubheading{Contoh pertama}
\begin{eulercomment}
Menyederhanakan bentuk aljabar:

\end{eulercomment}
\begin{eulerformula}
\[
6x^{-3}y^5\times -7x^2y^{-9}
\]
\end{eulerformula}
\begin{eulercomment}
\end{eulercomment}
\begin{eulerprompt}
>$&6*x^(-3)*y^5*-7*x^2*y^(-9)
\end{eulerprompt}
\begin{eulerformula}
\[
-\frac{42}{x\,y^4}
\]
\end{eulerformula}
\begin{eulercomment}
CONTOH CONTOH SOAL

MENYEDERHANAKAN
\end{eulercomment}
\begin{eulerprompt}
>$&(8*y^5)*(9*y)
\end{eulerprompt}
\begin{eulerformula}
\[
72\,y^6
\]
\end{eulerformula}
\begin{eulerprompt}
>$& (3*a^2)*(-7*a^4)
\end{eulerprompt}
\begin{eulerformula}
\[
-21\,a^6
\]
\end{eulerformula}
\begin{eulerprompt}
>$&(6*x*y^3)*(9*x^4*y^2)
\end{eulerprompt}
\begin{eulerformula}
\[
54\,x^5\,y^5
\]
\end{eulerformula}
\begin{eulercomment}
Menjabarkan:

\end{eulercomment}
\begin{eulerformula}
\[
(6x^{-3}+y^5)(-7x^2-y^{-9})
\]
\end{eulerformula}
\begin{eulerprompt}
>$&showev('expand((6*x^(-3)+y^5)*(-7*x^2-y^(-9))))
\end{eulerprompt}
\begin{eulerformula}
\[
{\it expand}\left(\left(-\frac{1}{y^9}-7\,x^2\right)\,\left(y^5+
 \frac{6}{x^3}\right)\right)=-7\,x^2\,y^5-\frac{1}{y^4}-\frac{6}{x^3
 \,y^9}-\frac{42}{x}
\]
\end{eulerformula}
\begin{eulerprompt}
>$&expand ((x+3)^2)
\end{eulerprompt}
\begin{eulerformula}
\[
x^2+6\,x+9
\]
\end{eulerformula}
\begin{eulerprompt}
>$&expand ((n+6)*(n-6))
\end{eulerprompt}
\begin{eulerformula}
\[
n^2-36
\]
\end{eulerformula}
\begin{eulerprompt}
>$expand((y-5)^2)
\end{eulerprompt}
\begin{eulerformula}
\[
y^2-10\,y+25
\]
\end{eulerformula}
\begin{eulerprompt}
>$&expand ((5*x-3)^2)
\end{eulerprompt}
\begin{eulerformula}
\[
25\,x^2-30\,x+9
\]
\end{eulerformula}
\eulersubheading{Baris Perintah}
\begin{eulercomment}
Baris perintah Euler terdiri dari satu atau beberapa perintah Euler
diikuti dengan titik koma ";" atau koma ",". Titik koma mencegah
pencetakan hasil. Koma setelah perintah terakhir dapat dihilangkan.

Baris perintah berikut hanya akan mencetak hasil ekspresi, bukan tugas
atau perintah format.
\end{eulercomment}
\begin{eulerprompt}
>r:=2; h:=4; pi*r^2*h/3
\end{eulerprompt}
\begin{euleroutput}
  16.7551608191
\end{euleroutput}
\begin{eulercomment}
Perintah harus dipisahkan dengan tanda kosong. Baris perintah berikut
mencetak dua hasilnya.
\end{eulercomment}
\begin{eulerprompt}
>pi*2*r*h, %+2*pi*r*h // Ingat tanda % menyatakan hasil perhitungan terakhir sebelumnya
\end{eulerprompt}
\begin{euleroutput}
  50.2654824574
  100.530964915
\end{euleroutput}
\begin{eulercomment}
Baris perintah dieksekusi dalam urutan yang ditekan pengguna kembali.
Jadi, Anda mendapatkan nilai baru setiap kali menjalankan baris kedua.
\end{eulercomment}
\begin{eulerprompt}
>x := 1;
>x := cos(x) // nilai cosinus (x dalam radian)
\end{eulerprompt}
\begin{euleroutput}
  0.540302305868
\end{euleroutput}
\begin{eulerprompt}
>x := cos(x)
\end{eulerprompt}
\begin{euleroutput}
  0.857553215846
\end{euleroutput}
\begin{eulercomment}
Jika dua jalur dihubungkan dengan "..." kedua jalur akan selalu
dijalankan secara bersamaan.
\end{eulercomment}
\begin{eulerprompt}
>x := 1.5; ...
>x := (x+2/x)/2, x := (x+2/x)/2, x := (x+2/x)/2, 
\end{eulerprompt}
\begin{euleroutput}
  1.41666666667
  1.41421568627
  1.41421356237
\end{euleroutput}
\begin{eulercomment}
Ini juga cara yang baik untuk menyebarkan perintah panjang ke dua
baris atau lebih. Anda dapat menekan Ctrl + Return untuk membagi garis
menjadi dua pada posisi kursor saat ini, atau Ctlr + Back untuk
menggabungkan garis.

Untuk melipat semua multi-garis tekan Ctrl + L. Kemudian garis
berikutnya hanya akan terlihat, jika salah satunya memiliki fokus.
Untuk melipat satu garis banyak, mulailah baris pertama dengan "\% +".
\end{eulercomment}
\begin{eulerprompt}
>%+ x=4+5; ...
\end{eulerprompt}
\begin{eulercomment}
Garis yang dimulai dengan \%\% akan benar-benar tidak terlihat.
\end{eulercomment}
\begin{euleroutput}
  81
\end{euleroutput}
\begin{eulercomment}
Euler mendukung loop dalam baris perintah, asalkan sesuai dengan satu
baris atau beberapa baris. Dalam program, batasan ini tidak berlaku,
tentu saja. Untuk informasi lebih lanjut, lihat pengantar berikut.
\end{eulercomment}
\begin{eulerprompt}
>x=1; for i=1 to 5; x := (x+2/x)/2, end; // menghitung akar 2
\end{eulerprompt}
\begin{euleroutput}
  1.5
  1.41666666667
  1.41421568627
  1.41421356237
  1.41421356237
\end{euleroutput}
\begin{eulerprompt}
>x := 1.5; // comments go here before the ...
>repeat xnew:=(x+2/x)/2; until xnew~=x; ...
>   x := xnew; ...
>end; ...
>x,
\end{eulerprompt}
\begin{euleroutput}
  1.41421356237
\end{euleroutput}
\begin{eulercomment}
Tidak apa-apa menggunakan multi-garis. Pastikan baris diakhiri dengan
"...".
\end{eulercomment}
\begin{eulerprompt}
>x := 1.5; // komentar diletakkan di sini sebelum ...
>ulang xnew:=(x+2/x)/2; until xnew~=x; ...
>   x := xnew; ...
>end; ...
>x,
\end{eulerprompt}
\begin{euleroutput}
  Variable ulang not found!
  Error in:
  x := 1.5; ulang xnew:=(x+2/x)/2; until xnew~=x;    x := xnew;  ...
                  ^
\end{euleroutput}
\begin{eulercomment}
Struktur bersyarat juga berfungsi.
\end{eulercomment}
\begin{eulerprompt}
>if E^pi>pi^E; then "Thought so!", endif;
\end{eulerprompt}
\begin{euleroutput}
  Thought so!
\end{euleroutput}
\begin{eulerprompt}
>if E^pi>pi^E; then "Berpikir begitu!", berakhir jika;
\end{eulerprompt}
\begin{euleroutput}
  Berpikir begitu!
  Variable berakhir not found!
  Error in:
  if E^pi>pi^E; then "Berpikir begitu!", berakhir jika; ...
                                                  ^
\end{euleroutput}
\begin{eulercomment}
Saat Anda menjalankan perintah, kursor dapat berada di posisi mana pun
di baris perintah. Anda dapat kembali ke perintah sebelumnya atau
melompat ke perintah berikutnya dengan tombol panah. Atau Anda dapat
mengklik bagian komentar di atas perintah untuk membuka perintah.

Saat Anda menggerakkan kursor di sepanjang garis, pasangan buka dan
tutup tanda kurung atau tanda kurung akan disorot. Juga, perhatikan
baris statusnya. Setelah kurung buka dari fungsi sqrt (), baris status
akan menampilkan teks bantuan untuk fungsi tersebut. Jalankan perintah
dengan tombol kembali.
\end{eulercomment}
\begin{eulerprompt}
>sqrt(sin(10°)/cos(20°))
\end{eulerprompt}
\begin{euleroutput}
  0.429875017772
\end{euleroutput}
\begin{eulercomment}
Untuk melihat bantuan untuk perintah terbaru, buka jendela bantuan
dengan F1. Di sana, Anda dapat memasukkan teks untuk dicari. Pada
baris kosong, bantuan untuk jendela bantuan akan ditampilkan. Anda
dapat menekan escape untuk menghapus garis, atau untuk menutup jendela
bantuan.

Anda dapat mengklik dua kali pada perintah apa pun untuk membuka
bantuan untuk perintah ini. Coba klik dua kali perintah exp di bawah
ini di baris perintah.
\end{eulercomment}
\begin{eulerprompt}
>exp(log(2.5))
\end{eulerprompt}
\begin{euleroutput}
  2.5
\end{euleroutput}
\begin{eulercomment}
Anda juga dapat menyalin dan menempel di Euler. Gunakan Ctrl-C dan
Ctrl-V untuk ini. Untuk menandai teks, seret mouse atau gunakan shift
bersama dengan tombol kursor apa pun. Selain itu, Anda dapat menyalin
tanda kurung yang disorot.
\end{eulercomment}
\eulersubheading{Sintaks Dasar}
\begin{eulercomment}
Euler mengetahui fungsi matematika biasa. Seperti yang Anda lihat di
atas, fungsi trigonometri bekerja dalam radian atau derajat. Untuk
mengonversi menjadi derajat, tambahkan simbol derajat (dengan tombol
F7) ke nilainya, atau gunakan fungsi rad (x). Fungsi akar kuadrat
disebut akar kuadrat di Euler. Tentu saja, x \textasciicircum{} (1/2) juga
dimungkinkan.

Untuk menyetel variabel, gunakan "=" atau ": =". Demi kejelasan,
pendahuluan ini menggunakan bentuk yang terakhir. Spasi tidak penting.
Tapi ruang antar perintah diharapkan.

Beberapa perintah dalam satu baris dipisahkan dengan "," atau ";".
Titik koma menekan keluaran dari perintah. Di akhir baris perintah,
"," diasumsikan, jika ";" hilang.
\end{eulercomment}
\begin{eulerprompt}
>g:=9.81; t:=2.5; 1/2*g*t^2
\end{eulerprompt}
\begin{euleroutput}
  30.65625
\end{euleroutput}
\begin{eulercomment}
EMT menggunakan sintaks pemrograman untuk ekspresi. Memasuki

\end{eulercomment}
\begin{eulerformula}
\[
e ^ 2(\frac{1}{3+4\log(0.6)}+\frac{1}{7})
\]
\end{eulerformula}
\begin{eulercomment}
Anda harus mengatur tanda kurung yang benar dan menggunakan / untuk
pecahan. Perhatikan tanda kurung yang disorot untuk bantuan.
Perhatikan bahwa konstanta Euler e dinamai E dalam EMT.
\end{eulercomment}
\begin{eulerprompt}
>E^2*(1/(3+4*log(0.6))+1/7)
\end{eulerprompt}
\begin{euleroutput}
  8.77908249441
\end{euleroutput}
\begin{eulercomment}
Untuk menghitung ekspresi yang rumit seperti

\end{eulercomment}
\begin{eulerformula}
\[
(\frac{\frac 17+\frac 18 + 2}{\frac 13+\frac 12})^ 2\ pi
\]
\end{eulerformula}
\begin{eulercomment}
Anda harus memasukkannya dalam bentuk baris.
\end{eulercomment}
\begin{eulerprompt}
>((1/7 + 1/8 + 2) / (1/3 + 1/2))^2 * pi
\end{eulerprompt}
\begin{euleroutput}
  23.2671801626
\end{euleroutput}
\begin{eulercomment}
Tempatkan tanda kurung dengan hati-hati di sekitar sub-ekspresi yang
perlu dihitung terlebih dahulu. EMT membantu Anda dengan menyoroti
ekspresi bahwa kurung tutup selesai. Anda juga harus memasukkan nama
"pi" untuk huruf Yunani pi.

Hasil dari perhitungan ini adalah bilangan floating point. Ini secara
default dicetak dengan akurasi sekitar 12 digit. Di baris perintah
berikut, kita juga belajar bagaimana kita bisa merujuk ke hasil
sebelumnya dalam baris yang sama.
\end{eulercomment}
\begin{eulerprompt}
>1/3+1/7, fraction %
\end{eulerprompt}
\begin{euleroutput}
  0.47619047619
  10/21
\end{euleroutput}
\begin{eulercomment}
Perintah Euler bisa berupa ekspresi atau perintah primitif. Ekspresi
dibuat dari operator dan fungsi. Jika perlu, itu harus berisi tanda
kurung untuk memaksa urutan eksekusi yang benar. Jika ragu, menetapkan
braket adalah ide yang bagus. Perhatikan bahwa EMT menampilkan tanda
kurung buka dan tutup saat mengedit baris perintah.
\end{eulercomment}
\begin{eulerprompt}
>(cos(pi/4)+1)^3*(sin(pi/4)+1)^2
\end{eulerprompt}
\begin{euleroutput}
  14.4978445072
\end{euleroutput}
\begin{eulercomment}
Operator numerik Euler termasuk

\end{eulercomment}
\begin{eulerttcomment}
 + unary atau operator plus
 - unary atau operator minus
 *, /
 . produk matriks
 a ^ b pangkat untuk positif a atau bilangan bulat b (a ** b bekerja
\end{eulerttcomment}
\begin{eulercomment}
juga)\\
\end{eulercomment}
\begin{eulerttcomment}
 n! operator faktorial
\end{eulerttcomment}
\begin{eulercomment}

dan masih banyak lagi.

Berikut beberapa fungsi yang mungkin Anda perlukan. Masih banyak lagi.

\end{eulercomment}
\begin{eulerttcomment}
 sin, cos, tan, atan, asin, acos, rad, deg
 log, exp, log10, sqrt, logbase
 bin, logbin, logfac, mod, floor, ceil, round, abs, sign
 konj, re, im, arg, konj, nyata, kompleks
 beta, betai, gamma, complexgamma, ellrf, ellf, ellrd, elle
 bitand, bitor, bitxor, bitnot
\end{eulerttcomment}
\begin{eulercomment}

Beberapa perintah memiliki alias, mis. ln untuk log.
\end{eulercomment}
\begin{eulerprompt}
>ln(E^2), arctan(tan(0.5))
\end{eulerprompt}
\begin{euleroutput}
  2
  0.5
\end{euleroutput}
\begin{eulerprompt}
>sin(30°)
\end{eulerprompt}
\begin{euleroutput}
  0.5
\end{euleroutput}
\begin{eulercomment}
Pastikan untuk menggunakan tanda kurung (tanda kurung bulat), setiap
kali ada keraguan tentang urutan eksekusi! Berikut ini tidak sama
dengan (2 \textasciicircum{} 3) \textasciicircum{} 4, yang merupakan default untuk 2 \textasciicircum{} 3 \textasciicircum{} 4 di EMT
(beberapa sistem numerik melakukannya dengan cara lain).
\end{eulercomment}
\begin{eulerprompt}
>2^3^4, (2^3)^4, 2^(3^4)
\end{eulerprompt}
\begin{euleroutput}
  2.41785163923e+24
  4096
  2.41785163923e+24
\end{euleroutput}
\eulersubheading{Bilangan Nyata}
\begin{eulercomment}
Tipe data utama di Euler adalah bilangan real. Real direpresentasikan
dalam format IEEE dengan akurasi sekitar 16 digit desimal.
\end{eulercomment}
\begin{eulerprompt}
>longest 1/3
\end{eulerprompt}
\begin{euleroutput}
       0.3333333333333333 
\end{euleroutput}
\begin{eulercomment}
Representasi ganda internal membutuhkan 8 byte.
\end{eulercomment}
\begin{eulerprompt}
>printdual(1/3)
\end{eulerprompt}
\begin{euleroutput}
  1.0101010101010101010101010101010101010101010101010101*2^-2
\end{euleroutput}
\begin{eulerprompt}
>printhex(1/3)
\end{eulerprompt}
\begin{euleroutput}
  5.5555555555554*16^-1
\end{euleroutput}
\eulersubheading{String}
\begin{eulercomment}
Sebuah string di Euler didefinisikan dengan "...".
\end{eulercomment}
\begin{eulerprompt}
>"Sebuah string bisa berisi apa saja."
\end{eulerprompt}
\begin{euleroutput}
  Sebuah string bisa berisi apa saja.
\end{euleroutput}
\begin{eulercomment}
String bisa digabungkan dengan \textbar{} atau dengan +. Ini juga berfungsi
dengan angka, yang diubah menjadi string dalam kasus itu.
\end{eulercomment}
\begin{eulerprompt}
>"The area of the circle with radius " + 2 + " cm is " + pi*4 + " cm^2."
\end{eulerprompt}
\begin{euleroutput}
  The area of the circle with radius 2 cm is 12.5663706144 cm^2.
\end{euleroutput}
\begin{eulercomment}
Fungsi cetak juga mengubah angka menjadi string. Ini bisa mengambil
sejumlah digit dan sejumlah tempat (0 untuk output padat), dan secara
optimal satu unit.
\end{eulercomment}
\begin{eulerprompt}
>"Golden Ratio : " + print((1+sqrt(5))/2,5,0)
\end{eulerprompt}
\begin{euleroutput}
  Golden Ratio : 1.61803
\end{euleroutput}
\begin{eulercomment}
Tidak ada string khusus, yang tidak dicetak. Itu dikembalikan oleh
beberapa fungsi, ketika hasilnya tidak penting. (Ini dikembalikan
secara otomatis, jika fungsi tidak memiliki pernyataan pengembalian.)
\end{eulercomment}
\begin{eulerprompt}
>none
\end{eulerprompt}
\begin{eulercomment}
Untuk mengonversi string menjadi angka, cukup evaluasi. Ini berfungsi
untuk ekspresi juga (lihat di bawah).
\end{eulercomment}
\begin{eulerprompt}
>"1234.5"()
\end{eulerprompt}
\begin{euleroutput}
  1234.5
\end{euleroutput}
\begin{eulercomment}
Untuk mendefinisikan vektor string, gunakan notasi vektor [...].
\end{eulercomment}
\begin{eulerprompt}
>v:=["affe","charlie","bravo"]
\end{eulerprompt}
\begin{euleroutput}
  affe
  charlie
  bravo
\end{euleroutput}
\begin{eulercomment}
Vektor string kosong dilambangkan dengan [tidak ada]. Vektor string
dapat digabungkan.
\end{eulercomment}
\begin{eulerprompt}
>w:=[none]; w|v|v
\end{eulerprompt}
\begin{euleroutput}
  affe
  charlie
  bravo
  affe
  charlie
  bravo
\end{euleroutput}
\begin{eulercomment}
String dapat berisi karakter Unicode. Secara internal, string ini
berisi kode UTF-8. Untuk menghasilkan string seperti itu, gunakan u
"..." dan salah satu entitas HTML.

String unicode dapat digabungkan seperti string lainnya.
\end{eulercomment}
\begin{eulerprompt}
>u"&alpha; = " + 45 + u"&deg;" // pdfLaTeX mungkin gagal menampilkan secara benar
\end{eulerprompt}
\begin{euleroutput}
  α = 45°
\end{euleroutput}
\begin{eulercomment}
I
\end{eulercomment}
\begin{eulercomment}
Di komentar, entitas yang sama seperti \& alpha ;, \& beta; dll. dapat
digunakan. Ini mungkin alternatif cepat untuk Latex. (Lebih detail
tentang komentar di bawah).
\end{eulercomment}
\begin{eulercomment}
Ada beberapa fungsi untuk membuat atau menganalisis string unicode.
Fungsi strtochar () akan mengenali string Unicode, dan
menerjemahkannya dengan benar.
\end{eulercomment}
\begin{eulerprompt}
>v=strtochar(u"&Auml; is a German letter")
\end{eulerprompt}
\begin{euleroutput}
  [196,  32,  105,  115,  32,  97,  32,  71,  101,  114,  109,  97,  110,
  32,  108,  101,  116,  116,  101,  114]
\end{euleroutput}
\begin{eulercomment}
Hasilnya adalah vektor bilangan Unicode. Fungsi kebalikannya adalah
chartoutf ().
\end{eulercomment}
\begin{eulerprompt}
>v[1]=strtochar(u"&Uuml;")[1]; chartoutf(v)
\end{eulerprompt}
\begin{euleroutput}
  Ü is a German letter
\end{euleroutput}
\begin{eulercomment}
Fungsi utf () dapat menerjemahkan string dengan entitas dalam variabel
menjadi string Unicode.
\end{eulercomment}
\begin{eulerprompt}
>s="We have &alpha;=&beta;."; utf(s) // pdfLaTeX mungkin gagal menampilkan secara benar
\end{eulerprompt}
\begin{euleroutput}
  We have α=β.
\end{euleroutput}
\begin{eulercomment}
Dimungkinkan juga untuk menggunakan entitas numerik.
\end{eulercomment}
\begin{eulerprompt}
>u"&#196;hnliches"
\end{eulerprompt}
\begin{euleroutput}
  Ähnliches
\end{euleroutput}
\eulersubheading{Nilai Boolean}
\begin{eulercomment}
Nilai Boolean diwakili dengan 1 = true atau 0 = false di Euler. String
dapat dibandingkan, seperti halnya angka.
\end{eulercomment}
\begin{eulerprompt}
>2<1, "apel"<"banana"
\end{eulerprompt}
\begin{euleroutput}
  0
  1
\end{euleroutput}
\begin{eulercomment}
"dan" adalah operator "\&\&" dan "atau" adalah operator "\textbar{}\textbar{}", seperti
dalam bahasa C. (Kata "dan" dan "atau" hanya dapat digunakan dalam
kondisi untuk "jika".)
\end{eulercomment}
\begin{eulerprompt}
>2<E && E<3
\end{eulerprompt}
\begin{euleroutput}
  1
\end{euleroutput}
\begin{eulercomment}
Operator Boolean mematuhi aturan bahasa matriks.
\end{eulercomment}
\begin{eulerprompt}
>(1:10)>5, nonzeros(%)
\end{eulerprompt}
\begin{euleroutput}
  [0,  0,  0,  0,  0,  1,  1,  1,  1,  1]
  [6,  7,  8,  9,  10]
\end{euleroutput}
\begin{eulercomment}
Anda dapat menggunakan fungsi nonzeros () untuk mengekstrak elemen
tertentu dari vektor. Dalam contoh, kami menggunakan isprime bersyarat
(n).
\end{eulercomment}
\begin{eulerprompt}
>N=2|3:2:99 // N berisi elemen 2 dan bilangan2 ganjil dari 3 s.d. 99
\end{eulerprompt}
\begin{euleroutput}
  [2,  3,  5,  7,  9,  11,  13,  15,  17,  19,  21,  23,  25,  27,  29,
  31,  33,  35,  37,  39,  41,  43,  45,  47,  49,  51,  53,  55,  57,
  59,  61,  63,  65,  67,  69,  71,  73,  75,  77,  79,  81,  83,  85,
  87,  89,  91,  93,  95,  97,  99]
\end{euleroutput}
\begin{eulerprompt}
>N[nonzeros(isprime(N))] //pilih anggota2 N yang prima
\end{eulerprompt}
\begin{euleroutput}
  [2,  3,  5,  7,  11,  13,  17,  19,  23,  29,  31,  37,  41,  43,  47,
  53,  59,  61,  67,  71,  73,  79,  83,  89,  97]
\end{euleroutput}
\eulersubheading{Format Keluaran}
\begin{eulercomment}
Format keluaran default EMT mencetak 12 digit. Untuk memastikan bahwa
kami melihat default, kami mengatur ulang format.
\end{eulercomment}
\begin{eulerprompt}
>defformat; pi
\end{eulerprompt}
\begin{euleroutput}
  3.14159265359
\end{euleroutput}
\begin{eulercomment}
Secara internal, EMT menggunakan standar IEEE untuk bilangan ganda
dengan sekitar 16 digit desimal. Untuk melihat jumlah digit secara
lengkap, gunakan perintah "longestformat", atau gunakan operator
"longest" untuk menampilkan hasil dalam format terpanjang.
\end{eulercomment}
\begin{eulerprompt}
>longest pi
\end{eulerprompt}
\begin{euleroutput}
        3.141592653589793 
\end{euleroutput}
\begin{eulercomment}
Berikut adalah representasi heksadesimal internal dari bilangan ganda.
\end{eulercomment}
\begin{eulerprompt}
>printhex(pi)
\end{eulerprompt}
\begin{euleroutput}
  3.243F6A8885A30*16^0
\end{euleroutput}
\begin{eulercomment}
Format keluaran dapat diubah secara permanen dengan perintah format.
\end{eulercomment}
\begin{eulerprompt}
>format(12,5); 1/3, pi, sin(1)
\end{eulerprompt}
\begin{euleroutput}
      0.33333 
      3.14159 
      0.84147 
\end{euleroutput}
\begin{eulercomment}
Standarnya adalah format (12).
\end{eulercomment}
\begin{eulerprompt}
>format(12); 1/3
\end{eulerprompt}
\begin{euleroutput}
  0.333333333333
\end{euleroutput}
\begin{eulercomment}
Fungsi seperti "shortestformat", "shortformat", "longformat" bekerja
untuk vektor dengan cara berikut.
\end{eulercomment}
\begin{eulerprompt}
>shortestformat; random(3,8)
\end{eulerprompt}
\begin{euleroutput}
    0.66    0.2   0.89   0.28   0.53   0.31   0.44    0.3 
    0.28   0.88   0.27    0.7   0.22   0.45   0.31   0.91 
    0.19   0.46  0.095    0.6   0.43   0.73   0.47   0.32 
\end{euleroutput}
\begin{eulercomment}
Format default untuk skalar adalah format (12). Tapi ini bisa diubah.
\end{eulercomment}
\begin{eulerprompt}
>setscalarformat(5); pi
\end{eulerprompt}
\begin{euleroutput}
  3.1416
\end{euleroutput}
\begin{eulercomment}
Fungsi "format terpanjang" mengatur format skalar juga.
\end{eulercomment}
\begin{eulerprompt}
>longestformat; pi
\end{eulerprompt}
\begin{euleroutput}
  3.141592653589793
\end{euleroutput}
\begin{eulercomment}
Sebagai referensi, berikut adalah daftar format keluaran terpenting.

\end{eulercomment}
\begin{eulerttcomment}
 format terpendek format pendek format panjang, format terpanjang
 format (panjang, digit) format yang baik (panjang)
 fracformat (panjang)
 defformat
\end{eulerttcomment}
\begin{eulercomment}

Akurasi internal EMT adalah sekitar 16 tempat desimal, yang merupakan
standar IEEE. Angka disimpan dalam format internal ini.

Tetapi format keluaran EMT dapat diatur dengan cara yang fleksibel.
\end{eulercomment}
\begin{eulerprompt}
>longestformat; pi,
\end{eulerprompt}
\begin{euleroutput}
  3.141592653589793
\end{euleroutput}
\begin{eulerprompt}
>format(10,5); pi
\end{eulerprompt}
\begin{euleroutput}
    3.14159 
\end{euleroutput}
\begin{eulercomment}
Standarnya adalah defformat ().
\end{eulercomment}
\begin{eulerprompt}
>defformat; // default
\end{eulerprompt}
\begin{eulercomment}
Ada operator pendek yang hanya mencetak satu nilai. Operator
"terpanjang" akan mencetak semua digit nomor yang valid.
\end{eulercomment}
\begin{eulerprompt}
>longest pi^2/2
\end{eulerprompt}
\begin{euleroutput}
        4.934802200544679 
\end{euleroutput}
\begin{eulercomment}
Ada juga operator singkat untuk mencetak hasil dalam format pecahan.
Kami telah menggunakannya di atas.
\end{eulercomment}
\begin{eulerprompt}
>fraction 1+1/2+1/3+1/4
\end{eulerprompt}
\begin{euleroutput}
  25/12
\end{euleroutput}
\begin{eulercomment}
Karena format internal menggunakan cara biner untuk menyimpan angka,
nilai 0.1 tidak akan direpresentasikan dengan tepat. Kesalahan
bertambah sedikit, seperti yang Anda lihat dalam perhitungan berikut.
\end{eulercomment}
\begin{eulerprompt}
>longest 0.1+0.1+0.1+0.1+0.1+0.1+0.1+0.1+0.1+0.1-1
\end{eulerprompt}
\begin{euleroutput}
   -1.110223024625157e-16 
\end{euleroutput}
\begin{eulercomment}
Tetapi dengan "longformat" default Anda tidak akan melihat ini. Untuk
kenyamanan, keluaran angka yang sangat kecil adalah 0.
\end{eulercomment}
\begin{eulerprompt}
>0.1+0.1+0.1+0.1+0.1+0.1+0.1+0.1+0.1+0.1-1
\end{eulerprompt}
\begin{euleroutput}
  0
\end{euleroutput}
\eulerheading{Ekspresi}
\begin{eulercomment}
String atau nama dapat digunakan untuk menyimpan ekspresi matematika,
yang dapat dievaluasi oleh EMT. Untuk ini, gunakan tanda kurung
setelah ekspresi. Jika Anda bermaksud menggunakan string sebagai
ekspresi, gunakan konvensi untuk menamainya "fx" atau "fxy" dll.
Ekspresi lebih diutamakan daripada fungsi.

Variabel global dapat digunakan dalam evaluasi.
\end{eulercomment}
\begin{eulerprompt}
>r:=2; fx:="pi*r^2"; longest fx()
\end{eulerprompt}
\begin{euleroutput}
        12.56637061435917 
\end{euleroutput}
\begin{eulercomment}
Parameter ditetapkan ke x, y, dan z dalam urutan itu. Parameter
tambahan dapat ditambahkan menggunakan parameter yang ditetapkan.
\end{eulercomment}
\begin{eulerprompt}
>fx:="a*sin(x)^2"; fx(5,a=-1)
\end{eulerprompt}
\begin{euleroutput}
  -0.919535764538
\end{euleroutput}
\begin{eulercomment}
Perhatikan bahwa ekspresi akan selalu menggunakan variabel global,
meskipun ada variabel dalam fungsi dengan nama yang sama. (Jika tidak,
evaluasi ekspresi dalam fungsi dapat memiliki hasil yang sangat
membingungkan bagi pengguna yang memanggil fungsi tersebut.)
\end{eulercomment}
\begin{eulerprompt}
>at:=4; function f(expr,x,at) := expr(x); ...
>f("at*x^2",3,5) // computes 4*3^2 not 5*3^2
\end{eulerprompt}
\begin{euleroutput}
  36
\end{euleroutput}
\begin{eulercomment}
Jika Anda ingin menggunakan nilai lain untuk "at" daripada nilai
global, Anda perlu menambahkan "at = value".
\end{eulercomment}
\begin{eulerprompt}
>at:=4; function f(expr,x,a) := expr(x,at=a); ...
>f("at*x^2",3,5)
\end{eulerprompt}
\begin{euleroutput}
  45
\end{euleroutput}
\begin{eulercomment}
Sebagai referensi, kami berkomentar bahwa koleksi panggilan (dibahas
di tempat lain) dapat berisi ekspresi. Jadi contoh diatas bisa kita
buat sebagai berikut.
\end{eulercomment}
\begin{eulerprompt}
>at:=4; function f(expr,x) := expr(x); ...
>f(\{\{"at*x^2",at=5\}\},3)
\end{eulerprompt}
\begin{euleroutput}
  45
\end{euleroutput}
\begin{eulercomment}
Ekspresi dalam x sering digunakan seperti halnya fungsi.\\
Perhatikan bahwa mendefinisikan fungsi dengan nama yang sama seperti
ekspresi simbolik global akan menghapus variabel ini untuk menghindari
kebingungan antara ekspresi simbolik dan fungsi.
\end{eulercomment}
\begin{eulerprompt}
>f &= 5*x;
>function f(x) := 6*x;
>f(2)
\end{eulerprompt}
\begin{euleroutput}
  12
\end{euleroutput}
\begin{eulercomment}
Dengan cara konvensi, ekspresi simbolik atau numerik harus diberi nama
fx, fxy dll. Skema penamaan ini tidak boleh digunakan untuk fungsi.
\end{eulercomment}
\begin{eulerprompt}
>fx &= diff(x^x,x); $&fx
\end{eulerprompt}
\begin{eulerformula}
\[
x^{x}\,\left(\log x+1\right)
\]
\end{eulerformula}
\begin{eulercomment}
Bentuk ekspresi khusus memungkinkan variabel apa pun sebagai parameter
tanpa nama untuk mengevaluasi ekspresi, tidak hanya "x", "y", dll.
Untuk ini, mulailah ekspresi dengan "@ (variabel) ...".
\end{eulercomment}
\begin{eulerprompt}
>"@(a,b) a^2+b^2", %(4,5)
\end{eulerprompt}
\begin{euleroutput}
  @(a,b) a^2+b^2
  41
\end{euleroutput}
\begin{eulercomment}
Hal ini memungkinkan untuk memanipulasi ekspresi dalam variabel lain
untuk fungsi EMT yang membutuhkan ekspresi dalam "x".

Cara paling dasar untuk mendefinisikan fungsi sederhana adalah dengan
menyimpan rumusnya dalam ekspresi simbolik atau numerik. Jika variabel
utamanya adalah x, ekspresi tersebut dapat dievaluasi seperti fungsi.

Seperti yang Anda lihat pada contoh berikut, variabel global terlihat
selama evaluasi.
\end{eulercomment}
\begin{eulerprompt}
>fx &= x^3-a*x;  ...
>a=1.2; fx(0.5)
\end{eulerprompt}
\begin{euleroutput}
  -0.475
\end{euleroutput}
\begin{eulercomment}
Semua variabel lain dalam ekspresi dapat ditentukan dalam evaluasi
menggunakan parameter yang ditetapkan.
\end{eulercomment}
\begin{eulerprompt}
>fx(0.5,a=1.1)
\end{eulerprompt}
\begin{euleroutput}
  -0.425
\end{euleroutput}
\begin{eulercomment}
Ekspresi tidak perlu simbolis. Ini diperlukan, jika ekspresi berisi
fungsi, yang hanya dikenal di kernel numerik, bukan di Maxima.

\begin{eulercomment}
\eulerheading{Matematika Simbolis}
\begin{eulercomment}
EMT melakukan matematika simbolis dengan bantuan Maxima. Untuk
detailnya, mulailah dengan tutorial berikut, atau telusuri referensi
untuk Maxima. Para ahli di Maxima harus memperhatikan bahwa ada
perbedaan dalam sintaks antara sintaks asli dari Maxima dan sintaks
default dari ekspresi simbolik di EMT.

Matematika simbolik terintegrasi mulus ke dalam Euler dengan \&.
Ekspresi apa pun yang dimulai dengan \& adalah ekspresi simbolis. Itu
dievaluasi dan dicetak oleh Maxima.

Pertama-tama, Maxima memiliki aritmatika "tak terbatas" yang dapat
menangani angka yang sangat besar.
\end{eulercomment}
\begin{eulerprompt}
>$&44!
\end{eulerprompt}
\begin{eulerformula}
\[
2658271574788448768043625811014615890319638528000000000
\]
\end{eulerformula}
\begin{eulercomment}
Dengan cara ini, Anda dapat menghitung hasil yang besar dengan tepat.
Mari kita hitung

\end{eulercomment}
\begin{eulerformula}
\[
C (44,10) = \frac{44!}{34!\cdot10!}
\]
\end{eulerformula}
\begin{eulerprompt}
>$& 44!/(34!*10!) // nilai C(44,10)
\end{eulerprompt}
\begin{eulerformula}
\[
2481256778
\]
\end{eulerformula}
\begin{eulercomment}
Tentu saja, Maxima memiliki fungsi yang lebih efisien untuk ini
(seperti halnya bagian numerik EMT).
\end{eulercomment}
\begin{eulerprompt}
>$binomial(44,10) //menghitung C(44,10) menggunakan fungsi binomial()
\end{eulerprompt}
\begin{eulerformula}
\[
2481256778
\]
\end{eulerformula}
\begin{eulercomment}
Untuk mempelajari lebih lanjut tentang fungsi tertentu, klik dua kali
di atasnya. Misalnya, coba klik dua kali pada "\& binomial" di baris
perintah sebelumnya. Ini membuka dokumentasi Maxima yang disediakan
oleh penulis program itu.

Anda akan belajar bahwa yang berikut ini juga berfungsi.

\end{eulercomment}
\begin{eulerformula}
\[
C (x, 3) = \frac {x!} {(x-3)! 3!} = \frac {(x-2) (x-1) x} {6}
\]
\end{eulerformula}
\begin{eulerprompt}
>$binomial(x,3) // C(x,3)
\end{eulerprompt}
\begin{eulerformula}
\[
\frac{\left(x-2\right)\,\left(x-1\right)\,x}{6}
\]
\end{eulerformula}
\begin{eulercomment}
Jika Anda ingin mengganti x dengan nilai tertentu, gunakan "dengan".
\end{eulercomment}
\begin{eulerprompt}
>$&binomial(x,3) with x=10 // substitusi x=10 ke C(x,3)
\end{eulerprompt}
\begin{eulerformula}
\[
120
\]
\end{eulerformula}
\begin{eulercomment}
Dengan begitu, Anda bisa menggunakan solusi persamaan di persamaan
lain.

Ekspresi simbolik dicetak oleh Maxima dalam bentuk 2D. Alasannya
adalah adanya bendera simbolis khusus dalam string tersebut.

Seperti yang akan Anda lihat pada contoh sebelumnya dan berikut, jika
Anda menginstal LaTeX, Anda dapat mencetak ekspresi simbolik dengan
Latex. Jika tidak, perintah berikut akan mengeluarkan pesan kesalahan.

Untuk mencetak ekspresi simbolik dengan LaTeX, gunakan \textdollar{} infront dari
\& (atau Anda dapat menghilangkan \&) sebelum perintah. Jangan
menjalankan perintah Maxima dengan \textdollar{}, jika Anda belum menginstal
LaTeX.
\end{eulercomment}
\begin{eulerprompt}
>$(3+x)/(x^2+1)
\end{eulerprompt}
\begin{eulerformula}
\[
\frac{x+3}{x^2+1}
\]
\end{eulerformula}
\begin{eulercomment}
Ekspresi simbolik diurai oleh Euler. Jika Anda membutuhkan sintaks
kompleks dalam satu ekspresi, Anda dapat mengapit ekspresi dalam
"...". Menggunakan lebih dari sekedar ekspresi sederhana dimungkinkan,
tetapi sangat tidak disarankan.
\end{eulercomment}
\begin{eulerprompt}
>&"v := 5; v^2"
\end{eulerprompt}
\begin{euleroutput}
  
                                    25
  
\end{euleroutput}
\begin{eulercomment}
Untuk kelengkapan, kami menyatakan bahwa ekspresi simbolik dapat
digunakan dalam program, tetapi perlu diapit tanda kutip. Selain itu,
jauh lebih efektif untuk memanggil Maxima pada waktu kompilasi jika
memungkinkan.
\end{eulercomment}
\begin{eulerprompt}
>$&expand((1+x)^4), $&factor(diff(%,x)) // diff: turunan, factor: faktor
\end{eulerprompt}
\begin{eulerformula}
\[
x^4+4\,x^3+6\,x^2+4\,x+1
\]
\end{eulerformula}
\begin{eulerformula}
\[
4\,\left(x+1\right)^3
\]
\end{eulerformula}
\begin{eulercomment}
Sekali lagi,\% mengacu pada hasil sebelumnya.

Untuk mempermudah, kami menyimpan solusi ke variabel simbolik.
Variabel simbolik didefinisikan dengan "\& =".
\end{eulercomment}
\begin{eulerprompt}
>fx &= (x+1)/(x^4+1); $&fx
\end{eulerprompt}
\begin{eulerformula}
\[
\frac{x+1}{x^4+1}
\]
\end{eulerformula}
\begin{eulercomment}
Ekspresi simbolik dapat digunakan dalam ekspresi simbolik lainnya.
\end{eulercomment}
\begin{eulerprompt}
>$&factor(diff(fx,x))
\end{eulerprompt}
\begin{eulerformula}
\[
\frac{-3\,x^4-4\,x^3+1}{\left(x^4+1\right)^2}
\]
\end{eulerformula}
\begin{eulercomment}
Masukan langsung dari perintah Maxima juga tersedia. Mulai baris
perintah dengan "::". Sintaks Maxima disesuaikan dengan sintaks EMT
(disebut "mode kompatibilitas").
\end{eulercomment}
\begin{eulerprompt}
>&factor(20!)
\end{eulerprompt}
\begin{euleroutput}
  
                           2432902008176640000
  
\end{euleroutput}
\begin{eulerprompt}
>::: factor(10!)
\end{eulerprompt}
\begin{euleroutput}
  
                                 8  4  2
                                2  3  5  7
  
\end{euleroutput}
\begin{eulerprompt}
>:: factor(20!)
\end{eulerprompt}
\begin{euleroutput}
  
                          18  8  4  2
                         2   3  5  7  11 13 17 19
  
\end{euleroutput}
\begin{eulercomment}
Jika Anda ahli dalam Maxima, Anda mungkin ingin menggunakan sintaks
asli Maxima. Anda dapat melakukan ini dengan ":::".
\end{eulercomment}
\begin{eulerprompt}
>::: av:g$ av^2;
\end{eulerprompt}
\begin{euleroutput}
  
                                     2
                                    g
  
\end{euleroutput}
\begin{eulerprompt}
>fx &= x^3*exp(x), $fx
\end{eulerprompt}
\begin{euleroutput}
  
                                   3  x
                                  x  E
  
\end{euleroutput}
\begin{eulerformula}
\[
x^3\,e^{x}
\]
\end{eulerformula}
\begin{eulercomment}
Variabel semacam itu dapat digunakan dalam ekspresi simbolik lainnya.
Perhatikan, bahwa dalam perintah berikut, sisi kanan \& = dievaluasi
sebelum penugasan ke Fx.
\end{eulercomment}
\begin{eulerprompt}
>&(fx with x=5), $%, &float(%)
\end{eulerprompt}
\begin{euleroutput}
  
                                       5
                                  125 E
  
\end{euleroutput}
\begin{eulerformula}
\[
125\,e^5
\]
\end{eulerformula}
\begin{euleroutput}
  
                            18551.64488782208
  
\end{euleroutput}
\begin{eulerprompt}
>fx(5)
\end{eulerprompt}
\begin{euleroutput}
  18551.6448878
\end{euleroutput}
\begin{eulercomment}
Untuk evaluasi ekspresi dengan nilai variabel tertentu, Anda dapat
menggunakan operator "dengan".

Baris perintah berikut juga menunjukkan bahwa Maxima bisa mengevaluasi
ekspresi secara numerik dengan float ().
\end{eulercomment}
\begin{eulerprompt}
>&(fx with x=10)-(fx with x=5), &float(%)
\end{eulerprompt}
\begin{euleroutput}
  
                                  10        5
                            1000 E   - 125 E
  
  
                           2.20079141499189e+7
  
\end{euleroutput}
\begin{eulerprompt}
>$factor(diff(fx,x,2))
\end{eulerprompt}
\begin{eulerformula}
\[
x\,\left(x^2+6\,x+6\right)\,e^{x}
\]
\end{eulerformula}
\begin{eulercomment}
Untuk mendapatkan kode Latex untuk ekspresi, Anda dapat menggunakan
perintah tex.
\end{eulercomment}
\begin{eulerprompt}
>tex(fx)
\end{eulerprompt}
\begin{euleroutput}
  x^3\(\backslash\),e^\{x\}
\end{euleroutput}
\begin{eulercomment}
Ekspresi simbolik dapat dievaluasi seperti ekspresi numerik.
\end{eulercomment}
\begin{eulerprompt}
>fx(0.5)
\end{eulerprompt}
\begin{euleroutput}
  0.206090158838
\end{euleroutput}
\begin{eulercomment}
Dalam ekspresi simbolik, ini tidak berfungsi, karena Maxima tidak
mendukungnya. Sebagai gantinya, gunakan sintaks "with" (bentuk yang
lebih bagus dari perintah at (...) Maxima).
\end{eulercomment}
\begin{eulerprompt}
>$&fx with x=1/2
\end{eulerprompt}
\begin{eulerformula}
\[
\frac{\sqrt{e}}{8}
\]
\end{eulerformula}
\begin{eulercomment}
Penugasan juga bisa bersifat simbolis.
\end{eulercomment}
\begin{eulerprompt}
>$&fx with x=1+t
\end{eulerprompt}
\begin{eulerformula}
\[
\left(t+1\right)^3\,e^{t+1}
\]
\end{eulerformula}
\begin{eulercomment}
Perintah memecahkan memecahkan ekspresi simbolik untuk variabel di
Maxima. Hasilnya adalah vektor solusi.
\end{eulercomment}
\begin{eulerprompt}
>$&solve(x^2+x=4,x)
\end{eulerprompt}
\begin{eulerformula}
\[
\left[ x=\frac{-\sqrt{17}-1}{2} , x=\frac{\sqrt{17}-1}{2} \right] 
\]
\end{eulerformula}
\begin{eulercomment}
Bandingkan dengan perintah "selesaikan" numerik di Euler, yang
membutuhkan nilai awal, dan secara opsional nilai target.
\end{eulercomment}
\begin{eulerprompt}
>solve("x^2+x",1,y=4)
\end{eulerprompt}
\begin{euleroutput}
  1.56155281281
\end{euleroutput}
\begin{eulercomment}
Nilai numerik dari solusi simbolik dapat dihitung dengan evaluasi
hasil simbolik. Euler akan membaca tugas x = dll. Jika Anda tidak
memerlukan hasil numerik untuk perhitungan lebih lanjut, Anda juga
dapat membiarkan Maxima menemukan nilai numerik.
\end{eulercomment}
\begin{eulerprompt}
>sol &= solve(x^2+2*x=4,x); $&sol, sol(), $&float(sol)
\end{eulerprompt}
\begin{eulerformula}
\[
\left[ x=-\sqrt{5}-1 , x=\sqrt{5}-1 \right] 
\]
\end{eulerformula}
\begin{euleroutput}
  [-3.23607,  1.23607]
\end{euleroutput}
\begin{eulerformula}
\[
\left[ x=-3.23606797749979 , x=1.23606797749979 \right] 
\]
\end{eulerformula}
\begin{eulercomment}
Untuk mendapatkan solusi simbolik tertentu, seseorang dapat
menggunakan "dengan" dan indeks.
\end{eulercomment}
\begin{eulerprompt}
>$&solve(x^2+x=1,x), x2 &= x with %[2]; $&x2
\end{eulerprompt}
\begin{eulerformula}
\[
\left[ x=\frac{-\sqrt{5}-1}{2} , x=\frac{\sqrt{5}-1}{2} \right] 
\]
\end{eulerformula}
\begin{eulerformula}
\[
\frac{\sqrt{5}-1}{2}
\]
\end{eulerformula}
\begin{eulercomment}
Untuk menyelesaikan sistem persamaan, gunakan vektor persamaan.
Hasilnya adalah vektor solusi.
\end{eulercomment}
\begin{eulerprompt}
>sol &= solve([x+y=3,x^2+y^2=5],[x,y]); $&sol, $&x*y with sol[1]
\end{eulerprompt}
\begin{eulerformula}
\[
\left[ \left[ x=2 , y=1 \right]  , \left[ x=1 , y=2 \right] 
  \right] 
\]
\end{eulerformula}
\begin{eulerformula}
\[
2
\]
\end{eulerformula}
\begin{eulercomment}
Ekspresi simbolis dapat memiliki bendera, yang menunjukkan perlakuan
khusus dalam Maxima. Beberapa flag juga dapat digunakan sebagai
perintah, yang lainnya tidak. Bendera ditambahkan dengan "\textbar{}" (bentuk
yang lebih bagus dari "ev (..., flags)")
\end{eulercomment}
\begin{eulerprompt}
>$& diff((x^3-1)/(x+1),x) //turunan bentuk pecahan
\end{eulerprompt}
\begin{eulerformula}
\[
\frac{3\,x^2}{x+1}-\frac{x^3-1}{\left(x+1\right)^2}
\]
\end{eulerformula}
\begin{eulerprompt}
>$& diff((x^3-1)/(x+1),x) | ratsimp //menyederhanakan pecahan
\end{eulerprompt}
\begin{eulerformula}
\[
\frac{2\,x^3+3\,x^2+1}{x^2+2\,x+1}
\]
\end{eulerformula}
\begin{eulerprompt}
>$&factor(%)
\end{eulerprompt}
\begin{eulerformula}
\[
\frac{2\,x^3+3\,x^2+1}{\left(x+1\right)^2}
\]
\end{eulerformula}
\eulerheading{Fungsi}
\begin{eulercomment}
Dalam EMT, fungsi adalah program yang ditentukan dengan perintah
"fungsi". Ini bisa menjadi fungsi satu baris atau fungsi multiline.\\
Fungsi satu baris dapat berupa numerik atau simbolik. Fungsi satu
baris numerik ditentukan oleh ": =".
\end{eulercomment}
\begin{eulerprompt}
>function f(x) := x*sqrt(x^2+1)
\end{eulerprompt}
\begin{eulercomment}
Untuk gambaran umum, kami menunjukkan semua kemungkinan definisi untuk
fungsi satu baris. Sebuah fungsi dapat dievaluasi sama seperti fungsi
Euler bawaan lainnya.
\end{eulercomment}
\begin{eulerprompt}
>f(2)
\end{eulerprompt}
\begin{euleroutput}
  4.472135955
\end{euleroutput}
\begin{eulercomment}
Fungsi ini akan bekerja untuk vektor juga, mengikuti bahasa matriks
Euler, karena ekspresi yang digunakan dalam fungsi tersebut adalah
vektorisasi.
\end{eulercomment}
\begin{eulerprompt}
>f(0:0.1:1)
\end{eulerprompt}
\begin{euleroutput}
  [0,  0.100499,  0.203961,  0.313209,  0.430813,  0.559017,  0.699714,
  0.854459,  1.0245,  1.21083,  1.41421]
\end{euleroutput}
\begin{eulercomment}
Fungsi bisa diplot. Sebagai ganti ekspresi, kita hanya perlu
memberikan nama fungsi.

Berbeda dengan ekspresi simbolik atau numerik, nama fungsi harus
diberikan dalam sebuah string.
\end{eulercomment}
\begin{eulerprompt}
>solve("f",1,y=1)
\end{eulerprompt}
\begin{euleroutput}
  0.786151377757
\end{euleroutput}
\begin{eulercomment}
Secara default, jika Anda perlu menimpa fungsi built-in, Anda harus
menambahkan kata kunci "overwrite". Menimpa fungsi built-in berbahaya
dan dapat menyebabkan masalah pada fungsi lain tergantung pada
fungsinya.

Anda masih bisa memanggil fungsi built-in sebagai "\_...", jika itu
berfungsi di inti Euler.
\end{eulercomment}
\begin{eulerprompt}
>function overwrite sin (x) := _sin(x°) // redine sine in degrees
>sin(45)
\end{eulerprompt}
\begin{euleroutput}
  0.707106781187
\end{euleroutput}
\begin{eulercomment}
Lebih baik kita menghapus definisi ulang dosa ini.
\end{eulercomment}
\begin{eulerprompt}
>forget sin; sin(pi/4)
\end{eulerprompt}
\begin{euleroutput}
  0.707106781187
\end{euleroutput}
\eulersubheading{Parameter Default}
\begin{eulercomment}
Fungsi numerik dapat memiliki parameter default.
\end{eulercomment}
\begin{eulerprompt}
>function f(x,a=1) := a*x^2
\end{eulerprompt}
\begin{eulercomment}
Menghilangkan parameter ini menggunakan nilai default.
\end{eulercomment}
\begin{eulerprompt}
>f(4)
\end{eulerprompt}
\begin{euleroutput}
  16
\end{euleroutput}
\begin{eulercomment}
Menyetelnya menimpa nilai default.
\end{eulercomment}
\begin{eulerprompt}
>f(4,5)
\end{eulerprompt}
\begin{euleroutput}
  80
\end{euleroutput}
\begin{eulercomment}
Parameter yang ditetapkan juga menimpanya. Ini digunakan oleh banyak
fungsi Euler seperti plot2d, plot3d.
\end{eulercomment}
\begin{eulerprompt}
>f(4,a=1)
\end{eulerprompt}
\begin{euleroutput}
  16
\end{euleroutput}
\begin{eulercomment}
Jika variabel bukan parameter, itu harus global. Fungsi satu baris
dapat melihat variabel global.
\end{eulercomment}
\begin{eulerprompt}
>function f(x) := a*x^2
>a=6; f(2)
\end{eulerprompt}
\begin{euleroutput}
  24
\end{euleroutput}
\begin{eulercomment}
Tetapi parameter yang ditetapkan menggantikan nilai global.

Jika argumen tidak ada dalam daftar parameter yang ditentukan
sebelumnya, itu harus dideklarasikan dengan ": ="!
\end{eulercomment}
\begin{eulerprompt}
>f(2,a:=5)
\end{eulerprompt}
\begin{euleroutput}
  20
\end{euleroutput}
\begin{eulercomment}
Fungsi simbolik didefinisikan dengan "\& =". Mereka didefinisikan di
Euler dan Maxima, dan bekerja di kedua dunia. Ekspresi yang menentukan
dijalankan melalui Maxima sebelum definisi.
\end{eulercomment}
\begin{eulerprompt}
>function g(x) &= x^3-x*exp(-x); $&g(x)
\end{eulerprompt}
\begin{eulerformula}
\[
x^3-x\,e^ {- x }
\]
\end{eulerformula}
\begin{eulercomment}
Fungsi simbolik dapat digunakan dalam ekspresi simbolik.
\end{eulercomment}
\begin{eulerprompt}
>$&diff(g(x),x), $&% with x=4/3
\end{eulerprompt}
\begin{eulerformula}
\[
x\,e^ {- x }-e^ {- x }+3\,x^2
\]
\end{eulerformula}
\begin{eulerformula}
\[
\frac{e^ {- \frac{4}{3} }}{3}+\frac{16}{3}
\]
\end{eulerformula}
\begin{eulercomment}
Mereka juga dapat digunakan dalam ekspresi numerik. Tentu saja, ini
hanya akan berfungsi jika EMT dapat menafsirkan semua yang ada di
dalam fungsi tersebut.
\end{eulercomment}
\begin{eulerprompt}
>g(5+g(1))
\end{eulerprompt}
\begin{euleroutput}
  178.635099908
\end{euleroutput}
\begin{eulercomment}
Mereka dapat digunakan untuk mendefinisikan fungsi atau ekspresi
simbolik lainnya.
\end{eulercomment}
\begin{eulerprompt}
>function G(x) &= factor(integrate(g(x),x)); $&G(c) // integrate: mengintegralkan
\end{eulerprompt}
\begin{eulerformula}
\[
\frac{e^ {- c }\,\left(c^4\,e^{c}+4\,c+4\right)}{4}
\]
\end{eulerformula}
\begin{eulerprompt}
>solve(&g(x),0.5)
\end{eulerprompt}
\begin{euleroutput}
  0.703467422498
\end{euleroutput}
\begin{eulercomment}
Cara berikut juga berlaku, karena Euler menggunakan ekspresi simbolik
dalam fungsi g, jika tidak menemukan variabel simbolis g, dan jika ada
fungsi simbolik g.
\end{eulercomment}
\begin{eulerprompt}
>solve(&g,0.5)
\end{eulerprompt}
\begin{euleroutput}
  0.703467422498
\end{euleroutput}
\begin{eulerprompt}
>function P(x,n) &= (2*x-1)^n; $&P(x,n)
\end{eulerprompt}
\begin{eulerformula}
\[
\left(2\,x-1\right)^{n}
\]
\end{eulerformula}
\begin{eulerprompt}
>function Q(x,n) &= (x+2)^n; $&Q(x,n)
\end{eulerprompt}
\begin{eulerformula}
\[
\left(x+2\right)^{n}
\]
\end{eulerformula}
\begin{eulerprompt}
>$&P(x,4), $&expand(%)
\end{eulerprompt}
\begin{eulerformula}
\[
\left(2\,x-1\right)^4
\]
\end{eulerformula}
\begin{eulerformula}
\[
16\,x^4-32\,x^3+24\,x^2-8\,x+1
\]
\end{eulerformula}
\begin{eulerprompt}
>P(3,4)
\end{eulerprompt}
\begin{euleroutput}
  625
\end{euleroutput}
\begin{eulerprompt}
>$&P(x,4)+ Q(x,3), $&expand(%)
\end{eulerprompt}
\begin{eulerformula}
\[
\left(2\,x-1\right)^4+\left(x+2\right)^3
\]
\end{eulerformula}
\begin{eulerformula}
\[
16\,x^4-31\,x^3+30\,x^2+4\,x+9
\]
\end{eulerformula}
\begin{eulerprompt}
>$&P(x,4)-Q(x,3), $&expand(%), $&factor(%)
\end{eulerprompt}
\begin{eulerformula}
\[
\left(2\,x-1\right)^4-\left(x+2\right)^3
\]
\end{eulerformula}
\begin{eulerformula}
\[
16\,x^4-33\,x^3+18\,x^2-20\,x-7
\]
\end{eulerformula}
\begin{eulerformula}
\[
16\,x^4-33\,x^3+18\,x^2-20\,x-7
\]
\end{eulerformula}
\begin{eulerprompt}
>$&P(x,4)*Q(x,3), $&expand(%), $&factor(%)
\end{eulerprompt}
\begin{eulerformula}
\[
\left(x+2\right)^3\,\left(2\,x-1\right)^4
\]
\end{eulerformula}
\begin{eulerformula}
\[
16\,x^7+64\,x^6+24\,x^5-120\,x^4-15\,x^3+102\,x^2-52\,x+8
\]
\end{eulerformula}
\begin{eulerformula}
\[
\left(x+2\right)^3\,\left(2\,x-1\right)^4
\]
\end{eulerformula}
\begin{eulerprompt}
>$&P(x,4)/Q(x,1), $&expand(%), $&factor(%)
\end{eulerprompt}
\begin{eulerformula}
\[
\frac{\left(2\,x-1\right)^4}{x+2}
\]
\end{eulerformula}
\begin{eulerformula}
\[
\frac{16\,x^4}{x+2}-\frac{32\,x^3}{x+2}+\frac{24\,x^2}{x+2}-\frac{8
 \,x}{x+2}+\frac{1}{x+2}
\]
\end{eulerformula}
\begin{eulerformula}
\[
\frac{\left(2\,x-1\right)^4}{x+2}
\]
\end{eulerformula}
\begin{eulerprompt}
>function f(x) &= x^3-x; $&f(x)
\end{eulerprompt}
\begin{eulerformula}
\[
x^3-x
\]
\end{eulerformula}
\begin{eulercomment}
Dengan \& = fungsinya adalah simbolik, dan dapat digunakan dalam
ekspresi simbolik lainnya.
\end{eulercomment}
\begin{eulerprompt}
>$&integrate(f(x),x)
\end{eulerprompt}
\begin{eulerformula}
\[
\frac{x^4}{4}-\frac{x^2}{2}
\]
\end{eulerformula}
\begin{eulercomment}
Dengan: = fungsinya adalah numerik. Contoh yang baik adalah seperti
integral pasti

\end{eulercomment}
\begin{eulerformula}
\[
f (x) = \int_1^x t^t\, dt,
\]
\end{eulerformula}
\begin{eulercomment}
yang tidak dapat dievaluasi secara simbolis.

Jika kita mendefinisikan ulang fungsi dengan kata kunci "map", ini
dapat digunakan untuk vektor x. Secara internal, fungsi ini dipanggil
untuk semua nilai x satu kali, dan hasilnya disimpan dalam vektor.
\end{eulercomment}
\begin{eulerprompt}
>function map f(x) := integrate("x^x",1,x)
>f(0:0.5:2)
\end{eulerprompt}
\begin{euleroutput}
  [-0.783431,  -0.410816,  0,  0.676863,  2.05045]
\end{euleroutput}
\begin{eulercomment}
Fungsi dapat memiliki nilai default untuk parameter.
\end{eulercomment}
\begin{eulerprompt}
>function mylog (x,base=10) := ln(x)/ln(base);
\end{eulerprompt}
\begin{eulercomment}
Sekarang fungsi tersebut dapat dipanggil dengan atau tanpa parameter
"base".
\end{eulercomment}
\begin{eulerprompt}
>mylog(100), mylog(2^6.7,2)
\end{eulerprompt}
\begin{euleroutput}
  2
  6.7
\end{euleroutput}
\begin{eulercomment}
Selain itu, dimungkinkan untuk menggunakan parameter yang ditetapkan.
\end{eulercomment}
\begin{eulerprompt}
>mylog(E^2,base=E)
\end{eulerprompt}
\begin{euleroutput}
  2
\end{euleroutput}
\begin{eulercomment}
Seringkali, kami ingin menggunakan fungsi untuk vektor di satu tempat,
dan untuk elemen individu di tempat lain. Ini dimungkinkan dengan
parameter vektor.
\end{eulercomment}
\begin{eulerprompt}
>function f([a,b]) &= a^2+b^2-a*b+b; $&f(a,b), $&f(x,y)
\end{eulerprompt}
\begin{eulerformula}
\[
b^2-a\,b+b+a^2
\]
\end{eulerformula}
\begin{eulerformula}
\[
y^2-x\,y+y+x^2
\]
\end{eulerformula}
\begin{eulercomment}
Fungsi simbolik seperti itu dapat digunakan untuk variabel simbolik.

Tetapi fungsinya juga dapat digunakan untuk vektor numerik.
\end{eulercomment}
\begin{eulerprompt}
>v=[3,4]; f(v)
\end{eulerprompt}
\begin{euleroutput}
  17
\end{euleroutput}
\begin{eulercomment}
Ada juga fungsi simbolik murni, yang tidak dapat digunakan secara
numerik.
\end{eulercomment}
\begin{eulerprompt}
>function lapl(expr,x,y) &&= diff(expr,x,2)+diff(expr,y,2)//turunan parsial kedua
\end{eulerprompt}
\begin{euleroutput}
  
                   diff(expr, y, 2) + diff(expr, x, 2)
  
\end{euleroutput}
\begin{eulerprompt}
>$&realpart((x+I*y)^4), $&lapl(%,x,y)
\end{eulerprompt}
\begin{eulerformula}
\[
y^4-6\,x^2\,y^2+x^4
\]
\end{eulerformula}
\begin{eulerformula}
\[
0
\]
\end{eulerformula}
\begin{eulercomment}
Tetapi tentu saja, mereka dapat digunakan dalam ekspresi simbolik atau
dalam definisi fungsi simbolik.
\end{eulercomment}
\begin{eulerprompt}
>function f(x,y) &= factor(lapl((x+y^2)^5,x,y)); $&f(x,y)
\end{eulerprompt}
\begin{eulerformula}
\[
10\,\left(y^2+x\right)^3\,\left(9\,y^2+x+2\right)
\]
\end{eulerformula}
\begin{eulercomment}
Untuk meringkas

- \&= mendefinisikan fungsi simbolik,\\
- := mendefinisikan fungsi numerik,\\
- \&\&= mendefinisikan fungsi simbolik murni.

\begin{eulercomment}
\eulerheading{Memecahkan Ekspresi}
\begin{eulercomment}
Ekspresi dapat diselesaikan secara numerik dan simbolik.

Untuk menyelesaikan ekspresi sederhana dari satu variabel, kita dapat
menggunakan fungsi Solving (). Diperlukan nilai awal untuk memulai
pencarian. Secara internal, Solving () menggunakan metode garis
potong.
\end{eulercomment}
\begin{eulerprompt}
>solve("x^2-2",1)
\end{eulerprompt}
\begin{euleroutput}
  1.41421356237
\end{euleroutput}
\begin{eulercomment}
Ini bekerja untuk ekspresi simbolik juga. Ambil fungsi berikut.
\end{eulercomment}
\begin{eulerprompt}
>$&solve(x^2=2,x)
\end{eulerprompt}
\begin{eulerformula}
\[
\left[ x=-\sqrt{2} , x=\sqrt{2} \right] 
\]
\end{eulerformula}
\begin{eulerprompt}
>$&solve(x^2-2,x)
\end{eulerprompt}
\begin{eulerformula}
\[
\left[ x=-\sqrt{2} , x=\sqrt{2} \right] 
\]
\end{eulerformula}
\begin{eulerprompt}
>$&solve(a*x^2+b*x+c=0,x)
\end{eulerprompt}
\begin{eulerformula}
\[
\left[ x=\frac{-\sqrt{b^2-4\,a\,c}-b}{2\,a} , x=\frac{\sqrt{b^2-4\,
 a\,c}-b}{2\,a} \right] 
\]
\end{eulerformula}
\begin{eulerprompt}
>$&solve([a*x+b*y=c,d*x+e*y=f],[x,y])
\end{eulerprompt}
\begin{eulerformula}
\[
\left[ \left[ x=-\frac{c\,e}{b\,\left(d-5\right)-a\,e} , y=\frac{c
 \,\left(d-5\right)}{b\,\left(d-5\right)-a\,e} \right]  \right] 
\]
\end{eulerformula}
\begin{eulerprompt}
>px &= 4*x^8+x^7-x^4-x; $&px
\end{eulerprompt}
\begin{eulerformula}
\[
4\,x^8+x^7-x^4-x
\]
\end{eulerformula}
\begin{eulercomment}
Sekarang kita mencari titik, di mana polinomialnya adalah 2. Dalam
Solving (), nilai target default y = 0 dapat diubah dengan variabel
yang ditetapkan.\\
Kami menggunakan y = 2 dan memeriksa dengan mengevaluasi polinomial
pada hasil sebelumnya.
\end{eulercomment}
\begin{eulerprompt}
>solve(px,1,y=2), px(%)
\end{eulerprompt}
\begin{euleroutput}
  0.966715594851
  2
\end{euleroutput}
\begin{eulercomment}
Memecahkan ekspresi simbolis dalam bentuk simbolik mengembalikan
daftar solusi. Kami menggunakan penyelesaian pemecah simbolik () yang
disediakan oleh Maxima.
\end{eulercomment}
\begin{eulerprompt}
>sol &= solve(x^2-x-1,x); $&sol
\end{eulerprompt}
\begin{eulerformula}
\[
\left[ x=\frac{1-\sqrt{5}}{2} , x=\frac{\sqrt{5}+1}{2} \right] 
\]
\end{eulerformula}
\begin{eulercomment}
Cara termudah untuk mendapatkan nilai numerik adalah dengan
mengevaluasi solusi secara numerik seperti ekspresi.
\end{eulercomment}
\begin{eulerprompt}
>longest sol()
\end{eulerprompt}
\begin{euleroutput}
      -0.6180339887498949       1.618033988749895 
\end{euleroutput}
\begin{eulercomment}
Untuk menggunakan solusi secara simbolis dalam ekspresi lain, cara
termudah adalah "dengan".
\end{eulercomment}
\begin{eulerprompt}
>$&x^2 with sol[1], $&expand(x^2-x-1 with sol[2])
\end{eulerprompt}
\begin{eulerformula}
\[
\frac{\left(\sqrt{5}-1\right)^2}{4}
\]
\end{eulerformula}
\begin{eulerformula}
\[
0
\]
\end{eulerformula}
\begin{eulercomment}
Sistem pemecahan persamaan secara simbolis dapat dilakukan dengan
vektor persamaan dan penyelesaian pemecah simbolik (). Jawabannya
adalah daftar persamaan.
\end{eulercomment}
\begin{eulerprompt}
>$&solve([x+y=2,x^3+2*y+x=4],[x,y])
\end{eulerprompt}
\begin{eulerformula}
\[
\left[ \left[ x=-1 , y=3 \right]  , \left[ x=1 , y=1 \right]  , 
 \left[ x=0 , y=2 \right]  \right] 
\]
\end{eulerformula}
\begin{eulercomment}
Fungsi f () dapat melihat variabel global. Namun seringkali kita ingin
menggunakan parameter lokal.

\end{eulercomment}
\begin{eulerformula}
\[
a^ x-x^a = 0,1
\]
\end{eulerformula}
\begin{eulercomment}
dengan a = 3.
\end{eulercomment}
\begin{eulerprompt}
>function f(x,a) := x^a-a^x;
\end{eulerprompt}
\begin{eulercomment}
Salah satu cara untuk meneruskan parameter tambahan ke f () adalah
dengan menggunakan daftar dengan nama fungsi dan parameternya (cara
lainnya adalah parameter titik koma).
\end{eulercomment}
\begin{eulerprompt}
>solve(\{\{"f",3\}\},2,y=0.1)
\end{eulerprompt}
\begin{euleroutput}
  2.54116291558
\end{euleroutput}
\begin{eulercomment}
Ini juga bekerja dengan ekspresi. Tapi kemudian, elemen daftar bernama
harus digunakan. (Lebih lanjut tentang daftar di tutorial tentang
sintaks EMT).
\end{eulercomment}
\begin{eulerprompt}
>solve(\{\{"x^a-a^x",a=3\}\},2,y=0.1)
\end{eulerprompt}
\begin{euleroutput}
  2.54116291558
\end{euleroutput}
\eulerheading{Menyelesaikan Pertidaksamaan}
\begin{eulercomment}
Untuk menyelesaikan pertidaksamaan, EMT tidak akan dapat melakukannya,
melainkan dengan bantuan Maxima, artinya secara eksak (simbolik).
Perintah Maxima yang digunakan adalah fourier\_elim(), yang harus
dipanggil dengan perintah "load(fourier\_elim)" terlebih dahulu.
\end{eulercomment}
\begin{eulerprompt}
>&load(fourier_elim)
\end{eulerprompt}
\begin{euleroutput}
  
          C:/Program Files/Euler x64/maxima/share/maxima/5.35.1/share/f\(\backslash\)
  ourier_elim/fourier_elim.lisp
  
\end{euleroutput}
\begin{eulerprompt}
>$&fourier_elim([x^2 - 1>0],[x]) // x^2-1 > 0
\end{eulerprompt}
\begin{eulerformula}
\[
\left[ 1<x \right] \lor \left[ x<-1 \right] 
\]
\end{eulerformula}
\begin{eulerprompt}
>$&fourier_elim([x^2 - 1<0],[x]) // x^2-1 < 0
\end{eulerprompt}
\begin{eulerformula}
\[
\left[ -1<x , x<1 \right] 
\]
\end{eulerformula}
\begin{eulerprompt}
>$&fourier_elim([x^2 - 1 # 0],[x]) // x^-1 <> 0
\end{eulerprompt}
\begin{eulerformula}
\[
\left[ -1<x , x<1 \right] \lor \left[ 1<x \right] \lor \left[ x<-1
  \right] 
\]
\end{eulerformula}
\begin{eulerprompt}
>$&fourier_elim([x # 6],[x])
\end{eulerprompt}
\begin{eulerformula}
\[
\left[ x<6 \right] \lor \left[ 6<x \right] 
\]
\end{eulerformula}
\begin{eulerprompt}
>$&fourier_elim([x < 1, x > 1],[x]) // tidak memiliki penyelesaian
\end{eulerprompt}
\begin{eulerformula}
\[
{\it emptyset}
\]
\end{eulerformula}
\begin{eulerprompt}
>$&fourier_elim([minf < x, x < inf],[x]) // solusinya R
\end{eulerprompt}
\begin{eulerformula}
\[
{\it universalset}
\]
\end{eulerformula}
\begin{eulerprompt}
>$&fourier_elim([x^3 - 1 > 0],[x])
\end{eulerprompt}
\begin{eulerformula}
\[
\left[ 1<x , x^2+x+1>0 \right] \lor \left[ x<1 , -x^2-x-1>0
  \right] 
\]
\end{eulerformula}
\begin{eulerprompt}
>$&fourier_elim([cos(x) < 1/2],[x]) // ??? gagal
\end{eulerprompt}
\begin{eulerformula}
\[
\left[ 1-2\,\cos x>0 \right] 
\]
\end{eulerformula}
\begin{eulerprompt}
>$&fourier_elim([y-x < 5, x - y < 7, 10 < y],[x,y]) // sistem pertidaksamaan
\end{eulerprompt}
\begin{eulerformula}
\[
\left[ y-5<x , x<y+7 , 10<y \right] 
\]
\end{eulerformula}
\begin{eulerprompt}
>$&fourier_elim([y-x < 5, x - y < 7, 10 < y],[y,x])
\end{eulerprompt}
\begin{eulerformula}
\[
\left[ {\it max}\left(10 , x-7\right)<y , y<x+5 , 5<x \right] 
\]
\end{eulerformula}
\begin{eulerprompt}
>$&fourier_elim((x + y < 5) and (x - y >8),[x,y])
\end{eulerprompt}
\begin{eulerformula}
\[
\left[ y+8<x , x<5-y , y<-\frac{3}{2} \right] 
\]
\end{eulerformula}
\begin{eulerprompt}
>$&fourier_elim(((x + y < 5) and x < 1) or  (x - y >8),[x,y])
\end{eulerprompt}
\begin{eulerformula}
\[
\left[ y+8<x \right] \lor \left[ x<{\it min}\left(1 , 5-y\right)
  \right] 
\]
\end{eulerformula}
\begin{eulerprompt}
>&fourier_elim([max(x,y) > 6, x # 8, abs(y-1) > 12],[x,y])
\end{eulerprompt}
\begin{euleroutput}
  
          [6 < x, x < 8, y < - 11] or [8 < x, y < - 11]
   or [x < 8, 13 < y] or [x = y, 13 < y] or [8 < x, x < y, 13 < y]
   or [y < x, 13 < y]
  
\end{euleroutput}
\begin{eulerprompt}
>$&fourier_elim([(x+6)/(x-9) <= 6],[x])
\end{eulerprompt}
\begin{eulerformula}
\[
\left[ x=12 \right] \lor \left[ 12<x \right] \lor \left[ x<9
  \right] 
\]
\end{eulerformula}
\eulerheading{Bahasa Matriks}
\begin{eulercomment}
Dokumentasi inti EMT berisi diskusi terperinci tentang bahasa matriks
Euler.

Vektor dan matriks dimasukkan dengan tanda kurung siku, elemen
dipisahkan dengan koma, baris dipisahkan dengan titik koma.
\end{eulercomment}
\begin{eulerprompt}
>A=[1,2;3,4]
\end{eulerprompt}
\begin{euleroutput}
              1             2 
              3             4 
\end{euleroutput}
\begin{eulercomment}
Produk matriks dilambangkan dengan titik.
\end{eulercomment}
\begin{eulerprompt}
>b=[3;4]
\end{eulerprompt}
\begin{euleroutput}
              3 
              4 
\end{euleroutput}
\begin{eulerprompt}
>b' // transpose b
\end{eulerprompt}
\begin{euleroutput}
  [3,  4]
\end{euleroutput}
\begin{eulerprompt}
>inv(A) //inverse A
\end{eulerprompt}
\begin{euleroutput}
             -2             1 
            1.5          -0.5 
\end{euleroutput}
\begin{eulerprompt}
>A.b //perkalian matriks
\end{eulerprompt}
\begin{euleroutput}
             11 
             25 
\end{euleroutput}
\begin{eulerprompt}
>A.inv(A)
\end{eulerprompt}
\begin{euleroutput}
              1             0 
              0             1 
\end{euleroutput}
\begin{eulercomment}
Poin utama dari bahasa matriks adalah bahwa semua fungsi dan operator
mengerjakan elemen untuk elemen.
\end{eulercomment}
\begin{eulerprompt}
>A.A
\end{eulerprompt}
\begin{euleroutput}
              7            10 
             15            22 
\end{euleroutput}
\begin{eulerprompt}
>A^2 //perpangkatan elemen2 A
\end{eulerprompt}
\begin{euleroutput}
              1             4 
              9            16 
\end{euleroutput}
\begin{eulerprompt}
>A.A.A
\end{eulerprompt}
\begin{euleroutput}
             37            54 
             81           118 
\end{euleroutput}
\begin{eulerprompt}
>power(A,3) //perpangkatan matriks
\end{eulerprompt}
\begin{euleroutput}
             37            54 
             81           118 
\end{euleroutput}
\begin{eulerprompt}
>A/A //pembagian elemen-elemen matriks yang seletak
\end{eulerprompt}
\begin{euleroutput}
              1             1 
              1             1 
\end{euleroutput}
\begin{eulerprompt}
>A/b //pembagian elemen2 A oleh elemen2 b kolom demi kolom (karena b vektor kolom)
\end{eulerprompt}
\begin{euleroutput}
       0.333333      0.666667 
           0.75             1 
\end{euleroutput}
\begin{eulerprompt}
>A\(\backslash\)b // hasilkali invers A dan b, A^(-1)b 
\end{eulerprompt}
\begin{euleroutput}
             -2 
            2.5 
\end{euleroutput}
\begin{eulerprompt}
>inv(A).b
\end{eulerprompt}
\begin{euleroutput}
             -2 
            2.5 
\end{euleroutput}
\begin{eulerprompt}
>A\(\backslash\)A   //A^(-1)A
\end{eulerprompt}
\begin{euleroutput}
              1             0 
              0             1 
\end{euleroutput}
\begin{eulerprompt}
>inv(A).A
\end{eulerprompt}
\begin{euleroutput}
              1             0 
              0             1 
\end{euleroutput}
\begin{eulerprompt}
>A*A //perkalin elemen-elemen matriks seletak
\end{eulerprompt}
\begin{euleroutput}
              1             4 
              9            16 
\end{euleroutput}
\begin{eulercomment}
Ini bukan hasil perkalian matriks, tetapi perkalian elemen dengan
elemen. Hal yang sama juga berlaku untuk vektor.
\end{eulercomment}
\begin{eulerprompt}
>b^2 // perpangkatan elemen-elemen matriks/vektor
\end{eulerprompt}
\begin{euleroutput}
              9 
             16 
\end{euleroutput}
\begin{eulercomment}
Jika salah satu operan adalah vektor atau skalar, itu diperluas dengan
cara alami.
\end{eulercomment}
\begin{eulerprompt}
>2*A
\end{eulerprompt}
\begin{euleroutput}
              2             4 
              6             8 
\end{euleroutput}
\begin{eulercomment}
Misalnya, jika operan adalah vektor kolom, elemennya diterapkan ke
semua baris A.
\end{eulercomment}
\begin{eulerprompt}
>[1,2]*A
\end{eulerprompt}
\begin{euleroutput}
              1             4 
              3             8 
\end{euleroutput}
\begin{eulercomment}
Jika itu adalah vektor baris, itu diterapkan ke semua kolom A.
\end{eulercomment}
\begin{eulerprompt}
>A*[2,3]
\end{eulerprompt}
\begin{euleroutput}
              2             6 
              6            12 
\end{euleroutput}
\begin{eulercomment}
Dapat dibayangkan perkalian ini seolah-olah vektor baris v telah
diduplikasi untuk membentuk matriks dengan ukuran yang sama dengan A.
\end{eulercomment}
\begin{eulerprompt}
>dup([1,2],2) // dup: menduplikasi/menggandakan vektor [1,2] sebanyak 2 kali (baris)
\end{eulerprompt}
\begin{euleroutput}
              1             2 
              1             2 
\end{euleroutput}
\begin{eulerprompt}
>A*dup([1,2],2) 
\end{eulerprompt}
\begin{euleroutput}
              1             4 
              3             8 
\end{euleroutput}
\begin{eulercomment}
Ini juga berlaku untuk dua vektor di mana satu adalah vektor baris dan
yang lainnya adalah vektor kolom. Kami menghitung i * j untuk i, j
dari 1 sampai 5. Triknya adalah mengalikan 1: 5 dengan transposenya.
Bahasa matriks Euler secara otomatis menghasilkan tabel nilai.
\end{eulercomment}
\begin{eulerprompt}
>(1:5)*(1:5)' // hasilkali elemen-elemen vektor baris dan vektor kolom
\end{eulerprompt}
\begin{euleroutput}
              1             2             3             4             5 
              2             4             6             8            10 
              3             6             9            12            15 
              4             8            12            16            20 
              5            10            15            20            25 
\end{euleroutput}
\begin{eulercomment}
Sekali lagi, ingatlah bahwa ini bukan hasil perkalian matriks!
\end{eulercomment}
\begin{eulerprompt}
>(1:5).(1:5)' // hasilkali vektor baris dan vektor kolom
\end{eulerprompt}
\begin{euleroutput}
  55
\end{euleroutput}
\begin{eulerprompt}
>sum((1:5)*(1:5)) // sama hasilnya
\end{eulerprompt}
\begin{euleroutput}
  55
\end{euleroutput}
\begin{eulercomment}
Bahkan operator seperti \textless{}atau == bekerja dengan cara yang sama.
\end{eulercomment}
\begin{eulerprompt}
>(1:10)<6 // menguji elemen-elemen yang kurang dari 6
\end{eulerprompt}
\begin{euleroutput}
  [1,  1,  1,  1,  1,  0,  0,  0,  0,  0]
\end{euleroutput}
\begin{eulercomment}
Misalnya, kita dapat menghitung jumlah elemen yang memenuhi kondisi
tertentu dengan fungsi sum ().
\end{eulercomment}
\begin{eulerprompt}
>sum((1:10)<6) // banyak elemen yang kurang dari 6
\end{eulerprompt}
\begin{euleroutput}
  5
\end{euleroutput}
\begin{eulercomment}
Euler memiliki operator perbandingan, seperti "==", yang memeriksa
kesetaraan.

Kami mendapatkan vektor 0 dan 1, di mana 1 berarti benar.
\end{eulercomment}
\begin{eulerprompt}
>t=(1:10)^2; t==25 //menguji elemen2 t yang sama dengan 25 (hanya ada 1)
\end{eulerprompt}
\begin{euleroutput}
  [0,  0,  0,  0,  1,  0,  0,  0,  0,  0]
\end{euleroutput}
\begin{eulercomment}
Dari vektor seperti itu, "nonzeros" memilih elemen bukan nol.

Dalam hal ini, kami mendapatkan indeks dari semua elemen yang lebih
besar dari 50.
\end{eulercomment}
\begin{eulerprompt}
>nonzeros(t>50) //indeks elemen2 t yang lebih besar daripada 50
\end{eulerprompt}
\begin{euleroutput}
  [8,  9,  10]
\end{euleroutput}
\begin{eulercomment}
Tentu saja, kita dapat menggunakan vektor indeks ini untuk mendapatkan
nilai t yang sesuai.
\end{eulercomment}
\begin{eulerprompt}
>t[nonzeros(t>50)] //elemen2 t yang lebih besar daripada 50
\end{eulerprompt}
\begin{euleroutput}
  [64,  81,  100]
\end{euleroutput}
\begin{eulercomment}
Sebagai contoh, mari kita cari semua kuadrat dari angka 1 sampai 1000,
yaitu 5 modulo 11 dan 3 modulo 13.
\end{eulercomment}
\begin{eulerprompt}
>t=1:1000; nonzeros(mod(t^2,11)==5 && mod(t^2,13)==3)
\end{eulerprompt}
\begin{euleroutput}
  [4,  48,  95,  139,  147,  191,  238,  282,  290,  334,  381,  425,
  433,  477,  524,  568,  576,  620,  667,  711,  719,  763,  810,  854,
  862,  906,  953,  997]
\end{euleroutput}
\begin{eulercomment}
EMT tidak sepenuhnya efektif untuk perhitungan integer. Ini
menggunakan titik mengambang presisi ganda secara internal. Namun,
seringkali ini sangat berguna.

Kita bisa memeriksa keutamaan. Mari kita cari tahu, berapa banyak
kuadrat ditambah 1 yang merupakan bilangan prima.
\end{eulercomment}
\begin{eulerprompt}
>t=1:1000; length(nonzeros(isprime(t^2+1)))
\end{eulerprompt}
\begin{euleroutput}
  112
\end{euleroutput}
\begin{eulercomment}
Fungsi nonzeros () hanya berfungsi untuk vektor. Untuk matriks, ada
mnonzeros ().
\end{eulercomment}
\begin{eulerprompt}
>seed(2); A=random(3,4)
\end{eulerprompt}
\begin{euleroutput}
       0.765761      0.401188      0.406347      0.267829 
        0.13673      0.390567      0.495975      0.952814 
       0.548138      0.006085      0.444255      0.539246 
\end{euleroutput}
\begin{eulercomment}
Ini mengembalikan indeks elemen, yang bukan nol.
\end{eulercomment}
\begin{eulerprompt}
>k=mnonzeros(A<0.4) //indeks elemen2 A yang kurang dari 0,4
\end{eulerprompt}
\begin{euleroutput}
              1             4 
              2             1 
              2             2 
              3             2 
\end{euleroutput}
\begin{eulercomment}
Indeks ini dapat digunakan untuk mengatur elemen ke nilai tertentu.
\end{eulercomment}
\begin{eulerprompt}
>mset(A,k,0) //mengganti elemen2 suatu matriks pada indeks tertentu
\end{eulerprompt}
\begin{euleroutput}
       0.765761      0.401188      0.406347             0 
              0             0      0.495975      0.952814 
       0.548138             0      0.444255      0.539246 
\end{euleroutput}
\begin{eulercomment}
Fungsi mset () juga dapat menyetel elemen pada indeks ke entri
beberapa matriks lainnya.
\end{eulercomment}
\begin{eulerprompt}
>mset(A,k,-random(size(A)))
\end{eulerprompt}
\begin{euleroutput}
       0.765761      0.401188      0.406347     -0.126917 
      -0.122404     -0.691673      0.495975      0.952814 
       0.548138     -0.483902      0.444255      0.539246 
\end{euleroutput}
\begin{eulercomment}
Dan dimungkinkan untuk mendapatkan elemen dalam vektor.
\end{eulercomment}
\begin{eulerprompt}
>mget(A,k)
\end{eulerprompt}
\begin{euleroutput}
  [0.267829,  0.13673,  0.390567,  0.006085]
\end{euleroutput}
\begin{eulercomment}
Fungsi berguna lainnya adalah extrema, yang mengembalikan nilai
minimal dan maksimal di setiap baris matriks dan posisinya.
\end{eulercomment}
\begin{eulerprompt}
>ex=extrema(A)
\end{eulerprompt}
\begin{euleroutput}
       0.267829             4      0.765761             1 
        0.13673             1      0.952814             4 
       0.006085             2      0.548138             1 
\end{euleroutput}
\begin{eulercomment}
Kita dapat menggunakan ini untuk mengekstrak nilai maksimal di setiap
baris.
\end{eulercomment}
\begin{eulerprompt}
>ex[,3]'
\end{eulerprompt}
\begin{euleroutput}
  [0.765761,  0.952814,  0.548138]
\end{euleroutput}
\begin{eulercomment}
Ini, tentu saja, sama dengan fungsi max ().
\end{eulercomment}
\begin{eulerprompt}
>max(A)'
\end{eulerprompt}
\begin{euleroutput}
  [0.765761,  0.952814,  0.548138]
\end{euleroutput}
\begin{eulercomment}
Tetapi dengan mget (), kita dapat mengekstrak indeks dan menggunakan
informasi ini untuk mengekstrak elemen pada posisi yang sama dari
matriks lain.
\end{eulercomment}
\begin{eulerprompt}
>j=(1:rows(A))'|ex[,4], mget(-A,j)
\end{eulerprompt}
\begin{euleroutput}
              1             1 
              2             4 
              3             1 
  [-0.765761,  -0.952814,  -0.548138]
\end{euleroutput}
\begin{eulercomment}
\begin{eulercomment}
\eulerheading{Fungsi Matriks Lainnya (Building Matrix)}
\begin{eulercomment}
Untuk membangun matriks, kita dapat menumpuk satu matriks di atas
matriks lainnya. Jika keduanya tidak memiliki jumlah kolom yang sama,
kolom yang lebih pendek diisi dengan 0.
\end{eulercomment}
\begin{eulerprompt}
>v=1:3; v_v
\end{eulerprompt}
\begin{euleroutput}
              1             2             3 
              1             2             3 
\end{euleroutput}
\begin{eulercomment}
Demikian juga, kita dapat melampirkan matriks ke sisi lain secara
berdampingan, jika keduanya memiliki jumlah baris yang sama.
\end{eulercomment}
\begin{eulerprompt}
>A=random(3,4); A|v'
\end{eulerprompt}
\begin{euleroutput}
       0.032444     0.0534171      0.595713      0.564454             1 
        0.83916      0.175552      0.396988       0.83514             2 
      0.0257573      0.658585      0.629832      0.770895             3 
\end{euleroutput}
\begin{eulercomment}
Jika mereka tidak memiliki jumlah baris yang sama, matriks yang lebih
pendek diisi dengan 0.

Ada pengecualian untuk aturan ini. Bilangan real yang melekat pada
matriks akan digunakan sebagai kolom yang diisi dengan bilangan real
tersebut.
\end{eulercomment}
\begin{eulerprompt}
>A|1
\end{eulerprompt}
\begin{euleroutput}
       0.032444     0.0534171      0.595713      0.564454             1 
        0.83916      0.175552      0.396988       0.83514             1 
      0.0257573      0.658585      0.629832      0.770895             1 
\end{euleroutput}
\begin{eulercomment}
Dimungkinkan untuk membuat matriks vektor baris dan kolom.
\end{eulercomment}
\begin{eulerprompt}
>[v;v]
\end{eulerprompt}
\begin{euleroutput}
              1             2             3 
              1             2             3 
\end{euleroutput}
\begin{eulerprompt}
>[v',v']
\end{eulerprompt}
\begin{euleroutput}
              1             1 
              2             2 
              3             3 
\end{euleroutput}
\begin{eulercomment}
Tujuan utamanya adalah untuk menafsirkan vektor ekspresi untuk vektor
kolom.
\end{eulercomment}
\begin{eulerprompt}
>"[x,x^2]"(v')
\end{eulerprompt}
\begin{euleroutput}
              1             1 
              2             4 
              3             9 
\end{euleroutput}
\begin{eulercomment}
Untuk mendapatkan ukuran A, kita dapat menggunakan fungsi-fungsi
berikut.
\end{eulercomment}
\begin{eulerprompt}
>C=zeros(2,4); rows(C), cols(C), size(C), length(C)
\end{eulerprompt}
\begin{euleroutput}
  2
  4
  [2,  4]
  4
\end{euleroutput}
\begin{eulercomment}
Untuk vektor, ada panjang ().
\end{eulercomment}
\begin{eulerprompt}
>length(2:10)
\end{eulerprompt}
\begin{euleroutput}
  9
\end{euleroutput}
\begin{eulercomment}
Ada banyak fungsi lain yang menghasilkan matriks.
\end{eulercomment}
\begin{eulerprompt}
>ones(2,2)
\end{eulerprompt}
\begin{euleroutput}
              1             1 
              1             1 
\end{euleroutput}
\begin{eulercomment}
Ini juga dapat digunakan dengan satu parameter. Untuk mendapatkan
vektor dengan angka selain 1, gunakan yang berikut ini.
\end{eulercomment}
\begin{eulerprompt}
>ones(5)*6
\end{eulerprompt}
\begin{euleroutput}
  [6,  6,  6,  6,  6]
\end{euleroutput}
\begin{eulercomment}
Juga matriks bilangan acak dapat dihasilkan dengan acak (distribusi
seragam) atau normal (distribusi Gauß).
\end{eulercomment}
\begin{eulerprompt}
>random(2,2)
\end{eulerprompt}
\begin{euleroutput}
        0.66566      0.831835 
          0.977      0.544258 
\end{euleroutput}
\begin{eulercomment}
Berikut adalah fungsi berguna lainnya, yang menyusun kembali
elemen-elemen matriks menjadi matriks lain.
\end{eulercomment}
\begin{eulerprompt}
>redim(1:9,3,3) // menyusun elemen2 1, 2, 3, ..., 9 ke bentuk matriks 3x3
\end{eulerprompt}
\begin{euleroutput}
              1             2             3 
              4             5             6 
              7             8             9 
\end{euleroutput}
\begin{eulercomment}
Dengan fungsi berikut, kita dapat menggunakan this dan fungsi dup
untuk menulis fungsi rep (), yang mengulang vektor sebanyak n kali.
\end{eulercomment}
\begin{eulerprompt}
>function rep(v,n) := redim(dup(v,n),1,n*cols(v))
\end{eulerprompt}
\begin{eulercomment}
Mari kita uji.
\end{eulercomment}
\begin{eulerprompt}
>rep(1:3,5)
\end{eulerprompt}
\begin{euleroutput}
  [1,  2,  3,  1,  2,  3,  1,  2,  3,  1,  2,  3,  1,  2,  3]
\end{euleroutput}
\begin{eulercomment}
Fungsi multdup () menduplikasi elemen vektor.
\end{eulercomment}
\begin{eulerprompt}
>multdup(1:3,5), multdup(1:3,[2,3,2])
\end{eulerprompt}
\begin{euleroutput}
  [1,  1,  1,  1,  1,  2,  2,  2,  2,  2,  3,  3,  3,  3,  3]
  [1,  1,  2,  2,  2,  3,  3]
\end{euleroutput}
\begin{eulercomment}
Fungsi flipx () dan flipy () mengembalikan urutan baris atau kolom
matriks. Yaitu, fungsi flipx () membalik secara horizontal.
\end{eulercomment}
\begin{eulerprompt}
>flipx(1:5) //membalik elemen2 vektor baris
\end{eulerprompt}
\begin{euleroutput}
  [5,  4,  3,  2,  1]
\end{euleroutput}
\begin{eulercomment}
Untuk rotasi, Euler memiliki rotleft () dan rotright ().
\end{eulercomment}
\begin{eulerprompt}
>rotleft(1:5) // memutar elemen2 vektor baris
\end{eulerprompt}
\begin{euleroutput}
  [2,  3,  4,  5,  1]
\end{euleroutput}
\begin{eulercomment}
Sebuah fungsi khusus adalah drop (v, i), yang menghilangkan elemen
dengan indeks di i dari vektor v.
\end{eulercomment}
\begin{eulerprompt}
>drop(10:20,3)
\end{eulerprompt}
\begin{euleroutput}
  [10,  11,  13,  14,  15,  16,  17,  18,  19,  20]
\end{euleroutput}
\begin{eulercomment}
Perhatikan bahwa vektor i dalam drop (v, i) mengacu pada indeks elemen
di v, bukan nilai elemen. Jika Anda ingin menghapus elemen, Anda harus
menemukan elemennya terlebih dahulu. Indeks fungsi (v, x) dapat
digunakan untuk mencari elemen x dalam vektor yang diurutkan v.
\end{eulercomment}
\begin{eulerprompt}
>v=primes(50), i=indexof(v,10:20), drop(v,i)
\end{eulerprompt}
\begin{euleroutput}
  [2,  3,  5,  7,  11,  13,  17,  19,  23,  29,  31,  37,  41,  43,  47]
  [0,  5,  0,  6,  0,  0,  0,  7,  0,  8,  0]
  [2,  3,  5,  7,  23,  29,  31,  37,  41,  43,  47]
\end{euleroutput}
\begin{eulercomment}
Seperti yang Anda lihat, tidak ada salahnya untuk menyertakan indeks
di luar rentang (seperti 0), indeks ganda, atau indeks yang tidak
disortir.
\end{eulercomment}
\begin{eulerprompt}
>drop(1:10,shuffle([0,0,5,5,7,12,12]))
\end{eulerprompt}
\begin{euleroutput}
  [1,  2,  3,  4,  6,  8,  9,  10]
\end{euleroutput}
\begin{eulercomment}
Ada beberapa fungsi khusus untuk mengatur diagonal atau untuk
menghasilkan matriks diagonal.

Kami mulai dengan matriks identitas.
\end{eulercomment}
\begin{eulerprompt}
>A=id(5) // matriks identitas 5x5
\end{eulerprompt}
\begin{euleroutput}
              1             0             0             0             0 
              0             1             0             0             0 
              0             0             1             0             0 
              0             0             0             1             0 
              0             0             0             0             1 
\end{euleroutput}
\begin{eulercomment}
Kemudian kami mengatur diagonal bawah (-1) menjadi 1: 4.
\end{eulercomment}
\begin{eulerprompt}
>setdiag(A,-1,1:4) //mengganti diagonal di bawah diagonal utama
\end{eulerprompt}
\begin{euleroutput}
              1             0             0             0             0 
              1             1             0             0             0 
              0             2             1             0             0 
              0             0             3             1             0 
              0             0             0             4             1 
\end{euleroutput}
\begin{eulercomment}
Perhatikan bahwa kami tidak mengubah matriks A. Kami mendapatkan
matriks baru sebagai hasil dari setdiag ().

Berikut adalah fungsi yang mengembalikan matriks tri-diagonal.
\end{eulercomment}
\begin{eulerprompt}
>function tridiag (n,a,b,c) := setdiag(setdiag(b*id(n),1,c),-1,a); ...
>tridiag(5,1,2,3)
\end{eulerprompt}
\begin{euleroutput}
              2             3             0             0             0 
              1             2             3             0             0 
              0             1             2             3             0 
              0             0             1             2             3 
              0             0             0             1             2 
\end{euleroutput}
\begin{eulercomment}
Diagonal matriks juga dapat diekstraksi dari matriks. Untuk
mendemonstrasikan ini, kami merestrukturisasi vektor 1: 9 menjadi
matriks 3x3.
\end{eulercomment}
\begin{eulerprompt}
>A=redim(1:9,3,3)
\end{eulerprompt}
\begin{euleroutput}
              1             2             3 
              4             5             6 
              7             8             9 
\end{euleroutput}
\begin{eulercomment}
Sekarang kita bisa mengekstrak diagonal.
\end{eulercomment}
\begin{eulerprompt}
>d=getdiag(A,0)
\end{eulerprompt}
\begin{euleroutput}
  [1,  5,  9]
\end{euleroutput}
\begin{eulercomment}
Misalnya. Kita dapat membagi matriks dengan diagonalnya. Bahasa
matriks memperhatikan bahwa vektor kolom d diterapkan ke matriks baris
demi baris.
\end{eulercomment}
\begin{eulerprompt}
>fraction A/d'
\end{eulerprompt}
\begin{euleroutput}
          1         2         3 
        4/5         1       6/5 
        7/9       8/9         1 
\end{euleroutput}
\eulerheading{Vektorisasi}
\begin{eulercomment}
Hampir semua fungsi di Euler bekerja untuk matriks dan input vektor
juga, jika memungkinkan.

Misalnya, fungsi sqrt () menghitung akar kuadrat dari semua elemen
vektor atau matriks.
\end{eulercomment}
\begin{eulerprompt}
>sqrt(1:3)
\end{eulerprompt}
\begin{euleroutput}
  [1,  1.41421,  1.73205]
\end{euleroutput}
\begin{eulercomment}
Jadi Anda dapat dengan mudah membuat tabel nilai. Ini adalah salah
satu cara untuk memplot fungsi (alternatifnya menggunakan ekspresi).
\end{eulercomment}
\begin{eulerprompt}
>x=1:0.01:5; y=log(x)/x^2; // terlalu panjang untuk ditampikan
\end{eulerprompt}
\begin{eulercomment}
Dengan ini dan operator titik dua a: delta: b, vektor nilai fungsi
dapat dibuat dengan mudah.

Dalam contoh berikut, kami menghasilkan vektor nilai t [i] dengan
jarak 0,1 dari -1 hingga 1. Kemudian kami menghasilkan vektor nilai
fungsi

lateks: s = t \textasciicircum{} 3-t
\end{eulercomment}
\begin{eulerprompt}
>t=-1:0.1:1; s=t^3-t
\end{eulerprompt}
\begin{euleroutput}
  [0,  0.171,  0.288,  0.357,  0.384,  0.375,  0.336,  0.273,  0.192,
  0.099,  0,  -0.099,  -0.192,  -0.273,  -0.336,  -0.375,  -0.384,
  -0.357,  -0.288,  -0.171,  0]
\end{euleroutput}
\begin{eulercomment}
EMT memperluas operator untuk skalar, vektor, dan matriks dengan cara
yang jelas.

Misalnya, vektor kolom dikali vektor baris mengembang menjadi matriks,
jika operator diterapkan. Berikut ini, v 'adalah vektor yang dialihkan
(vektor kolom).
\end{eulercomment}
\begin{eulerprompt}
>shortest (1:5)*(1:5)'
\end{eulerprompt}
\begin{euleroutput}
       1      2      3      4      5 
       2      4      6      8     10 
       3      6      9     12     15 
       4      8     12     16     20 
       5     10     15     20     25 
\end{euleroutput}
\begin{eulercomment}
Perhatikan, ini sangat berbeda dari hasil perkalian matriks. Produk
matriks dilambangkan dengan titik "." di EMT.
\end{eulercomment}
\begin{eulerprompt}
>(1:5).(1:5)'
\end{eulerprompt}
\begin{euleroutput}
  55
\end{euleroutput}
\begin{eulercomment}
Secara default, vektor baris dicetak dalam format kompak.
\end{eulercomment}
\begin{eulerprompt}
>[1,2,3,4]
\end{eulerprompt}
\begin{euleroutput}
  [1,  2,  3,  4]
\end{euleroutput}
\begin{eulercomment}
Untuk matriks, operator khusus. menunjukkan perkalian matriks, dan A
'menunjukkan transposing. Matriks 1x1 dapat digunakan seperti bilangan
real.
\end{eulercomment}
\begin{eulerprompt}
>v:=[1,2]; v.v', %^2
\end{eulerprompt}
\begin{euleroutput}
  5
  25
\end{euleroutput}
\begin{eulercomment}
Untuk mengubah urutan matriks, kami menggunakan apostrof.
\end{eulercomment}
\begin{eulerprompt}
>v=1:4; v'
\end{eulerprompt}
\begin{euleroutput}
              1 
              2 
              3 
              4 
\end{euleroutput}
\begin{eulercomment}
Sehingga kita dapat menghitung matriks A dikali vektor b.
\end{eulercomment}
\begin{eulerprompt}
>A=[1,2,3,4;5,6,7,8]; A.v'
\end{eulerprompt}
\begin{euleroutput}
             30 
             70 
\end{euleroutput}
\begin{eulercomment}
Perhatikan bahwa v masih merupakan vektor baris. Jadi v'.v berbeda
dari v.v '.
\end{eulercomment}
\begin{eulerprompt}
>v'.v
\end{eulerprompt}
\begin{euleroutput}
              1             2             3             4 
              2             4             6             8 
              3             6             9            12 
              4             8            12            16 
\end{euleroutput}
\begin{eulercomment}
v.v 'menghitung norma v kuadrat untuk vektor baris v. Hasilnya adalah
vektor 1x1, yang bekerja seperti bilangan real.
\end{eulercomment}
\begin{eulerprompt}
>v.v'
\end{eulerprompt}
\begin{euleroutput}
  30
\end{euleroutput}
\begin{eulercomment}
Ada juga norma fungsi (bersama dengan banyak fungsi lain dari Aljabar
Linear).
\end{eulercomment}
\begin{eulerprompt}
>norm(v)^2
\end{eulerprompt}
\begin{euleroutput}
  30
\end{euleroutput}
\begin{eulercomment}
Operator dan fungsi mematuhi bahasa matriks Euler.

Berikut adalah ringkasan aturannya.

- Fungsi yang diterapkan ke vektor atau matriks diterapkan ke setiap
elemen.

- Seorang operator yang beroperasi pada dua matriks dengan ukuran yang
sama diterapkan berpasangan ke elemen matriks.

- Jika kedua matriks memiliki dimensi yang berbeda, keduanya diperluas
dengan cara yang masuk akal, sehingga memiliki ukuran yang sama.

Misalnya, nilai skalar dikalikan vektor mengalikan nilai dengan setiap
elemen vektor. Atau matriks dikalikan dengan vektor (dengan *, bukan.)
Memperluas vektor ke ukuran matriks dengan menduplikasinya.

Berikut ini adalah kasus sederhana dengan operator \textasciicircum{}.
\end{eulercomment}
\begin{eulerprompt}
>[1,2,3]^2
\end{eulerprompt}
\begin{euleroutput}
  [1,  4,  9]
\end{euleroutput}
\begin{eulercomment}
Ini kasus yang lebih rumit. Vektor baris dikalikan kolom mengembang
kelelahan dengan menduplikasi.
\end{eulercomment}
\begin{eulerprompt}
>v:=[1,2,3]; v*v'
\end{eulerprompt}
\begin{euleroutput}
              1             2             3 
              2             4             6 
              3             6             9 
\end{euleroutput}
\begin{eulercomment}
Perhatikan bahwa produk skalar menggunakan produk matriks, bukan *!
\end{eulercomment}
\begin{eulerprompt}
>v.v'
\end{eulerprompt}
\begin{euleroutput}
  14
\end{euleroutput}
\begin{eulercomment}
Ada banyak fungsi untuk matriks. Kami memberikan daftar singkat. Anda
harus membaca dokumentasi untuk informasi lebih lanjut tentang
perintah ini.

\end{eulercomment}
\begin{eulerttcomment}
  sum, prod menghitung jumlah dan produk dari baris
  cumsum, cumprod melakukan hal yang sama secara kumulatif
  menghitung nilai ekstrem dari setiap baris
  extrema mengembalikan vektor dengan informasi ekstrem
  diag (A, i) mengembalikan diagonal ke-i
  setdiag (A, i, v) mengatur diagonal ke-i
  id (n) matriks identitas
  det (A) determinan
  charpoly (A) polinomial karakteristik
  eigenvalues ??(A) eigenvalues
\end{eulerttcomment}
\begin{eulerprompt}
>v*v, sum(v*v), cumsum(v*v)
\end{eulerprompt}
\begin{euleroutput}
  [1,  4,  9]
  14
  [1,  5,  14]
\end{euleroutput}
\begin{eulercomment}
Operator: menghasilkan vektor baris spasi yang sama, secara opsional
dengan ukuran langkah.
\end{eulercomment}
\begin{eulerprompt}
>1:4, 1:2:10
\end{eulerprompt}
\begin{euleroutput}
  [1,  2,  3,  4]
  [1,  3,  5,  7,  9]
\end{euleroutput}
\begin{eulercomment}
Untuk menggabungkan matriks dan vektor ada operator "\textbar{}" dan "\_".
\end{eulercomment}
\begin{eulerprompt}
>[1,2,3]|[4,5], [1,2,3]_1
\end{eulerprompt}
\begin{euleroutput}
  [1,  2,  3,  4,  5]
              1             2             3 
              1             1             1 
\end{euleroutput}
\begin{eulercomment}
Unsur-unsur matriks disebut dengan "A [i, j]".
\end{eulercomment}
\begin{eulerprompt}
>A:=[1,2,3;4,5,6;7,8,9]; A[2,3]
\end{eulerprompt}
\begin{euleroutput}
  6
\end{euleroutput}
\begin{eulercomment}
Untuk vektor baris atau kolom, v [i] adalah elemen ke-i dari vektor.
Untuk matriks, ini mengembalikan baris ke-i lengkap dari matriks
tersebut.
\end{eulercomment}
\begin{eulerprompt}
>v:=[2,4,6,8]; v[3], A[3]
\end{eulerprompt}
\begin{euleroutput}
  6
  [7,  8,  9]
\end{euleroutput}
\begin{eulercomment}
Indeks juga dapat berupa vektor baris indeks. : menunjukkan semua
indeks.
\end{eulercomment}
\begin{eulerprompt}
>v[1:2], A[:,2]
\end{eulerprompt}
\begin{euleroutput}
  [2,  4]
              2 
              5 
              8 
\end{euleroutput}
\begin{eulercomment}
Bentuk singkat dari: adalah menghilangkan indeks sepenuhnya.
\end{eulercomment}
\begin{eulerprompt}
>A[,2:3]
\end{eulerprompt}
\begin{euleroutput}
              2             3 
              5             6 
              8             9 
\end{euleroutput}
\begin{eulercomment}
Untuk tujuan vektorisasi, elemen-elemen matriks dapat diakses
seolah-olah mereka adalah vektor.
\end{eulercomment}
\begin{eulerprompt}
>A\{4\}
\end{eulerprompt}
\begin{euleroutput}
  4
\end{euleroutput}
\begin{eulercomment}
Matriks juga bisa diratakan, menggunakan fungsi redim (). Ini
diimplementasikan dalam fungsi flatten ().
\end{eulercomment}
\begin{eulerprompt}
>redim(A,1,prod(size(A))), flatten(A)
\end{eulerprompt}
\begin{euleroutput}
  [1,  2,  3,  4,  5,  6,  7,  8,  9]
  [1,  2,  3,  4,  5,  6,  7,  8,  9]
\end{euleroutput}
\begin{eulercomment}
Untuk menggunakan matriks untuk tabel, mari kita reset ke format
default, dan menghitung tabel nilai sinus dan cosinus. Perhatikan
bahwa sudut dalam radian secara default.
\end{eulercomment}
\begin{eulerprompt}
>defformat; w=0°:45°:360°; w=w'; deg(w)
\end{eulerprompt}
\begin{euleroutput}
              0 
             45 
             90 
            135 
            180 
            225 
            270 
            315 
            360 
\end{euleroutput}
\begin{eulercomment}
Sekarang kami menambahkan kolom ke matriks.
\end{eulercomment}
\begin{eulerprompt}
>M = deg(w)|w|cos(w)|sin(w)
\end{eulerprompt}
\begin{euleroutput}
              0             0             1             0 
             45      0.785398      0.707107      0.707107 
             90        1.5708             0             1 
            135       2.35619     -0.707107      0.707107 
            180       3.14159            -1             0 
            225       3.92699     -0.707107     -0.707107 
            270       4.71239             0            -1 
            315       5.49779      0.707107     -0.707107 
            360       6.28319             1             0 
\end{euleroutput}
\begin{eulercomment}
Dengan menggunakan bahasa matriks, kita dapat menghasilkan beberapa
tabel dari beberapa fungsi sekaligus.

Dalam contoh berikut, kami menghitung t [j] \textasciicircum{} i untuk i dari 1 ke n.
Kami mendapatkan matriks, di mana setiap baris adalah tabel t \textasciicircum{} i
untuk satu i. Yaitu, matriks memiliki elemen lateks: a\_ \{i, j\} = t\_j \textasciicircum{}
i, \textbackslash{} quad 1 \textbackslash{} le j \textbackslash{} le 101, \textbackslash{} quad 1 \textbackslash{} le i \textbackslash{} le n

Fungsi yang tidak bekerja untuk input vektor harus di-vectorisasi. Hal
ini dapat dicapai dengan kata kunci "peta" dalam definisi fungsi.
Kemudian fungsi tersebut akan dievaluasi untuk setiap elemen dari
parameter vektor.

Integrasi numerik integ () hanya berfungsi untuk batas interval
skalar. Jadi kita perlu melakukan vektorisasi.
\end{eulercomment}
\begin{eulerprompt}
>function map f(x) := integrate("x^x",1,x)
\end{eulerprompt}
\begin{eulercomment}
Kata kunci "map" memvektorisasi fungsi. Fungsi tersebut sekarang akan
bekerja\\
untuk vektor angka.
\end{eulercomment}
\begin{eulerprompt}
>f([1:5])
\end{eulerprompt}
\begin{euleroutput}
  [0,  2.05045,  13.7251,  113.336,  1241.03]
\end{euleroutput}
\eulerheading{Sub-Matriks dan Elemen-Matriks}
\begin{eulercomment}
Untuk mengakses elemen matriks, gunakan notasi braket.
\end{eulercomment}
\begin{eulerprompt}
>A=[1,2,3;4,5,6;7,8,9], A[2,2]
\end{eulerprompt}
\begin{euleroutput}
              1             2             3 
              4             5             6 
              7             8             9 
  5
\end{euleroutput}
\begin{eulercomment}
Kita dapat mengakses baris matriks yang lengkap.
\end{eulercomment}
\begin{eulerprompt}
>A[2]
\end{eulerprompt}
\begin{euleroutput}
  [4,  5,  6]
\end{euleroutput}
\begin{eulercomment}
Dalam kasus vektor baris atau kolom, ini mengembalikan elemen vektor.
\end{eulercomment}
\begin{eulerprompt}
>v=1:3; v[2]
\end{eulerprompt}
\begin{euleroutput}
  2
\end{euleroutput}
\begin{eulercomment}
Untuk memastikan, Anda mendapatkan baris pertama untuk matriks 1xn dan
mxn, tentukan semua kolom menggunakan indeks kedua yang kosong.
\end{eulercomment}
\begin{eulerprompt}
>A[2,]
\end{eulerprompt}
\begin{euleroutput}
  [4,  5,  6]
\end{euleroutput}
\begin{eulercomment}
Jika indeks adalah vektor indeks, Euler akan mengembalikan baris yang
sesuai dari matriks.

Di sini kami ingin baris pertama dan kedua dari A.
\end{eulercomment}
\begin{eulerprompt}
>A[[1,2]]
\end{eulerprompt}
\begin{euleroutput}
              1             2             3 
              4             5             6 
\end{euleroutput}
\begin{eulercomment}
Kami bahkan dapat menyusun ulang A menggunakan vektor indeks.
Tepatnya, kami tidak mengubah A di sini, tetapi menghitung versi A.
\end{eulercomment}
\begin{eulerprompt}
>A[[3,2,1]]
\end{eulerprompt}
\begin{euleroutput}
              7             8             9 
              4             5             6 
              1             2             3 
\end{euleroutput}
\begin{eulercomment}
Trik indeks juga bekerja dengan kolom.

Contoh ini memilih semua baris A dan kolom kedua dan ketiga.
\end{eulercomment}
\begin{eulerprompt}
>A[1:3,2:3]
\end{eulerprompt}
\begin{euleroutput}
              2             3 
              5             6 
              8             9 
\end{euleroutput}
\begin{eulercomment}
Untuk singkatan ":" menunjukkan semua indeks baris atau kolom.
\end{eulercomment}
\begin{eulerprompt}
>A[:,3]
\end{eulerprompt}
\begin{euleroutput}
              3 
              6 
              9 
\end{euleroutput}
\begin{eulercomment}
Cara lainnya, biarkan indeks pertama kosong.
\end{eulercomment}
\begin{eulerprompt}
>A[,2:3]
\end{eulerprompt}
\begin{euleroutput}
              2             3 
              5             6 
              8             9 
\end{euleroutput}
\begin{eulercomment}
Kita juga bisa mendapatkan baris terakhir A.
\end{eulercomment}
\begin{eulerprompt}
>A[-1]
\end{eulerprompt}
\begin{euleroutput}
  [7,  8,  9]
\end{euleroutput}
\begin{eulercomment}
Sekarang mari kita ubah elemen A dengan menetapkan submatrix dari A ke
beberapa nilai. Ini sebenarnya mengubah matriks A yang disimpan.
\end{eulercomment}
\begin{eulerprompt}
>A[1,1]=4
\end{eulerprompt}
\begin{euleroutput}
              4             2             3 
              4             5             6 
              7             8             9 
\end{euleroutput}
\begin{eulercomment}
Kami juga dapat menetapkan nilai ke baris A.
\end{eulercomment}
\begin{eulerprompt}
>A[1]=[-1,-1,-1]
\end{eulerprompt}
\begin{euleroutput}
             -1            -1            -1 
              4             5             6 
              7             8             9 
\end{euleroutput}
\begin{eulercomment}
Kami bahkan dapat menetapkan ke sub-matriks jika memiliki ukuran yang
sesuai.
\end{eulercomment}
\begin{eulerprompt}
>A[1:2,1:2]=[5,6;7,8]
\end{eulerprompt}
\begin{euleroutput}
              5             6            -1 
              7             8             6 
              7             8             9 
\end{euleroutput}
\begin{eulercomment}
Selain itu, beberapa pintasan diperbolehkan.
\end{eulercomment}
\begin{eulerprompt}
>A[1:2,1:2]=0
\end{eulerprompt}
\begin{euleroutput}
              0             0            -1 
              0             0             6 
              7             8             9 
\end{euleroutput}
\begin{eulercomment}
Peringatan: Indeks di luar batas menampilkan matriks kosong, atau
pesan kesalahan, bergantung pada pengaturan sistem. Standarnya adalah
pesan kesalahan. Ingat, bagaimanapun, bahwa indeks negatif dapat
digunakan untuk mengakses elemen matriks yang dihitung dari akhir.
\end{eulercomment}
\begin{eulerprompt}
>A[4]
\end{eulerprompt}
\begin{euleroutput}
  Row index 4 out of bounds!
  Error in:
  A[4] ...
      ^
\end{euleroutput}
\eulerheading{Menyortir dan Mengocok}
\begin{eulercomment}
Fungsi sort () mengurutkan vektor baris.
\end{eulercomment}
\begin{eulerprompt}
>sort([5,6,4,8,1,9])
\end{eulerprompt}
\begin{euleroutput}
  [1,  4,  5,  6,  8,  9]
\end{euleroutput}
\begin{eulercomment}
Seringkali perlu untuk mengetahui indeks dari vektor yang diurutkan
dalam vektor asli. Ini dapat digunakan untuk menyusun ulang vektor
lain dengan cara yang sama.

Mari kita mengacak vektor.
\end{eulercomment}
\begin{eulerprompt}
>v=shuffle(1:10)
\end{eulerprompt}
\begin{euleroutput}
  [4,  5,  10,  6,  8,  9,  1,  7,  2,  3]
\end{euleroutput}
\begin{eulercomment}
Indeks tersebut berisi urutan v.
\end{eulercomment}
\begin{eulerprompt}
>\{vs,ind\}=sort(v); v[ind]
\end{eulerprompt}
\begin{euleroutput}
  [1,  2,  3,  4,  5,  6,  7,  8,  9,  10]
\end{euleroutput}
\begin{eulercomment}
Ini bekerja untuk vektor string juga.
\end{eulercomment}
\begin{eulerprompt}
>s=["a","d","e","a","aa","e"]
\end{eulerprompt}
\begin{euleroutput}
  a
  d
  e
  a
  aa
  e
\end{euleroutput}
\begin{eulerprompt}
>\{ss,ind\}=sort(s); ss
\end{eulerprompt}
\begin{euleroutput}
  a
  a
  aa
  d
  e
  e
\end{euleroutput}
\begin{eulercomment}
Seperti yang Anda lihat, posisi entri ganda agak acak.
\end{eulercomment}
\begin{eulerprompt}
>ind
\end{eulerprompt}
\begin{euleroutput}
  [4,  1,  5,  2,  6,  3]
\end{euleroutput}
\begin{eulercomment}
Fungsi unik mengembalikan daftar elemen unik vektor yang diurutkan.
\end{eulercomment}
\begin{eulerprompt}
>intrandom(1,10,10), unique(%)
\end{eulerprompt}
\begin{euleroutput}
  [4,  4,  9,  2,  6,  5,  10,  6,  5,  1]
  [1,  2,  4,  5,  6,  9,  10]
\end{euleroutput}
\begin{eulercomment}
Ini bekerja untuk vektor string juga.
\end{eulercomment}
\begin{eulerprompt}
>unique(s)
\end{eulerprompt}
\begin{euleroutput}
  a
  aa
  d
  e
\end{euleroutput}
\eulerheading{Aljabar linier}
\begin{eulercomment}
EMT memiliki banyak fungsi untuk menyelesaikan masalah sistem linier,
sistem jarang, atau regresi.

Untuk sistem linier Ax = b, Anda dapat menggunakan algoritma Gauss,
matriks invers atau fit linier. Operator A \textbackslash{} b menggunakan versi
algoritma Gauss.
\end{eulercomment}
\begin{eulerprompt}
>A=[1,2;3,4]; b=[5;6]; A\(\backslash\)b
\end{eulerprompt}
\begin{euleroutput}
             -4 
            4.5 
\end{euleroutput}
\begin{eulercomment}
Untuk contoh lain, kami menghasilkan matriks 200x200 dan jumlah
barisnya. Kemudian kita menyelesaikan Ax = b menggunakan matriks
invers. Kami mengukur kesalahan sebagai deviasi maksimal semua elemen
dari 1, yang tentu saja merupakan solusi yang tepat.
\end{eulercomment}
\begin{eulerprompt}
>A=normal(200,200); b=sum(A); longest totalmax(abs(inv(A).b-1))
\end{eulerprompt}
\begin{euleroutput}
    8.790745908981989e-13 
\end{euleroutput}
\begin{eulercomment}
Jika sistem tidak memiliki solusi, kesesuaian linier meminimalkan
norma kesalahan Ax-b.
\end{eulercomment}
\begin{eulerprompt}
>A=[1,2,3;4,5,6;7,8,9]
\end{eulerprompt}
\begin{euleroutput}
              1             2             3 
              4             5             6 
              7             8             9 
\end{euleroutput}
\begin{eulercomment}
Determinan dari matriks ini adalah 0.
\end{eulercomment}
\begin{eulerprompt}
>det(A)
\end{eulerprompt}
\begin{euleroutput}
  0
\end{euleroutput}
\eulerheading{Matriks Simbolik}
\begin{eulercomment}
Maxima memiliki matriks simbolis. Tentu saja, Maxima dapat digunakan
untuk soal-soal aljabar linier sederhana. Kita dapat mendefinisikan
matriks untuk Euler dan Maxima dengan \&: =, dan kemudian
menggunakannya dalam ekspresi simbolik. Bentuk [...] biasa untuk
mendefinisikan matriks dapat digunakan di Euler untuk mendefinisikan
matriks simbolik.
\end{eulercomment}
\begin{eulerprompt}
>A &= [a,1,1;1,a,1;1,1,a]; $A
\end{eulerprompt}
\begin{eulerformula}
\[
\begin{pmatrix}a & 1 & 1 \\ 1 & a & 1 \\ 1 & 1 & a \\ \end{pmatrix}
\]
\end{eulerformula}
\begin{eulerprompt}
>$&det(A), $&factor(%)
\end{eulerprompt}
\begin{eulerformula}
\[
a\,\left(a^2-1\right)-2\,a+2
\]
\end{eulerformula}
\begin{eulerformula}
\[
\left(a-1\right)^2\,\left(a+2\right)
\]
\end{eulerformula}
\begin{eulerprompt}
>$&invert(A) with a=0
\end{eulerprompt}
\begin{eulerformula}
\[
\begin{pmatrix}-\frac{1}{2} & \frac{1}{2} & \frac{1}{2} \\ \frac{1
 }{2} & -\frac{1}{2} & \frac{1}{2} \\ \frac{1}{2} & \frac{1}{2} & -
 \frac{1}{2} \\ \end{pmatrix}
\]
\end{eulerformula}
\begin{eulerprompt}
>A &= [1,a;b,2]; $A
\end{eulerprompt}
\begin{eulerformula}
\[
\begin{pmatrix}1 & a \\ b & 2 \\ \end{pmatrix}
\]
\end{eulerformula}
\begin{eulercomment}
Seperti semua variabel simbolik, matriks ini dapat digunakan dalam
ekspresi simbolik lainnya.
\end{eulercomment}
\begin{eulerprompt}
>$&det(A-x*ident(2)), $&solve(%,x)
\end{eulerprompt}
\begin{eulerformula}
\[
\left(1-x\right)\,\left(2-x\right)-a\,b
\]
\end{eulerformula}
\begin{eulerformula}
\[
\left[ x=\frac{3-\sqrt{4\,a\,b+1}}{2} , x=\frac{\sqrt{4\,a\,b+1}+3
 }{2} \right] 
\]
\end{eulerformula}
\begin{eulercomment}
Nilai eigen juga dapat dihitung secara otomatis. Hasilnya adalah
vektor dengan dua vektor nilai eigen dan kelipatannya.
\end{eulercomment}
\begin{eulerprompt}
>$&eigenvalues([a,1;1,a])
\end{eulerprompt}
\begin{eulerformula}
\[
\left[ \left[ a-1 , a+1 \right]  , \left[ 1 , 1 \right]  \right] 
\]
\end{eulerformula}
\begin{eulercomment}
Untuk mengekstrak vektor eigen tertentu, perlu pengindeksan yang
cermat.
\end{eulercomment}
\begin{eulerprompt}
>$&eigenvectors([a,1;1,a]), &%[2][1][1]
\end{eulerprompt}
\begin{eulerformula}
\[
\left[ \left[ \left[ a-1 , a+1 \right]  , \left[ 1 , 1 \right] 
  \right]  , \left[ \left[ \left[ 1 , -1 \right]  \right]  , \left[ 
 \left[ 1 , 1 \right]  \right]  \right]  \right] 
\]
\end{eulerformula}
\begin{euleroutput}
  
                                 [1, - 1]
  
\end{euleroutput}
\begin{eulercomment}
Matriks simbolik dapat dievaluasi dalam Euler secara numerik seperti
ekspresi simbolik lainnya.
\end{eulercomment}
\begin{eulerprompt}
>A(a=4,b=5)
\end{eulerprompt}
\begin{euleroutput}
              1             4 
              5             2 
\end{euleroutput}
\begin{eulercomment}
Dalam ekspresi simbolik, gunakan dengan.
\end{eulercomment}
\begin{eulerprompt}
>$&A with [a=4,b=5]
\end{eulerprompt}
\begin{eulerformula}
\[
\begin{pmatrix}1 & 4 \\ 5 & 2 \\ \end{pmatrix}
\]
\end{eulerformula}
\begin{eulercomment}
Akses ke baris matriks simbolik berfungsi seperti halnya dengan
matriks numerik.
\end{eulercomment}
\begin{eulerprompt}
>$&A[1]
\end{eulerprompt}
\begin{eulerformula}
\[
\left[ 1 , a \right] 
\]
\end{eulerformula}
\begin{eulercomment}
Ekspresi simbolis dapat berisi tugas. Dan itu mengubah matriks A.
\end{eulercomment}
\begin{eulerprompt}
>&A[1,1]:=t+1; $&A
\end{eulerprompt}
\begin{eulerformula}
\[
\begin{pmatrix}t+1 & a \\ b & 2 \\ \end{pmatrix}
\]
\end{eulerformula}
\begin{eulercomment}
Ada fungsi simbolik dalam Maxima untuk membuat vektor dan matriks.
Untuk ini, lihat dokumentasi Maxima atau tutorial tentang Maxima di
EMT.
\end{eulercomment}
\begin{eulerprompt}
>v &= makelist(1/(i+j),i,1,3); $v
\end{eulerprompt}
\begin{eulerformula}
\[
\left[ \frac{1}{j+1} , \frac{1}{j+2} , \frac{1}{j+3} \right] 
\]
\end{eulerformula}
\begin{eulerttcomment}
 
\end{eulerttcomment}
\begin{eulerprompt}
>B &:= [1,2;3,4]; $B, $&invert(B)
\end{eulerprompt}
\begin{eulerformula}
\[
\begin{pmatrix}1 & 2 \\ 3 & 4 \\ \end{pmatrix}
\]
\end{eulerformula}
\begin{eulerformula}
\[
\begin{pmatrix}-2 & 1 \\ \frac{3}{2} & -\frac{1}{2} \\ 
 \end{pmatrix}
\]
\end{eulerformula}
\begin{eulercomment}
Hasilnya dapat dievaluasi secara numerik di Euler. Untuk informasi
lebih lanjut tentang Maxima, lihat pengantar Maxima.
\end{eulercomment}
\begin{eulerprompt}
>$&invert(B)()
\end{eulerprompt}
\begin{euleroutput}
             -2             1 
            1.5          -0.5 
\end{euleroutput}
\begin{eulercomment}
Euler juga memiliki fungsi xinv () yang kuat, yang membuat upaya lebih
besar dan mendapatkan hasil yang lebih tepat.

Perhatikan, bahwa dengan \&: = matriks B telah didefinisikan sebagai
simbolik dalam ekspresi simbolik dan numerik dalam ekspresi numerik.
Jadi kita bisa menggunakannya di sini.
\end{eulercomment}
\begin{eulerprompt}
>longest B.xinv(B)
\end{eulerprompt}
\begin{euleroutput}
                        1                       0 
                        0                       1 
\end{euleroutput}
\begin{eulercomment}
Misalnya. nilai eigen dari A dapat dihitung secara numerik.
\end{eulercomment}
\begin{eulerprompt}
>A=[1,2,3;4,5,6;7,8,9]; real(eigenvalues(A))
\end{eulerprompt}
\begin{euleroutput}
  [16.1168,  -1.11684,  0]
\end{euleroutput}
\begin{eulercomment}
Atau secara simbolis. Lihat tutorial tentang Maxima untuk detailnya.
\end{eulercomment}
\begin{eulerprompt}
>$&eigenvalues(@A)
\end{eulerprompt}
\begin{eulerformula}
\[
\left[ \left[ \frac{15-3\,\sqrt{33}}{2} , \frac{3\,\sqrt{33}+15}{2}
  , 0 \right]  , \left[ 1 , 1 , 1 \right]  \right] 
\]
\end{eulerformula}
\eulerheading{Nilai Numerik dalam Ekspresi simbolis}
\begin{eulercomment}
Ekspresi simbolik hanyalah string yang mengandung ekspresi. Jika kita
ingin mendefinisikan nilai untuk ekspresi simbolik dan ekspresi
numerik, kita harus menggunakan "\&: =".
\end{eulercomment}
\begin{eulerprompt}
>A &:= [1,pi;4,5]
\end{eulerprompt}
\begin{euleroutput}
              1       3.14159 
              4             5 
\end{euleroutput}
\begin{eulercomment}
Masih terdapat perbedaan antara bentuk numerik dan simbolik. Saat
mentransfer matriks ke bentuk simbolis, pendekatan pecahan untuk real
akan digunakan.
\end{eulercomment}
\begin{eulerprompt}
>$&A
\end{eulerprompt}
\begin{eulerformula}
\[
\begin{pmatrix}1 & \frac{1146408}{364913} \\ 4 & 5 \\ \end{pmatrix}
\]
\end{eulerformula}
\begin{eulercomment}
Untuk menghindari hal ini, ada fungsi "mxmset (variabel)".
\end{eulercomment}
\begin{eulerprompt}
>mxmset(A); $&A
\end{eulerprompt}
\begin{eulerformula}
\[
\begin{pmatrix}1 & 3.141592653589793 \\ 4 & 5 \\ \end{pmatrix}
\]
\end{eulerformula}
\begin{eulercomment}
Maxima juga dapat menghitung dengan angka floating point, dan bahkan
dengan angka mengambang besar dengan 32 digit. Namun, evaluasinya jauh
lebih lambat.
\end{eulercomment}
\begin{eulerprompt}
>$&bfloat(sqrt(2)), $&float(sqrt(2))
\end{eulerprompt}
\begin{eulerformula}
\[
1.4142135623730950488016887242097_B \times 10^{0}
\]
\end{eulerformula}
\begin{eulerformula}
\[
1.414213562373095
\]
\end{eulerformula}
\begin{eulercomment}
Ketepatan angka floating point besar dapat diubah.
\end{eulercomment}
\begin{eulerprompt}
>&fpprec:=100; &bfloat(pi)
\end{eulerprompt}
\begin{euleroutput}
  
          3.14159265358979323846264338327950288419716939937510582097494\(\backslash\)
  4592307816406286208998628034825342117068b0
  
\end{euleroutput}
\begin{eulercomment}
Variabel numerik dapat digunakan dalam ekspresi simbolik apa pun yang
menggunakan "@var".

Perhatikan bahwa ini hanya diperlukan, jika variabel telah ditentukan
dengan ": =" atau "=" sebagai variabel numerik.
\end{eulercomment}
\begin{eulerprompt}
>B:=[1,pi;3,4]; $&det(@B)
\end{eulerprompt}
\begin{eulerformula}
\[
-5.424777960769379
\]
\end{eulerformula}
\eulerheading{Demo - Suku Bunga}
\begin{eulercomment}
Di bawah ini, kami menggunakan Euler Math Toolbox (EMT) untuk
menghitung suku bunga. Kami melakukannya secara numerik dan simbolis
untuk menunjukkan kepada Anda bagaimana Euler dapat digunakan untuk
memecahkan masalah kehidupan nyata.

Asumsikan Anda memiliki modal awal 5000 (katakanlah dalam dolar).
\end{eulercomment}
\begin{eulerprompt}
>K=5000
\end{eulerprompt}
\begin{euleroutput}
  5000
\end{euleroutput}
\begin{eulercomment}
Sekarang kami mengasumsikan tingkat bunga 3\% per tahun. Mari kita
tambahkan satu tingkat sederhana dan hitung hasilnya.
\end{eulercomment}
\begin{eulerprompt}
>K*1.03
\end{eulerprompt}
\begin{euleroutput}
  5150
\end{euleroutput}
\begin{eulercomment}
Euler akan memahami sintaks berikut juga.
\end{eulercomment}
\begin{eulerprompt}
>K+K*3%
\end{eulerprompt}
\begin{euleroutput}
  5150
\end{euleroutput}
\begin{eulercomment}
Tapi lebih mudah menggunakan faktornya
\end{eulercomment}
\begin{eulerprompt}
>q=1+3%, K*q
\end{eulerprompt}
\begin{euleroutput}
  1.03
  5150
\end{euleroutput}
\begin{eulercomment}
Selama 10 tahun, kita cukup mengalikan faktor dan mendapatkan nilai
akhir dengan suku bunga majemuk.
\end{eulercomment}
\begin{eulerprompt}
>K*q^10
\end{eulerprompt}
\begin{euleroutput}
  6719.58189672
\end{euleroutput}
\begin{eulercomment}
Untuk tujuan kami, kami dapat mengatur format menjadi 2 digit setelah
titik desimal.
\end{eulercomment}
\begin{eulerprompt}
>format(12,2); K*q^10
\end{eulerprompt}
\begin{euleroutput}
      6719.58 
\end{euleroutput}
\begin{eulercomment}
Mari kita cetak yang dibulatkan menjadi 2 digit itu dalam kalimat
lengkap.
\end{eulercomment}
\begin{eulerprompt}
>"Mulai dari " + K + "$ Anda mendapatkan " + round(K*q^10,2) + "$."
\end{eulerprompt}
\begin{euleroutput}
  Mulai dari 5000$ Anda mendapatkan 6719.58$.
\end{euleroutput}
\begin{eulercomment}
Bagaimana jika kita ingin mengetahui hasil antara tahun 1 sampai tahun
9? Untuk ini, bahasa matriks Euler sangat membantu. Anda tidak perlu
menulis loop, tetapi cukup masukkan
\end{eulercomment}
\begin{eulerprompt}
>K*q^(0:10)
\end{eulerprompt}
\begin{euleroutput}
  Real 1 x 11 matrix
  
      5000.00     5150.00     5304.50     5463.64     ...
\end{euleroutput}
\begin{eulercomment}
Bagaimana keajaiban ini bekerja? Pertama, ekspresi 0:10 mengembalikan
vektor bilangan bulat.
\end{eulercomment}
\begin{eulerprompt}
>short 0:10
\end{eulerprompt}
\begin{euleroutput}
  [0,  1,  2,  3,  4,  5,  6,  7,  8,  9,  10]
\end{euleroutput}
\begin{eulercomment}
Kemudian semua operator dan fungsi di Euler dapat diterapkan ke elemen
vektor untuk elemen. jadi
\end{eulercomment}
\begin{eulerprompt}
>short q^(0:10)
\end{eulerprompt}
\begin{euleroutput}
  [1,  1.03,  1.0609,  1.0927,  1.1255,  1.1593,  1.1941,  1.2299,
  1.2668,  1.3048,  1.3439]
\end{euleroutput}
\begin{eulercomment}
adalah vektor faktor q \textasciicircum{} 0 hingga q \textasciicircum{} 10. Ini dikalikan dengan K, dan
kita mendapatkan nilai vektor.
\end{eulercomment}
\begin{eulerprompt}
>VK=K*q^(0:10);
\end{eulerprompt}
\begin{eulercomment}
Tentu saja, cara realistis untuk menghitung suku bunga ini adalah
dengan membulatkan ke sen terdekat setiap tahun. Mari kita tambahkan
fungsi untuk ini.
\end{eulercomment}
\begin{eulerprompt}
>function oneyear (K) := round(K*q,2)
\end{eulerprompt}
\begin{eulercomment}
Mari kita bandingkan kedua hasil tersebut, dengan dan tanpa
pembulatan.
\end{eulercomment}
\begin{eulerprompt}
>longest oneyear(1234.57), longest 1234.57*q
\end{eulerprompt}
\begin{euleroutput}
                  1271.61 
                1271.6071 
\end{euleroutput}
\begin{eulercomment}
Sekarang tidak ada rumus sederhana untuk tahun ke-n, dan kita harus
mengulang selama bertahun-tahun. Euler memberikan banyak solusi untuk
ini.

Cara termudah adalah fungsi iterasi, yang mengulang fungsi tertentu
beberapa kali.
\end{eulercomment}
\begin{eulerprompt}
>VKr=iterate("oneyear",5000,10)
\end{eulerprompt}
\begin{euleroutput}
  Real 1 x 11 matrix
  
      5000.00     5150.00     5304.50     5463.64     ...
\end{euleroutput}
\begin{eulercomment}
Kami dapat mencetaknya dengan cara yang ramah, menggunakan format kami
dengan tempat desimal tetap.
\end{eulercomment}
\begin{eulerprompt}
>VKr'
\end{eulerprompt}
\begin{euleroutput}
      5000.00 
      5150.00 
      5304.50 
      5463.64 
      5627.55 
      5796.38 
      5970.27 
      6149.38 
      6333.86 
      6523.88 
      6719.60 
\end{euleroutput}
\begin{eulercomment}
Untuk mendapatkan elemen tertentu dari vektor, kami menggunakan indeks
dalam tanda kurung siku.
\end{eulercomment}
\begin{eulerprompt}
>VKr[2], VKr[1:3]
\end{eulerprompt}
\begin{euleroutput}
      5150.00 
      5000.00     5150.00     5304.50 
\end{euleroutput}
\begin{eulercomment}
Anehnya, kita juga bisa menggunakan indeks vektor. Ingat bahwa 1: 3
menghasilkan vektor [1,2,3].

Mari kita bandingkan elemen terakhir dari nilai yang dibulatkan dengan
nilai penuh.
\end{eulercomment}
\begin{eulerprompt}
>VKr[-1], VK[-1]
\end{eulerprompt}
\begin{euleroutput}
      6719.60 
      6719.58 
\end{euleroutput}
\begin{eulercomment}
Perbedaannya sangat kecil.

\begin{eulercomment}
\eulerheading{Memecahkan Persamaan}
\begin{eulercomment}
Sekarang kita ambil fungsi yang lebih maju, yang menambahkan jumlah
uang tertentu setiap tahun.
\end{eulercomment}
\begin{eulerprompt}
>function onepay (K) := K*q+R
\end{eulerprompt}
\begin{eulercomment}
Kami tidak harus menentukan q atau R untuk definisi fungsi. Hanya jika
kita menjalankan perintah, kita harus menentukan nilai-nilai ini. Kami
memilih R = 200.
\end{eulercomment}
\begin{eulerprompt}
>R=200; iterate("onepay",5000,10)
\end{eulerprompt}
\begin{euleroutput}
  Real 1 x 11 matrix
  
      5000.00     5350.00     5710.50     6081.82     ...
\end{euleroutput}
\begin{eulercomment}
Bagaimana jika kita menghapus jumlah yang sama setiap tahun?
\end{eulercomment}
\begin{eulerprompt}
>R=-200; iterate("onepay",5000,10)
\end{eulerprompt}
\begin{euleroutput}
  Real 1 x 11 matrix
  
      5000.00     4950.00     4898.50     4845.45     ...
\end{euleroutput}
\begin{eulercomment}
Kami melihat bahwa uang berkurang. Jelas, jika kita hanya mendapatkan
150 bunga di tahun pertama, tetapi menghapus 200, kita kehilangan uang
setiap tahun.

Bagaimana kita bisa menentukan berapa tahun uang itu akan bertahan?
Kami harus menulis loop untuk ini. Cara termudah adalah mengulanginya
cukup lama.
\end{eulercomment}
\begin{eulerprompt}
>VKR=iterate("onepay",5000,50)
\end{eulerprompt}
\begin{euleroutput}
  Real 1 x 51 matrix
  
      5000.00     4950.00     4898.50     4845.45     ...
\end{euleroutput}
\begin{eulercomment}
Dengan menggunakan bahasa matriks, kita dapat menentukan nilai negatif
pertama dengan cara berikut.
\end{eulercomment}
\begin{eulerprompt}
>min(nonzeros(VKR<0))
\end{eulerprompt}
\begin{euleroutput}
        48.00 
\end{euleroutput}
\begin{eulercomment}
Alasan untuk ini adalah bahwa nonzeros (VKR \textless{}0) mengembalikan vektor
indeks i, di mana VKR [i] \textless{}0, dan min menghitung indeks minimal.

Karena vektor selalu dimulai dengan indeks 1, jawabannya adalah 47
tahun.

Fungsi iterate () memiliki satu trik lagi. Ini bisa mengambil kondisi
akhir sebagai argumen. Maka itu akan mengembalikan nilai dan jumlah
iterasi.
\end{eulercomment}
\begin{eulerprompt}
>\{x,n\}=iterate("onepay",5000,till="x<0"); x, n,
\end{eulerprompt}
\begin{euleroutput}
       -19.83 
        47.00 
\end{euleroutput}
\begin{eulercomment}
Mari kita coba menjawab pertanyaan yang lebih ambigu. Asumsikan kita
tahu bahwa nilainya 0 setelah 50 tahun. Berapa tingkat bunganya?

Ini adalah pertanyaan yang hanya bisa dijawab secara numerik. Di bawah
ini, kami akan mendapatkan rumus yang diperlukan. Maka Anda akan
melihat bahwa tidak ada rumus mudah untuk suku bunga. Tetapi untuk
saat ini, kami bertujuan untuk solusi numerik.

Langkah pertama adalah menentukan fungsi yang melakukan iterasi
sebanyak n kali. Kami menambahkan semua parameter ke fungsi ini.
\end{eulercomment}
\begin{eulerprompt}
>function f(K,R,P,n) := iterate("x*(1+P/100)+R",K,n;P,R)[-1]
\end{eulerprompt}
\begin{eulercomment}
Iterasinya sama seperti di atas

lateks: x\_ \{n + 1\} = x\_n \textbackslash{} cdot \textbackslash{} kiri (1+ \textbackslash{} frac \{P\} \{100\} \textbackslash{} kanan) +
R

Tapi kami lebih lama menggunakan nilai global R dalam ekspresi kami.
Fungsi seperti iterate () memiliki trik khusus di Euler. Anda dapat
meneruskan nilai variabel dalam ekspresi sebagai parameter titik koma.
Dalam hal ini P dan R.

Apalagi kami hanya tertarik pada nilai terakhir. Jadi kami mengambil
indeks [-1].

Mari kita coba tes.
\end{eulercomment}
\begin{eulerprompt}
>f(5000,-200,3,47)
\end{eulerprompt}
\begin{euleroutput}
       -19.83 
\end{euleroutput}
\begin{eulercomment}
Sekarang kita bisa menyelesaikan masalah kita.
\end{eulercomment}
\begin{eulerprompt}
>solve("f(5000,-200,x,50)",3)
\end{eulerprompt}
\begin{euleroutput}
         3.15 
\end{euleroutput}
\begin{eulercomment}
Rutin menyelesaikan memecahkan ekspresi = 0 untuk variabel x.
Jawabannya adalah 3,15\% per tahun. Kami mengambil nilai awal 3\% untuk
algoritme. Fungsi Solving () selalu membutuhkan nilai awal.

Kita dapat menggunakan fungsi yang sama untuk menjawab pertanyaan
berikut: Berapa banyak yang dapat kita keluarkan per tahun sehingga
modal awal habis setelah 20 tahun dengan asumsi tingkat bunga 3\% per
tahun.
\end{eulercomment}
\begin{eulerprompt}
>solve("f(5000,x,3,20)",-200)
\end{eulerprompt}
\begin{euleroutput}
      -336.08 
\end{euleroutput}
\begin{eulercomment}
Perhatikan bahwa Anda tidak dapat menyelesaikan jumlah tahun, karena
fungsi kami mengasumsikan n sebagai nilai integer.

\end{eulercomment}
\eulersubheading{Solusi Simbolis untuk Masalah Suku Bunga}
\begin{eulercomment}
Kita dapat menggunakan bagian simbolik Euler untuk mempelajari
masalahnya. Pertama kita mendefinisikan fungsi onepay () secara
simbolis.
\end{eulercomment}
\begin{eulerprompt}
>function op(K) &= K*q+R; $&op(K)
\end{eulerprompt}
\begin{eulerformula}
\[
R+q\,K
\]
\end{eulerformula}
\begin{eulercomment}
Kami sekarang dapat mengulang ini.
\end{eulercomment}
\begin{eulerprompt}
>$&op(op(op(op(K)))), $&expand(%)
\end{eulerprompt}
\begin{eulerformula}
\[
q\,\left(q\,\left(q\,\left(R+q\,K\right)+R\right)+R\right)+R
\]
\end{eulerformula}
\begin{eulerformula}
\[
q^3\,R+q^2\,R+q\,R+R+q^4\,K
\]
\end{eulerformula}
\begin{eulercomment}
Kami melihat sebuah pola. Setelah n periode yang kita miliki

lateks: K\_n = q \textasciicircum{} n K + R (1 + q + \textbackslash{} ldots + q \textasciicircum{} \{n-1\}) = q \textasciicircum{} n K + \textbackslash{}
frac \{q \textasciicircum{} n-1\} \{q-1\} R

Rumusnya adalah rumus jumlah geometris, yang dikenal dengan Maxima.
\end{eulercomment}
\begin{eulerprompt}
>&sum(q^k,k,0,n-1); $& % = ev(%,simpsum)
\end{eulerprompt}
\begin{eulerformula}
\[
\sum_{k=0}^{n-1}{q^{k}}=\frac{q^{n}-1}{q-1}
\]
\end{eulerformula}
\begin{eulercomment}
Ini agak rumit. Jumlahnya dievaluasi dengan bendera "simpsum" untuk
menguranginya menjadi hasil bagi.

Mari kita buat fungsi untuk ini.
\end{eulercomment}
\begin{eulerprompt}
>function fs(K,R,P,n) &= (1+P/100)^n*K + ((1+P/100)^n-1)/(P/100)*R; $&fs(K,R,P,n)
\end{eulerprompt}
\begin{eulerformula}
\[
\frac{100\,\left(\left(\frac{P}{100}+1\right)^{n}-1\right)\,R}{P}+K
 \,\left(\frac{P}{100}+1\right)^{n}
\]
\end{eulerformula}
\begin{eulercomment}
Fungsinya sama dengan fungsi f kita sebelumnya. Tapi itu lebih
efektif.
\end{eulercomment}
\begin{eulerprompt}
>longest f(5000,-200,3,47), longest fs(5000,-200,3,47)
\end{eulerprompt}
\begin{euleroutput}
       -19.82504734650985 
       -19.82504734652684 
\end{euleroutput}
\begin{eulercomment}
Sekarang kita dapat menggunakannya untuk menanyakan waktu n. Kapan
modal kita habis? Tebakan awal kami adalah 30 tahun.
\end{eulercomment}
\begin{eulerprompt}
>solve("fs(5000,-330,3,x)",30)
\end{eulerprompt}
\begin{euleroutput}
        20.51 
\end{euleroutput}
\begin{eulercomment}
Jawaban ini mengatakan bahwa itu akan menjadi negatif setelah 21
tahun.

Kita juga dapat menggunakan sisi simbolik Euler untuk menghitung rumus
pembayaran.

Asumsikan kita mendapat pinjaman sebesar K, dan membayar n pembayaran
R (dimulai setelah tahun pertama) meninggalkan sisa utang Kn (pada
saat pembayaran terakhir). Rumusnya jelas
\end{eulercomment}
\begin{eulerprompt}
>equ &= fs(K,R,P,n)=Kn; $&equ
\end{eulerprompt}
\begin{eulerformula}
\[
\frac{100\,\left(\left(\frac{P}{100}+1\right)^{n}-1\right)\,R}{P}+K
 \,\left(\frac{P}{100}+1\right)^{n}={\it Kn}
\]
\end{eulerformula}
\begin{eulercomment}
Biasanya rumus ini diberikan dalam bentuk

getah: i = \textbackslash{} frac \{P\} \{100\}
\end{eulercomment}
\begin{eulerprompt}
>equ &= (equ with P=100*i); $&equ
\end{eulerprompt}
\begin{eulerformula}
\[
\frac{\left(\left(i+1\right)^{n}-1\right)\,R}{i}+\left(i+1\right)^{
 n}\,K={\it Kn}
\]
\end{eulerformula}
\begin{eulercomment}
Kita bisa mencari nilai R secara simbolis.
\end{eulercomment}
\begin{eulerprompt}
>$&solve(equ,R)
\end{eulerprompt}
\begin{eulerformula}
\[
\left[ R=\frac{i\,{\it Kn}-i\,\left(i+1\right)^{n}\,K}{\left(i+1
 \right)^{n}-1} \right] 
\]
\end{eulerformula}
\begin{eulercomment}
Seperti yang Anda lihat dari rumusnya, fungsi ini mengembalikan
kesalahan titik mengambang untuk i = 0. Euler tetap merencanakannya.

Tentu saja, kami memiliki batasan berikut.
\end{eulercomment}
\begin{eulerprompt}
>$&limit(R(5000,0,x,10),x,0)
\end{eulerprompt}
\begin{eulerformula}
\[
\lim_{x\rightarrow 0}{R\left(5000 , 0 , x , 10\right)}
\]
\end{eulerformula}
\begin{eulercomment}
Jelas, tanpa bunga kita harus membayar kembali 10 bunga dari 500.

Persamaan ini juga bisa diselesaikan untuk n. Ini terlihat lebih
bagus, jika kita menerapkan beberapa penyederhanaan padanya.
\end{eulercomment}
\begin{eulerprompt}
>fn &= solve(equ,n) | ratsimp; $&fn
\end{eulerprompt}
\begin{eulerformula}
\[
\left[ n=\frac{\log \left(\frac{R+i\,{\it Kn}}{R+i\,K}\right)}{
 \log \left(i+1\right)} \right] 
\]
\end{eulerformula}
\begin{eulercomment}
TUGAS APLIKOM (berdasarkan topik)\\
1.Melakukan operasi bentuk bentuk aljabar
\end{eulercomment}
\begin{eulerprompt}
>$&(8*y^5)*(9*y)
\end{eulerprompt}
\begin{eulerformula}
\[
72\,y^6
\]
\end{eulerformula}
\begin{eulerprompt}
>$&(3*a^2)*(-7*a^4)
\end{eulerprompt}
\begin{eulerformula}
\[
-21\,a^6
\]
\end{eulerformula}
\begin{eulerprompt}
>$&expand((x+3)^2)
\end{eulerprompt}
\begin{eulerformula}
\[
x^2+6\,x+9
\]
\end{eulerformula}
\begin{eulerprompt}
>$&expand((2*x^2+3*y)^2)
\end{eulerprompt}
\begin{eulerformula}
\[
9\,y^2+12\,x^2\,y+4\,x^4
\]
\end{eulerformula}
\begin{eulerprompt}
>$&expand ((5*-3)^2)
\end{eulerprompt}
\begin{eulerformula}
\[
225
\]
\end{eulerformula}
\begin{eulerprompt}
>$&factor(t^2+8*t+15)
\end{eulerprompt}
\begin{eulerformula}
\[
\left(t+3\right)\,\left(t+5\right)
\]
\end{eulerformula}
\begin{eulerprompt}
>$&factor(2* x^2 + 11*x-21)
\end{eulerprompt}
\begin{eulerformula}
\[
\left(x+7\right)\,\left(2\,x-3\right)
\]
\end{eulerformula}
\begin{eulerprompt}
>$&expand((n+6)*(n-6))
\end{eulerprompt}
\begin{eulerformula}
\[
n^2-36
\]
\end{eulerformula}
\begin{eulercomment}
2. menulis ekspresi padanan tanpa negatif eksponen
\end{eulercomment}
\begin{eulerprompt}
>$&3^(-7)
\end{eulerprompt}
\begin{eulerformula}
\[
\frac{1}{2187}
\]
\end{eulerformula}
\begin{eulerprompt}
>$&(3*m^4)^3*(2*m^(-5))^4
\end{eulerprompt}
\begin{eulerformula}
\[
\frac{432}{m^8}
\]
\end{eulerformula}
\begin{eulerprompt}
>$&m^(-1)*n^(-12)/t^(-6)
\end{eulerprompt}
\begin{eulerformula}
\[
\frac{t^6}{m\,n^{12}}
\]
\end{eulerformula}
\begin{eulercomment}
3. Melakukan perhitungan dengan menggunakan bilangan kompleks
\end{eulercomment}
\begin{eulerprompt}
>(-5+3i)+(7+8i)
\end{eulerprompt}
\begin{euleroutput}
              2.00+11.00i 
\end{euleroutput}
\begin{eulerprompt}
>(-6-5i)+(9+2i)
\end{eulerprompt}
\begin{euleroutput}
               3.00-3.00i 
\end{euleroutput}
\begin{eulerprompt}
>(10+7i)-(5+3i)
\end{eulerprompt}
\begin{euleroutput}
               5.00+4.00i 
\end{euleroutput}
\begin{eulerprompt}
>(13+9i)-(8+2i)
\end{eulerprompt}
\begin{euleroutput}
               5.00+7.00i 
\end{euleroutput}
\begin{eulerprompt}
>(-6+7i)-(-5-21)
\end{eulerprompt}
\begin{euleroutput}
              20.00+7.00i 
\end{euleroutput}
\begin{eulercomment}
4. Melakukan perhitungan dengan perhitungan buatan sendiri
\end{eulercomment}
\begin{eulerprompt}
>function f(x):=3x+1
>function g(x):=x^2-2x-6
>function h(x):=x^3
>h(f(1))
\end{eulerprompt}
\begin{euleroutput}
        64.00 
\end{euleroutput}
\begin{eulerprompt}
>g(f(5))
\end{eulerprompt}
\begin{euleroutput}
       218.00 
\end{euleroutput}
\begin{eulerprompt}
>f(f(-4))
\end{eulerprompt}
\begin{euleroutput}
       -32.00 
\end{euleroutput}
\begin{eulerprompt}
>g(g(3))
\end{eulerprompt}
\begin{euleroutput}
         9.00 
\end{euleroutput}
\begin{eulerprompt}
>f(g(-1))
\end{eulerprompt}
\begin{euleroutput}
        -8.00 
\end{euleroutput}
\begin{eulercomment}
5. Menyelesaikan persamaan dan sistem persamaan\\
6. Menyelesaikan pertidaksamaan dan sistem pertidaksamaan\\
7. Melakukan manipulasi dan perhitungan matriks dan vektor
\end{eulercomment}
\begin{eulerprompt}
>$&solve(y^2+12*y+27)
\end{eulerprompt}
\begin{eulerformula}
\[
\left[ y=-9 , y=-3 \right] 
\]
\end{eulerformula}
\begin{eulerprompt}
>$&solve(x^2+100=20*x,x)
\end{eulerprompt}
\begin{eulerformula}
\[
\left[ x=10 \right] 
\]
\end{eulerformula}
\begin{eulerprompt}
>$&solve(t^2+8*t+15)
\end{eulerprompt}
\begin{eulerformula}
\[
\left[ t=-3 , t=-5 \right] 
\]
\end{eulerformula}
\begin{eulerprompt}
>$&solve(((x^2+x-6)/(x^2+8*x+15))*((x^2-25)/(x^2-4*x+4)),x)
\end{eulerprompt}
\begin{eulerformula}
\[
\left[ x=5 \right] 
\]
\end{eulerformula}
\begin{eulercomment}
TUGAS APLIKOM MENGERJAKAN SOAL (berdasarkan soal)

R exercise 2\\
\end{eulercomment}
\eulersubheading{}
\begin{eulercomment}
No 49\\
Menyederhanakan:\\
\end{eulercomment}
\begin{eulerformula}
\[
\left(\frac{24a^{10}b^{-8}c^7}{12a^6b^{-3}c^5}\right)^{-5}
\]
\end{eulerformula}
\begin{eulerprompt}
>$&((24*a^(10)*b^(-8)*c^7)/(12*a^6*b^(-3)*c^5))^(-5)
\end{eulerprompt}
\begin{eulerformula}
\[
\frac{b^{25}}{32\,a^{20}\,c^{10}}
\]
\end{eulerformula}
\begin{eulercomment}
No 50\\
Menyederhanakan:\\
\end{eulercomment}
\begin{eulerformula}
\[
\left(\frac{125p^{12}q^{-14}r^{22}}{25p^8q^6r^{-15}}\right)^{-4}
\]
\end{eulerformula}
\begin{eulerprompt}
>$&((125*p^(12)*q^(-14)*r^(22))/25*p^8*q^6*r^(-15))^(-4)
\end{eulerprompt}
\begin{eulerformula}
\[
\frac{q^{32}}{625\,p^{80}\,r^{28}}
\]
\end{eulerformula}
\begin{eulercomment}
No 90\\
operasi bilangan matematika\\
\end{eulercomment}
\begin{eulerformula}
\[
2^6*2^{-3}/2^{10}/2^{-8}
\]
\end{eulerformula}
\begin{eulerprompt}
>2^6*2^-3/2^10/2^-8
\end{eulerprompt}
\begin{euleroutput}
         2.00 
\end{euleroutput}
\begin{eulercomment}
No 91\\
Operasi bilangan matematika\\
\end{eulercomment}
\begin{eulerformula}
\[
\left(\frac{4(8-6)^2-4*3+2*8}{3^1+9^0}\right)
\]
\end{eulerformula}
\begin{eulerprompt}
>(4*(8-6)^2 - 4*3 + 2*8)/(3^1+19^0)
\end{eulerprompt}
\begin{euleroutput}
         5.00 
\end{euleroutput}
\begin{eulercomment}
No 92\\
Calculate\\
\end{eulercomment}
\begin{eulerformula}
\[
\left(\frac{[4(8-6)^2-4](3+2*8)}{2^2(2^5+5)}\right)
\]
\end{eulerformula}
\begin{eulercomment}
\end{eulercomment}
\begin{eulerprompt}
>((4*(8-6)^2-4)*3+2*8)/(3^1+9^0)
\end{eulerprompt}
\begin{euleroutput}
        13.00 
\end{euleroutput}
\begin{eulercomment}
R exercise 3\\
\end{eulercomment}
\eulersubheading{}
\begin{eulercomment}
no 27\\
\end{eulercomment}
\begin{eulerformula}
\[
(x+3)^2
\]
\end{eulerformula}
\begin{eulerprompt}
> $&showev('expand((x+3)^2))
\end{eulerprompt}
\begin{eulerformula}
\[
{\it expand}\left(\left(x+3\right)^2\right)=x^2+6\,x+9
\]
\end{eulerformula}
\begin{eulercomment}
no 29\\
\end{eulercomment}
\begin{eulerformula}
\[
(y-5)^2
\]
\end{eulerformula}
\begin{eulerprompt}
>$&showev('expand((y-5)^2))
\end{eulerprompt}
\begin{eulerformula}
\[
{\it expand}\left(\left(y-5\right)^2\right)=y^2-10\,y+25
\]
\end{eulerformula}
\begin{eulercomment}
no 33\\
\end{eulercomment}
\begin{eulerformula}
\[
(2x+3y)^2
\]
\end{eulerformula}
\begin{eulerprompt}
>$&showev('expand((2*x+3*y)^2))
\end{eulerprompt}
\begin{eulerformula}
\[
{\it expand}\left(\left(3\,y+2\,x\right)^2\right)=9\,y^2+12\,x\,y+4
 \,x^2
\]
\end{eulerformula}
\begin{eulercomment}
no 39\\
\end{eulercomment}
\begin{eulerformula}
\[
(3y+4)(3y-4)
\]
\end{eulerformula}
\begin{eulerprompt}
>$&showev('expand((3*y+4)*(3*y-4)))
\end{eulerprompt}
\begin{eulerformula}
\[
{\it expand}\left(\left(3\,y-4\right)\,\left(3\,y+4\right)\right)=9
 \,y^2-16
\]
\end{eulerformula}
\begin{eulercomment}
no 42\\
\end{eulercomment}
\begin{eulerformula}
\[
(3x+5y)(3x-5y)
\]
\end{eulerformula}
\begin{eulerprompt}
>$&showev('expand ((3*x + 5*y)*(3*x - 5*y)))
\end{eulerprompt}
\begin{eulerformula}
\[
{\it expand}\left(\left(3\,x-5\,y\right)\,\left(5\,y+3\,x\right)
 \right)=9\,x^2-25\,y^2
\]
\end{eulerformula}
\begin{eulercomment}
R exercise 4\\
\end{eulercomment}
\eulersubheading{}
\begin{eulercomment}
no 24\\
\end{eulercomment}
\begin{eulerformula}
\[
y^2+12y+27
\]
\end{eulerformula}
\begin{eulerprompt}
>$&solve(y^2+12*y+27)
\end{eulerprompt}
\begin{eulerformula}
\[
\left[ y=-9 , y=-3 \right] 
\]
\end{eulerformula}
\begin{eulercomment}
no 23\\
\end{eulercomment}
\begin{eulerformula}
\[
t^2+8t+15
\]
\end{eulerformula}
\begin{eulerprompt}
>$&solve(t^2+8*t+15)
\end{eulerprompt}
\begin{eulerformula}
\[
\left[ t=-3 , t=-5 \right] 
\]
\end{eulerformula}
\begin{eulercomment}
Nomor 47\\
\end{eulercomment}
\begin{eulerformula}
\[
z^2-81
\]
\end{eulerformula}
\begin{eulerprompt}
>$&solve(z^2-81)
\end{eulerprompt}
\begin{eulerformula}
\[
\left[ z=-9 , z=9 \right] 
\]
\end{eulerformula}
\begin{eulercomment}
no 48\\
\end{eulercomment}
\begin{eulerformula}
\[
m^2-4
\]
\end{eulerformula}
\begin{eulerprompt}
>$&solve(m^2-4)
\end{eulerprompt}
\begin{eulerformula}
\[
\left[ m=-2 , m=2 \right] 
\]
\end{eulerformula}
\begin{eulercomment}
no 49\\
\end{eulercomment}
\begin{eulerformula}
\[
16x^2-9
\]
\end{eulerformula}
\begin{eulerprompt}
>$&solve(16*x^2-9)
\end{eulerprompt}
\begin{eulerformula}
\[
\left[ x=-\frac{3}{4} , x=\frac{3}{4} \right] 
\]
\end{eulerformula}
\begin{eulercomment}
R exercise 5\\
\end{eulercomment}
\eulersubheading{}
\begin{eulercomment}
soal no 36\\
tentukan nilai y\\
\end{eulercomment}
\begin{eulerformula}
\[
y^2-4y-45=0
\]
\end{eulerformula}
\begin{eulerprompt}
>$&solve(y^2-4*y-45,y)
\end{eulerprompt}
\begin{eulerformula}
\[
\left[ y=9 , y=-5 \right] 
\]
\end{eulerformula}
\begin{eulercomment}
soal no 38\\
tentukan nilai y\\
\end{eulercomment}
\begin{eulerformula}
\[
t^2+6t=0
\]
\end{eulerformula}
\begin{eulerprompt}
>$&solve(t^2+ 6*t,t)
\end{eulerprompt}
\begin{eulerformula}
\[
\left[ t=-6 , t=0 \right] 
\]
\end{eulerformula}
\begin{eulercomment}
soal no 41\\
tentukan nilai x\\
\end{eulercomment}
\begin{eulerformula}
\[
x^2 + 100 = 20x
\]
\end{eulerformula}
\begin{eulerprompt}
>$&solve(x^2+100=20*x,x)
\end{eulerprompt}
\begin{eulerformula}
\[
\left[ x=10 \right] 
\]
\end{eulerformula}
\begin{eulercomment}
soal no 42\\
tentukan nilai y\\
\end{eulercomment}
\begin{eulerformula}
\[
y^2+25=10y
\]
\end{eulerformula}
\begin{eulerprompt}
>$&solve(y^2+25=10*y,y)
\end{eulerprompt}
\begin{eulerformula}
\[
\left[ y=5 \right] 
\]
\end{eulerformula}
\begin{eulercomment}
soal no 45\\
tentukan nilai y\\
\end{eulercomment}
\begin{eulerformula}
\[
3y^2 + 8y + 4=0
\]
\end{eulerformula}
\begin{eulerprompt}
>$&solve(3*y^2 + 8*y + 4,y)
\end{eulerprompt}
\begin{eulerformula}
\[
\left[ y=-\frac{2}{3} , y=-2 \right] 
\]
\end{eulerformula}
\begin{eulercomment}
soal no 47\\
tentukan nilai z\\
\end{eulercomment}
\begin{eulerformula}
\[
12z^2+z=6
\]
\end{eulerformula}
\begin{eulerprompt}
>$&solve(12*z^2+z=6,z)
\end{eulerprompt}
\begin{eulerformula}
\[
\left[ z=-\frac{3}{4} , z=\frac{2}{3} \right] 
\]
\end{eulerformula}
\begin{eulercomment}
soal no 60\\
tentukan nilai x\\
\end{eulercomment}
\begin{eulerformula}
\[
5x^2-75=0
\]
\end{eulerformula}
\begin{eulerprompt}
>$&solve(5*x^2-75=0,x)
\end{eulerprompt}
\begin{eulerformula}
\[
\left[ x=-\sqrt{15} , x=\sqrt{15} \right] 
\]
\end{eulerformula}
\begin{eulercomment}
R exercise 6\\
\end{eulercomment}
\eulersubheading{}
\begin{eulercomment}
no 9\\
Menyederhanakan\\
\end{eulercomment}
\begin{eulerformula}
\[
\frac{x^{2}-4}{x^{2}-4x+4}
\]
\end{eulerformula}
\begin{eulerprompt}
>$&((x^2)/(x^2-4*x+4)), $&factor(%)
\end{eulerprompt}
\begin{eulerformula}
\[
\frac{x^2}{x^2-4\,x+4}
\]
\end{eulerformula}
\begin{eulerformula}
\[
\frac{x^2}{\left(x-2\right)^2}
\]
\end{eulerformula}
\begin{eulercomment}
no 11\\
Menyederhanakan\\
\end{eulercomment}
\begin{eulerformula}
\[
\frac{x^{3}-6x^{2}+9x}{x^{3}-3x^{2}}
\]
\end{eulerformula}
\begin{eulerprompt}
>$&((x^3 - 6*x^2 + 9*x)/ (x^3 - 3*x^2)), $&factor(%)
\end{eulerprompt}
\begin{eulerformula}
\[
\frac{x^3-6\,x^2+9\,x}{x^3-3\,x^2}
\]
\end{eulerformula}
\begin{eulerformula}
\[
\frac{x-3}{x}
\]
\end{eulerformula}
\begin{eulercomment}
no 14\\
Menyederhanakan\\
\end{eulercomment}
\begin{eulerformula}
\[
\frac{2x^{2}-20x+50}{10x^{2}-30x-100}
\]
\end{eulerformula}
\begin{eulerprompt}
>$&((2*x^2-20*x+50)/(10*x^2-30*x-100)), $&factor(%)
\end{eulerprompt}
\begin{eulerformula}
\[
\frac{2\,x^2-20\,x+50}{10\,x^2-30\,x-100}
\]
\end{eulerformula}
\begin{eulerformula}
\[
\frac{x-5}{5\,\left(x+2\right)}
\]
\end{eulerformula}
\begin{eulercomment}
no 15\\
Menyederhanakan\\
\end{eulercomment}
\begin{eulerformula}
\[
\frac{4-x}{x^{2}+4x-32}
\]
\end{eulerformula}
\begin{eulerprompt}
>$&((4-x)/(x^2 +4*x-32)), $&factor(%)
\end{eulerprompt}
\begin{eulerformula}
\[
\frac{4-x}{x^2+4\,x-32}
\]
\end{eulerformula}
\begin{eulerformula}
\[
-\frac{1}{x+8}
\]
\end{eulerformula}
\begin{eulercomment}
no 16\\
Menyederhanakan\\
\end{eulercomment}
\begin{eulerformula}
\[
\frac{6-x}{x^{2}-36}
\]
\end{eulerformula}
\begin{eulerprompt}
>$&((6-x)/(x^2-36)), $&factor(%)
\end{eulerprompt}
\begin{eulerformula}
\[
\frac{6-x}{x^2-36}
\]
\end{eulerformula}
\begin{eulerformula}
\[
-\frac{1}{x+6}
\]
\end{eulerformula}
\begin{eulercomment}
no 23
\end{eulercomment}
\begin{eulerprompt}
>$&(((m^2-n^2)/(r+s))/((m-n)/r+s)), $&factor(%)
\end{eulerprompt}
\begin{eulerformula}
\[
\frac{m^2-n^2}{\left(s+\frac{m-n}{r}\right)\,\left(s+r\right)}
\]
\end{eulerformula}
\begin{eulerformula}
\[
\frac{\left(m-n\right)\,\left(n+m\right)\,r}{\left(s+r\right)\,
 \left(r\,s-n+m\right)}
\]
\end{eulerformula}
\begin{eulercomment}
no 25
\end{eulercomment}
\begin{eulerprompt}
>$&(((3*x+12)/(2*x-8))/(((x+4)^2)/(x-4)^2)), $&factor(%)
\end{eulerprompt}
\begin{eulerformula}
\[
\frac{\left(x-4\right)^2\,\left(3\,x+12\right)}{\left(x+4\right)^2
 \,\left(2\,x-8\right)}
\]
\end{eulerformula}
\begin{eulerformula}
\[
\frac{3\,\left(x-4\right)}{2\,\left(x+4\right)}
\]
\end{eulerformula}
\begin{eulercomment}
no 28
\end{eulercomment}
\begin{eulerprompt}
>$&(((c^3+8)/(c^2-4)) /((c^2 -2*c +4)/(c^2-4*c+4))), $&factor(%) 
\end{eulerprompt}
\begin{eulerformula}
\[
\frac{\left(c^2-4\,c+4\right)\,\left(c^3+8\right)}{\left(c^2-4
 \right)\,\left(c^2-2\,c+4\right)}
\]
\end{eulerformula}
\begin{eulerformula}
\[
c-2
\]
\end{eulerformula}
\begin{eulercomment}
Review\\
\end{eulercomment}
\eulersubheading{}
\begin{eulercomment}
no 73\\
\end{eulercomment}
\begin{eulerformula}
\[
(a^n-b^n)^x
\]
\end{eulerformula}
\begin{eulerprompt}
>function P(a,b,n,x) &= (a^n - b^n)^x; $&P(a,b,n,x)
\end{eulerprompt}
\begin{eulerformula}
\[
\left(a^{n}-b^{n}\right)^{x}
\]
\end{eulerformula}
\begin{eulerprompt}
>$&P(a,b,n,3), $&expand(%)
\end{eulerprompt}
\begin{eulerformula}
\[
\left(a^{n}-b^{n}\right)^3
\]
\end{eulerformula}
\begin{eulerformula}
\[
-b^{3\,n}+3\,a^{n}\,b^{2\,n}-3\,a^{2\,n}\,b^{n}+a^{3\,n}
\]
\end{eulerformula}
\begin{eulercomment}
no 71
\end{eulercomment}
\begin{eulerprompt}
>function P(a,n) &= (t^a+t^(-a))^n; $&P(a,n)
\end{eulerprompt}
\begin{eulerformula}
\[
\left(t^{a}+\frac{1}{t^{a}}\right)^{n}
\]
\end{eulerformula}
\begin{eulerprompt}
>$&P(a,2), $&expand(%)
\end{eulerprompt}
\begin{eulerformula}
\[
\left(t^{a}+\frac{1}{t^{a}}\right)^2
\]
\end{eulerformula}
\begin{eulerformula}
\[
t^{2\,a}+\frac{1}{t^{2\,a}}+2
\]
\end{eulerformula}
\begin{eulercomment}
no 39
\end{eulercomment}
\begin{eulerprompt}
>$&solve(9*x^2-30*x+25,x)
\end{eulerprompt}
\begin{eulerformula}
\[
\left[ x=\frac{5}{3} \right] 
\]
\end{eulerformula}
\begin{eulercomment}
no 41
\end{eulercomment}
\begin{eulerprompt}
>$&solve(18*x^2-3*x+6,x)
\end{eulerprompt}
\begin{eulerformula}
\[
\left[ x=\frac{1-\sqrt{47}\,i}{12} , x=\frac{\sqrt{47}\,i+1}{12}
  \right] 
\]
\end{eulerformula}
\begin{eulercomment}
no 48
\end{eulercomment}
\begin{eulerprompt}
>$&solve((8-3*x)=(-7+2*x),x)
\end{eulerprompt}
\begin{eulerformula}
\[
\left[ x=3 \right] 
\]
\end{eulerformula}
\begin{eulercomment}
no 70
\end{eulercomment}
\begin{eulerprompt}
>$& '((x^n+10)*(x^n-4)) = expand(((x^n+10)*(x^n-4)))
\end{eulerprompt}
\begin{eulerformula}
\[
\left(x^{n}-4\right)\,\left(x^{n}+10\right)=x^{2\,n}+6\,x^{n}-40
\]
\end{eulerformula}
\begin{eulercomment}
no 71
\end{eulercomment}
\begin{eulerprompt}
>$& '((t^a+t^(-a))^2) = expand(((t^a+t^(-a))^2))
\end{eulerprompt}
\begin{eulerformula}
\[
\left(t^{a}+\frac{1}{t^{a}}\right)^2=t^{2\,a}+\frac{1}{t^{2\,a}}+2
\]
\end{eulerformula}
\begin{eulercomment}
no 72
\end{eulercomment}
\begin{eulerprompt}
>$& '((y^b-z^c)*(y^b+z^c)) = expand((y^b-z^c)*(y^b+z^c))
\end{eulerprompt}
\begin{eulerformula}
\[
\left(y^{b}-z^{c}\right)\,\left(z^{c}+y^{b}\right)=y^{2\,b}-z^{2\,c
 }
\]
\end{eulerformula}
\begin{eulercomment}
no 73
\end{eulercomment}
\begin{eulerprompt}
>$& '((a^n-b^n)^3) = expand((a^n-b^n)^3)
\end{eulerprompt}
\begin{eulerformula}
\[
\left(a^{n}-b^{n}\right)^3=-b^{3\,n}+3\,a^{n}\,b^{2\,n}-3\,a^{2\,n}
 \,b^{n}+a^{3\,n}
\]
\end{eulerformula}
\begin{eulercomment}
chapter R test\\
\end{eulercomment}
\eulersubheading{}
\begin{eulerttcomment}
 no 32
\end{eulerttcomment}
\begin{eulerprompt}
>$& '(((x^2+x-6)/(x^2+8*x+15))*((x^2-25)/(x^2-4*x+4))) = simplify(((x^2+x-6)/(x^2+8*x+15))*((x^2-25)/(x^2-4*x+4)))
\end{eulerprompt}
\begin{eulerformula}
\[
\frac{\left(x^2-25\right)\,\left(x^2+x-6\right)}{\left(x^2-4\,x+4
 \right)\,\left(x^2+8\,x+15\right)}={\it simplify}\left(\frac{\left(x
 ^2-25\right)\,\left(x^2+x-6\right)}{\left(x^2-4\,x+4\right)\,\left(x
 ^2+8\,x+15\right)}\right)
\]
\end{eulerformula}
\begin{eulerprompt}
>$&solve(((x^2+x-6)/(x^2+8*x+15))*((x^2-25)/(x^2-4*x+4)),x)
\end{eulerprompt}
\begin{eulerformula}
\[
\left[ x=5 \right] 
\]
\end{eulerformula}
\begin{eulercomment}
no 33
\end{eulercomment}
\begin{eulerprompt}
>$& '(((x)/(x^2-1))-((3)/(x^2+4*x-5)))=simplify(((x)/(x^2-1))-((3)/(x^2+4*x-5)))
\end{eulerprompt}
\begin{eulerformula}
\[
\frac{x}{x^2-1}-\frac{3}{x^2+4\,x-5}={\it simplify}\left(\frac{x}{x
 ^2-1}-\frac{3}{x^2+4\,x-5}\right)
\]
\end{eulerformula}
\begin{eulerprompt}
>$&solve(((x)/(x^2-1))-((3)/(x^2+4*x-5)))
\end{eulerprompt}
\begin{eulerformula}
\[
\left[ x=-3 \right] 
\]
\end{eulerformula}
\begin{eulercomment}
exercise 2.3\\
\end{eulercomment}
\eulersubheading{}
\begin{eulercomment}
Cari

\end{eulercomment}
\begin{eulerformula}
\[
\left(f\circ g\right)\left(x\right) dan \left(g\circ f\right)\left(x\right)
\]
\end{eulerformula}
\begin{eulercomment}
dan domain nya !

\end{eulercomment}
\begin{eulerformula}
\[
1. f(x)=x+3\ ,\ g(x)=x-3
\]
\end{eulerformula}
\begin{eulercomment}
\end{eulercomment}
\begin{eulerformula}
\[
\left(f\circ g\right)\left(x\right)=
\]
\end{eulerformula}
\begin{eulerprompt}
>$&gx:=x-3; $&fx:=gx+3; $&fx
\end{eulerprompt}
\begin{eulerformula}
\[
x
\]
\end{eulerformula}
\begin{eulercomment}
dengan domainnya

\end{eulercomment}
\begin{eulerformula}
\[
D_{f\circ g}=\left\{x\in\mathbb{R}\right\}
\]
\end{eulerformula}
\begin{eulercomment}
\end{eulercomment}
\begin{eulerformula}
\[
\left(g\circ f\right)\left(x\right)=
\]
\end{eulerformula}
\begin{eulerprompt}
>$&fx:=x+3; $&gx:=fx-3; $&gx
\end{eulerprompt}
\begin{eulerformula}
\[
x
\]
\end{eulerformula}
\begin{eulercomment}
dengan domainnya\\
\end{eulercomment}
\begin{eulerformula}
\[
D_{g\circ f}=\left\{x\in\mathbb{R}\right\}
\]
\end{eulerformula}
\begin{eulercomment}
\end{eulercomment}
\begin{eulerformula}
\[
2. f(x)=4/(1-5x)\ \ ,\ g(x)=1/x
\]
\end{eulerformula}
\begin{eulerformula}
\[
\left(f\circ g\right)\left(x\right)=
\]
\end{eulerformula}
\begin{eulerprompt}
>$&gx:=1/x; $&fx:=4/(1-5*gx); $&fx
\end{eulerprompt}
\begin{eulerformula}
\[
\frac{4}{1-\frac{5}{x}}
\]
\end{eulerformula}
\begin{eulercomment}
dengan domain\\
\end{eulercomment}
\begin{eulerformula}
\[
D_{f\circ g}=\left\{x\in\mathbb{R}|x\neq 0\cup x\neq 5\right\}
\]
\end{eulerformula}
\begin{eulercomment}
\end{eulercomment}
\begin{eulerformula}
\[
\left(g\circ f\right)\left(x\right)=
\]
\end{eulerformula}
\begin{eulerprompt}
>$&fx:=4/(1-5*x); $&gx:=1/fx; $&gx
\end{eulerprompt}
\begin{eulerformula}
\[
\frac{1-5\,x}{4}
\]
\end{eulerformula}
\begin{eulercomment}
dengan domainnya\\
\end{eulercomment}
\begin{eulerformula}
\[
D_{g\circ f}=\left\{x\in\mathbb{R}\right\}
\]
\end{eulerformula}
\begin{eulercomment}
Diberikan fungsi\\
\end{eulercomment}
\begin{eulerformula}
\[
f(x)=3x+1 , g(x)=x^2-2x-6 , h(x)=x^3
\]
\end{eulerformula}
\begin{eulercomment}
cari\\
\end{eulercomment}
\begin{eulerformula}
\[
3.\ \left(f\circ g\right)\left(1/3\right)
\]
\end{eulerformula}
\begin{eulerprompt}
>$x:=1/3; $&gx:=x^2-2*x-6; $&fx:=3*gx+1; $&fx
\end{eulerprompt}
\begin{eulerformula}
\[
-\frac{56}{3}
\]
\end{eulerformula}
\begin{eulerformula}
\[
4. \left(g\circ h\right)\left(1/2\right)
\]
\end{eulerformula}
\begin{eulerprompt}
>$x:=1/2; $&hx:=x^3; $&gx:=hx^2-2*hx-6; $&gx
\end{eulerprompt}
\begin{eulerformula}
\[
-\frac{399}{64}
\]
\end{eulerformula}
\begin{eulerformula}
\[
5. \left(g\circ g\right)\left(-2\right)
\]
\end{eulerformula}
\begin{eulerprompt}
>$x:=-2; $&gx:=x^2-2*x-6; $&gx:=gx^2-2*gx-6; $&gx
\end{eulerprompt}
\begin{eulerformula}
\[
-6
\]
\end{eulerformula}
\begin{eulercomment}
Exercise 3.1 \\
\end{eulercomment}
\eulersubheading{}
\begin{eulercomment}
no 37\\
\end{eulercomment}
\begin{eulerformula}
\[
x^2-2=15
\]
\end{eulerformula}
\begin{eulerprompt}
>$&solve(x^2-2= 15,x)
\end{eulerprompt}
\begin{euleroutput}
  Maxima said:
  solve: all variables must not be numbers.
   -- an error. To debug this try: debugmode(true);
  
  Error in:
  $&solve(x^2-2= 15,x) ...
                      ^
\end{euleroutput}
\begin{eulercomment}
no 39\\
\end{eulercomment}
\begin{eulerformula}
\[
5m^2+3m=2
\]
\end{eulerformula}
\begin{eulerprompt}
>$&solve(5*m^2+3*m=2,m)
\end{eulerprompt}
\begin{eulerformula}
\[
\left[ m=\frac{2}{5} , m=-1 \right] 
\]
\end{eulerformula}
\begin{eulercomment}
no 40\\
\end{eulercomment}
\begin{eulerformula}
\[
2y^2-3y-2=0
\]
\end{eulerformula}
\begin{eulerprompt}
>$&solve(2*y^2-3*y-2=0,y)
\end{eulerprompt}
\begin{eulerformula}
\[
\left[ y=-\frac{1}{2} , y=2 \right] 
\]
\end{eulerformula}
\begin{eulercomment}
no 83\\
\end{eulercomment}
\begin{eulerformula}
\[
y^4+4y^2-5=0
\]
\end{eulerformula}
\begin{eulerprompt}
>$&solve(y^4+4*y^2-5=0,y)
\end{eulerprompt}
\begin{eulerformula}
\[
\left[ y=-1 , y=1 , y=-\sqrt{5}\,i , y=\sqrt{5}\,i \right] 
\]
\end{eulerformula}
\begin{eulercomment}
no 84\\
\end{eulercomment}
\begin{eulerformula}
\[
y^4-15y^2-16=0
\]
\end{eulerformula}
\begin{eulerprompt}
>$&solve(y^4-15*y^2-16=0,y)
\end{eulerprompt}
\begin{eulerformula}
\[
\left[ y=-i , y=i , y=-4 , y=4 \right] 
\]
\end{eulerformula}
\begin{eulerttcomment}
 Exercise 3.4 
\end{eulerttcomment}
\eulersubheading{}
\begin{eulerttcomment}
 no 1
\end{eulerttcomment}
\begin{eulerprompt}
>$&solve(1/4+1/5=1/t,t)
\end{eulerprompt}
\begin{eulerformula}
\[
\left[ t=\frac{20}{9} \right] 
\]
\end{eulerformula}
\begin{eulercomment}
no 2
\end{eulercomment}
\begin{eulerprompt}
>$&solve(1/4+1/5=1/t)$&solve(1/3-5/6=1/x,x):
\end{eulerprompt}
\begin{euleroutput}
  Maxima said:
  incorrect syntax: solve is not an infix operator
  &solve(
       ^
  
  Error in:
  $&solve(1/4+1/5=1/t)$&solve(1/3-5/6=1/x,x): ...
                                             ^
\end{euleroutput}
\begin{eulercomment}
no 6
\end{eulercomment}
\begin{eulerprompt}
>$&solve(1/t+1/2*t+1/3*t=5,t)
\end{eulerprompt}
\begin{eulerformula}
\[
\left[ t=\frac{15-\sqrt{195}}{5} , t=\frac{\sqrt{195}+15}{5}
  \right] 
\]
\end{eulerformula}
\begin{eulercomment}
no 7
\end{eulercomment}
\begin{eulerprompt}
>$&solve(5/3*x+2=3/2*x,x)
\end{eulerprompt}
\begin{euleroutput}
  Maxima said:
  solve: all variables must not be numbers.
   -- an error. To debug this try: debugmode(true);
  
  Error in:
  $&solve(5/3*x+2=3/2*x,x) ...
                          ^
\end{euleroutput}
\begin{eulercomment}
no 10
\end{eulercomment}
\begin{eulerprompt}
>$&solve(x-12/x=1,x)
\end{eulerprompt}
\begin{euleroutput}
  Maxima said:
  solve: all variables must not be numbers.
   -- an error. To debug this try: debugmode(true);
  
  Error in:
  $&solve(x-12/x=1,x) ...
                     ^
\end{euleroutput}
\begin{eulercomment}
no 18
\end{eulercomment}
\begin{eulerprompt}
>$&solve(3*y+5/y^2+5*y +y+4/y+5=y+1/y,y)
\end{eulerprompt}
\begin{eulerformula}
\[
\left[ y=-\frac{47\,\left(\frac{\sqrt{3}\,i}{2}-\frac{1}{2}\right)
 }{576\,\left(\frac{\sqrt{35539}}{128\,3^{\frac{3}{2}}}-\frac{3905}{
 13824}\right)^{\frac{1}{3}}}+\left(\frac{\sqrt{35539}}{128\,3^{
 \frac{3}{2}}}-\frac{3905}{13824}\right)^{\frac{1}{3}}\,\left(-\frac{
 \sqrt{3}\,i}{2}-\frac{1}{2}\right)-\frac{5}{24} , y=\left(\frac{
 \sqrt{35539}}{128\,3^{\frac{3}{2}}}-\frac{3905}{13824}\right)^{
 \frac{1}{3}}\,\left(\frac{\sqrt{3}\,i}{2}-\frac{1}{2}\right)-\frac{
 47\,\left(-\frac{\sqrt{3}\,i}{2}-\frac{1}{2}\right)}{576\,\left(
 \frac{\sqrt{35539}}{128\,3^{\frac{3}{2}}}-\frac{3905}{13824}\right)
 ^{\frac{1}{3}}}-\frac{5}{24} , y=\left(\frac{\sqrt{35539}}{128\,3^{
 \frac{3}{2}}}-\frac{3905}{13824}\right)^{\frac{1}{3}}-\frac{47}{576
 \,\left(\frac{\sqrt{35539}}{128\,3^{\frac{3}{2}}}-\frac{3905}{13824}
 \right)^{\frac{1}{3}}}-\frac{5}{24} \right] 
\]
\end{eulerformula}
\begin{eulercomment}
Exercise 3.5 \\
\end{eulercomment}
\eulersubheading{}
\begin{eulerprompt}
>&load(fourier_elim)
\end{eulerprompt}
\begin{euleroutput}
  
          C:/Program Files/Euler x64/maxima/share/maxima/5.35.1/share/f\(\backslash\)
  ourier_elim/fourier_elim.lisp
  
\end{euleroutput}
\begin{eulercomment}
no 44
\end{eulercomment}
\begin{eulerprompt}
>$&fourier_elim([4*x]>20,[x]) // 4*x > 20
\end{eulerprompt}
\begin{euleroutput}
  Maxima said:
  Function "$elim" expects a symbol, instead found -2
   -- an error. To debug this try: debugmode(true);
  
  Error in:
  $&fourier_elim([4*x]>20,[x]) // 4*x > 20 ...
                               ^
\end{euleroutput}
\begin{eulercomment}
no 45
\end{eulercomment}
\begin{eulerprompt}
>$&fourier_elim([x+8]<9,[x])// x+8<9
\end{eulerprompt}
\begin{euleroutput}
  Maxima said:
  Function "$elim" expects a symbol, instead found -2
   -- an error. To debug this try: debugmode(true);
  
  Error in:
  $&fourier_elim([x+8]<9,[x])// x+8<9 ...
                             ^
\end{euleroutput}
\begin{eulercomment}
no 47
\end{eulercomment}
\begin{eulerprompt}
>$&fourier_elim([x+8]>= 9,[x])//x+8 >=9
\end{eulerprompt}
\begin{euleroutput}
  Maxima said:
  Function "$elim" expects a symbol, instead found -2
   -- an error. To debug this try: debugmode(true);
  
  Error in:
  $&fourier_elim([x+8]>= 9,[x])//x+8 >=9 ...
                               ^
\end{euleroutput}
\begin{eulercomment}
no 52
\end{eulercomment}
\begin{eulerprompt}
>$&fourier_elim([3*x+4]<13,[x])//3*x+4<13
\end{eulerprompt}
\begin{euleroutput}
  Maxima said:
  Function "$elim" expects a symbol, instead found -2
   -- an error. To debug this try: debugmode(true);
  
  Error in:
  $&fourier_elim([3*x+4]<13,[x])//3*x+4<13 ...
                                ^
\end{euleroutput}
\begin{eulercomment}
no 62
\end{eulercomment}
\begin{eulerprompt}
>$&fourier_elim([3*x+5]<0,[x])//3*x+5<0
\end{eulerprompt}
\begin{euleroutput}
  Maxima said:
  Function "$elim" expects a symbol, instead found -2
   -- an error. To debug this try: debugmode(true);
  
  Error in:
  $&fourier_elim([3*x+5]<0,[x])//3*x+5<0 ...
                               ^
\end{euleroutput}
\begin{eulercomment}
Chapter 3\\
\end{eulercomment}
\eulersubheading{}
\begin{eulercomment}
no 8
\end{eulercomment}
\begin{eulerprompt}
>$&solve(3/3*x+4 + 2/x-1 =2,x)
\end{eulerprompt}
\begin{euleroutput}
  Maxima said:
  solve: all variables must not be numbers.
   -- an error. To debug this try: debugmode(true);
  
  Error in:
  $&solve(3/3*x+4 + 2/x-1 =2,x) ...
                               ^
\end{euleroutput}
\begin{eulerprompt}
>$&load(fourier_elim)
\end{eulerprompt}
\begin{eulercomment}
no 11
\end{eulercomment}
\begin{eulerprompt}
>$&fourier_elim([x+4]=7,[x])//x+4=7
\end{eulerprompt}
\begin{euleroutput}
  Maxima said:
  Function "$elim" expects a symbol, instead found -2
   -- an error. To debug this try: debugmode(true);
  
  Error in:
  $&fourier_elim([x+4]=7,[x])//x+4=7 ...
                             ^
\end{euleroutput}
\begin{eulercomment}
no 12
\end{eulercomment}
\begin{eulerprompt}
>$&fourier_elim([4*y-3]=5,[x])//4*y-3=5
\end{eulerprompt}
\begin{euleroutput}
  Maxima said:
  Function "$elim" expects a symbol, instead found -2
   -- an error. To debug this try: debugmode(true);
  
  Error in:
  $&fourier_elim([4*y-3]=5,[x])//4*y-3=5 ...
                               ^
\end{euleroutput}
\begin{eulercomment}
no 13
\end{eulercomment}
\begin{eulerprompt}
>$&fourier_elim([x+3]<=4,[x])//x+3<=4
\end{eulerprompt}
\begin{euleroutput}
  Maxima said:
  Function "$elim" expects a symbol, instead found -2
   -- an error. To debug this try: debugmode(true);
  
  Error in:
  $&fourier_elim([x+3]<=4,[x])//x+3<=4 ...
                              ^
\end{euleroutput}
\begin{eulercomment}
no 15
\end{eulercomment}
\begin{eulerprompt}
>$&fourier_elim([x+5]>2,[x])//x+5>2
\end{eulerprompt}
\begin{euleroutput}
  Maxima said:
  Function "$elim" expects a symbol, instead found -2
   -- an error. To debug this try: debugmode(true);
  
  Error in:
  $&fourier_elim([x+5]>2,[x])//x+5>2 ...
                             ^
\end{euleroutput}
\begin{eulercomment}
no 19
\end{eulercomment}
\begin{eulerprompt}
>$&solve(x^2+4*x =1,x)
\end{eulerprompt}
\begin{euleroutput}
  Maxima said:
  solve: all variables must not be numbers.
   -- an error. To debug this try: debugmode(true);
  
  Error in:
  $&solve(x^2+4*x =1,x) ...
                       ^
\end{euleroutput}
\eulerheading{4.1 Exercise Set}
\begin{eulercomment}
Use the substitution to determine whether 2,3 and -1 are zeros of\\
Nomor 23
\end{eulercomment}
\begin{eulerprompt}
>function P(x) &= (x^3-9*x^2+14*x+24); $&P(x)
\end{eulerprompt}
\begin{eulerformula}
\[
-48
\]
\end{eulerformula}
\begin{eulerprompt}
>P(4)
\end{eulerprompt}
\begin{euleroutput}
       -48.00 
\end{euleroutput}
\begin{eulerprompt}
>P(5)
\end{eulerprompt}
\begin{euleroutput}
       -48.00 
\end{euleroutput}
\begin{eulerprompt}
>P(-2)
\end{eulerprompt}
\begin{euleroutput}
       -48.00 
\end{euleroutput}
\begin{eulercomment}
Jadi hasil substitusi yang menghasilkan persamaan mempunyai nilai nol
adalah dengan mensubtsisusi angka 4

Nomor 24
\end{eulercomment}
\begin{eulerprompt}
>function P(x) &= (2*x^3-3*x^2+x+6);$&P(x)
\end{eulerprompt}
\begin{eulerformula}
\[
-24
\]
\end{eulerformula}
\begin{eulerprompt}
>P(2)
\end{eulerprompt}
\begin{euleroutput}
       -24.00 
\end{euleroutput}
\begin{eulerprompt}
>P(3)
\end{eulerprompt}
\begin{euleroutput}
       -24.00 
\end{euleroutput}
\begin{eulerprompt}
>P(-1)
\end{eulerprompt}
\begin{euleroutput}
       -24.00 
\end{euleroutput}
\begin{eulercomment}
Jadi hasil substitusi yang menghasilkan persamaan mempunyai nilai nol
adalah dengan mensubtsisusi angka -1

Nomor 25
\end{eulercomment}
\begin{eulerprompt}
>function P(x) &= (x^4-6*x^3+8*x^2+6*x-9);$&P(x)
\end{eulerprompt}
\begin{eulerformula}
\[
75
\]
\end{eulerformula}
\begin{eulerprompt}
>P(2)
\end{eulerprompt}
\begin{euleroutput}
        75.00 
\end{euleroutput}
\begin{eulerprompt}
>P(3)
\end{eulerprompt}
\begin{euleroutput}
        75.00 
\end{euleroutput}
\begin{eulerprompt}
>P(-1)
\end{eulerprompt}
\begin{euleroutput}
        75.00 
\end{euleroutput}
\begin{eulercomment}
Jadi hasil substitusi yang menghasilkan persamaan mempunyai nilai nol
adalah dengan mensubtsisusi angka 3 dan -1

Nomor 37
\end{eulercomment}
\begin{eulerprompt}
>$&solve(x^4-4*x^2+3,x)
\end{eulerprompt}
\begin{euleroutput}
  Maxima said:
  solve: all variables must not be numbers.
   -- an error. To debug this try: debugmode(true);
  
  Error in:
  $&solve(x^4-4*x^2+3,x) ...
                        ^
\end{euleroutput}
\begin{eulercomment}
Nomor 39
\end{eulercomment}
\begin{eulerprompt}
>$&solve(x^3+3*x^2-x-3,x)
\end{eulerprompt}
\begin{euleroutput}
  Maxima said:
  solve: all variables must not be numbers.
   -- an error. To debug this try: debugmode(true);
  
  Error in:
  $&solve(x^3+3*x^2-x-3,x) ...
                          ^
\end{euleroutput}
\begin{eulercomment}
Exercise 4.3\\
\end{eulercomment}
\eulersubheading{}
\begin{eulercomment}
no\\
For the function\\
\end{eulercomment}
\begin{eulerformula}
\[
f(x)= x^4-6x^3+x^2+24x-20
\]
\end{eulerformula}
\begin{eulercomment}
use long division to determine whether each of the following is a
factor of f(x)

a) x+1\\
b) x-2\\
c) x + 5
\end{eulercomment}
\begin{eulerprompt}
>function f(x) &= (x^4-6*x^3+x^2+24*x-20);$&f(x)
\end{eulerprompt}
\begin{eulerformula}
\[
0
\]
\end{eulerformula}
\begin{eulerprompt}
>$&f(x+1), $&expand(%)
\end{eulerprompt}
\begin{eulerformula}
\[
0
\]
\end{eulerformula}
\begin{eulerformula}
\[
0
\]
\end{eulerformula}
\begin{eulerprompt}
>$&f(x-2), $&expand(%)
\end{eulerprompt}
\begin{eulerformula}
\[
0
\]
\end{eulerformula}
\begin{eulerformula}
\[
0
\]
\end{eulerformula}
\begin{eulerprompt}
>$&f(x+5), $&expand(%)
\end{eulerprompt}
\begin{eulerformula}
\[
0
\]
\end{eulerformula}
\begin{eulerformula}
\[
0
\]
\end{eulerformula}
\begin{eulercomment}
no 23\\
Use synthetic division to find the function values.\\
\end{eulercomment}
\begin{eulerformula}
\[
f(x) = x^3-6x^2+11x-6
\]
\end{eulerformula}
\begin{eulercomment}
find f(1), f(-2), dan f(3)
\end{eulercomment}
\begin{eulerprompt}
>function f(x) &= (x^3-6*x^2+11*x-6);$&f(x)
\end{eulerprompt}
\begin{eulerformula}
\[
-60
\]
\end{eulerformula}
\begin{eulerprompt}
>f(1)
\end{eulerprompt}
\begin{euleroutput}
       -60.00 
\end{euleroutput}
\begin{eulerprompt}
>f(-2)
\end{eulerprompt}
\begin{euleroutput}
       -60.00 
\end{euleroutput}
\begin{eulerprompt}
>f(3)
\end{eulerprompt}
\begin{euleroutput}
       -60.00 
\end{euleroutput}
\begin{eulercomment}
no 24\\
\end{eulercomment}
\begin{eulerformula}
\[
f(x)=x63+7x^2-12x-3
\]
\end{eulerformula}
\begin{eulercomment}
find f(-3),f(-2), dan f(1)
\end{eulercomment}
\begin{eulerprompt}
>function f(x) &= (x^3+7*x^2-12*x-3);$&f(x)
\end{eulerprompt}
\begin{eulerformula}
\[
41
\]
\end{eulerformula}
\begin{eulerprompt}
>f(-3)
\end{eulerprompt}
\begin{euleroutput}
        41.00 
\end{euleroutput}
\begin{eulerprompt}
>f(-2)
\end{eulerprompt}
\begin{euleroutput}
        44.00 
\end{euleroutput}
\begin{eulerprompt}
>f(1)
\end{eulerprompt}
\begin{euleroutput}
        44.00 
\end{euleroutput}
\begin{eulercomment}
no 25\\
\end{eulercomment}
\begin{eulerformula}
\[
f(x) = x^4-3x^2+2x+8
\]
\end{eulerformula}
\begin{eulercomment}
find f(-1),f(4) dan f(-5)
\end{eulercomment}
\begin{eulerprompt}
>function f(x) &= (x^4-3*x^3+2*x+8);$&f(x)
\end{eulerprompt}
\begin{eulerformula}
\[
44
\]
\end{eulerformula}
\begin{eulerprompt}
>f(-1)
\end{eulerprompt}
\begin{euleroutput}
        44.00 
\end{euleroutput}
\begin{eulerprompt}
>f(4)
\end{eulerprompt}
\begin{euleroutput}
        44.00 
\end{euleroutput}
\begin{eulerprompt}
>f(-5)
\end{eulerprompt}
\begin{euleroutput}
        44.00 
\end{euleroutput}
\begin{eulercomment}
Factor the polynomial function f(x). Then Solve the equation f(x)=0\\
no 39\\
\end{eulercomment}
\begin{eulerformula}
\[
f(x)=x^3+4x^2+x-6
\]
\end{eulerformula}
\begin{eulerprompt}
>fx &= (x^3+4*x^2+x-6=0); $&fx
\end{eulerprompt}
\begin{eulerformula}
\[
0=0
\]
\end{eulerformula}
\begin{eulerprompt}
>$&factor((fx,x^3+4*x^2+x-6=0))
\end{eulerprompt}
\begin{eulerformula}
\[
0=0
\]
\end{eulerformula}
\begin{eulerprompt}
>$&solve(x^3+4*x^2+x-6=0,x)
\end{eulerprompt}
\begin{euleroutput}
  Maxima said:
  solve: all variables must not be numbers.
   -- an error. To debug this try: debugmode(true);
  
  Error in:
  $&solve(x^3+4*x^2+x-6=0,x) ...
                            ^
\end{euleroutput}
\begin{eulercomment}
no 40\\
\end{eulercomment}
\begin{eulerformula}
\[
f(x)=x^3+5x^2-2x-24
\]
\end{eulerformula}
\begin{eulerprompt}
>fx &= (x^3+5*x^2-2*x-24=0); $&fx
\end{eulerprompt}
\begin{eulerformula}
\[
-8=0
\]
\end{eulerformula}
\begin{eulerprompt}
>$&factor((fx,x^3+5*x^2-2*x-24=0))
\end{eulerprompt}
\begin{eulerformula}
\[
-8=0
\]
\end{eulerformula}
\begin{eulerprompt}
>$&solve(x^3+5*x^2-2*x-24=0,x)
\end{eulerprompt}
\begin{euleroutput}
  Maxima said:
  solve: all variables must not be numbers.
   -- an error. To debug this try: debugmode(true);
  
  Error in:
  $&solve(x^3+5*x^2-2*x-24=0,x) ...
                               ^
\end{euleroutput}
\begin{eulercomment}
Mid-Chapter Mixed Review\\
\end{eulercomment}
\eulersubheading{}
\begin{eulercomment}
Use synthetic division to find the function values\\
no 18\\
\end{eulercomment}
\begin{eulerformula}
\[
g(x) = x^3-9x^2+4x-10
\]
\end{eulerformula}
\begin{eulercomment}
find g(-5)
\end{eulercomment}
\begin{eulerprompt}
>function g(x) &= (x^3-9*x^2+4*x-10);$&g(x)
\end{eulerprompt}
\begin{eulerformula}
\[
-62
\]
\end{eulerformula}
\begin{eulerprompt}
>g(-5)
\end{eulerprompt}
\begin{euleroutput}
       -62.00 
\end{euleroutput}
\begin{eulercomment}
no 19\\
\end{eulercomment}
\begin{eulerformula}
\[
f(x)=20x^2-40x
\]
\end{eulerformula}
\begin{eulercomment}
find f(1/2)
\end{eulercomment}
\begin{eulerprompt}
>function f(x) &= (20*x^2-40*x);$&f(x)
\end{eulerprompt}
\begin{eulerformula}
\[
160
\]
\end{eulerformula}
\begin{eulerprompt}
>f(1/2)
\end{eulerprompt}
\begin{euleroutput}
       160.00 
\end{euleroutput}
\begin{eulercomment}
no\\
-1,5;\\
\end{eulercomment}
\begin{eulerformula}
\[
f(x)=x^6-35x^4+259x^2-225
\]
\end{eulerformula}
\begin{eulerprompt}
>function f(x) &= (x^6-35*x^4+259*x^2-225);$&f(x)
\end{eulerprompt}
\begin{eulerformula}
\[
315
\]
\end{eulerformula}
\begin{eulerprompt}
>f(-1.5)
\end{eulerprompt}
\begin{euleroutput}
       315.00 
\end{euleroutput}
\begin{eulercomment}
Factor the polynomial function f(x). Then solve the equation f(x) =0.\\
no 23\\
\end{eulercomment}
\begin{eulerformula}
\[
h(x) = x^3-2x^2-55x+56
\]
\end{eulerformula}
\begin{eulerprompt}
>hx &= (x^3-2*x^2-55*x+56=0); $&hx
\end{eulerprompt}
\begin{eulerformula}
\[
150=0
\]
\end{eulerformula}
\begin{eulerprompt}
>$&factor((hx,x^3-2*x^2-55*x+56=0))
\end{eulerprompt}
\begin{eulerformula}
\[
150=0
\]
\end{eulerformula}
\begin{eulerprompt}
>$&solve(x^3-2*x^2-55*x+56=0,x)
\end{eulerprompt}
\begin{euleroutput}
  Maxima said:
  solve: all variables must not be numbers.
   -- an error. To debug this try: debugmode(true);
  
  Error in:
  $&solve(x^3-2*x^2-55*x+56=0,x) ...
                                ^
\end{euleroutput}
\begin{eulercomment}
no 24\\
\end{eulercomment}
\begin{eulerformula}
\[
g(x) = x^4-2x^3-13x^2+14x+24
\]
\end{eulerformula}
\begin{eulerprompt}
>gx &= (x^4-2*x^3-13*x^2+14*x+24=0); $&gx
\end{eulerprompt}
\begin{eulerformula}
\[
-24=0
\]
\end{eulerformula}
\begin{eulerprompt}
>$&factor((gx,x^4-2*x^3-13*x^2+14*x+24=0))
\end{eulerprompt}
\begin{eulerformula}
\[
-24=0
\]
\end{eulerformula}
\begin{eulerprompt}
>$&solve(x^4-2*x^3-13*x^2+14*x+24=0,x)
\end{eulerprompt}
\begin{euleroutput}
  Maxima said:
  solve: all variables must not be numbers.
   -- an error. To debug this try: debugmode(true);
  
  Error in:
  $&solve(x^4-2*x^3-13*x^2+14*x+24=0,x) ...
                                       ^
\end{euleroutput}
\begin{eulerprompt}
>function g(x,a=1) :=a*x^3+2*(a*x)^2+4*a*x-10
>g(-5)
\end{eulerprompt}
\begin{euleroutput}
      -105.00 
\end{euleroutput}
\begin{eulerprompt}
>function f(x,a=1) :=5*(a*x)^4+a*x^3-x
>f(-(2^(-1/2)))
\end{eulerprompt}
\begin{euleroutput}
         1.60 
\end{euleroutput}
\begin{eulerprompt}
>function f(x,a=1) :=10*a*x^2-40*a*x
>f(1/2)
\end{eulerprompt}
\begin{euleroutput}
       -17.50 
\end{euleroutput}
\begin{eulerprompt}
>$&solve((x^5-5)/(x+1))
\end{eulerprompt}
\begin{eulerformula}
\[
\left[  \right] 
\]
\end{eulerformula}
\begin{eulerprompt}
>$&solve((3*x^4-x^3+2*x^2-6*x)/(x-2))
\end{eulerprompt}
\begin{eulerformula}
\[
\left[  \right] 
\]
\end{eulerformula}
\end{eulernotebook}
\end{document}
